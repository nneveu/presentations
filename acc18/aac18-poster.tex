\documentclass[portrait,final,paperwidth=111.8cm, paperheight=86.4cm,  fontscale=0.277]{baposter}
%a0paper,
\usepackage{calc}
\usepackage{graphicx}
\usepackage{amsmath}
\usepackage{amssymb}
\usepackage{relsize}
\usepackage{multirow}
\usepackage{rotating}
\usepackage{bm}
\usepackage{url}
\usepackage{setspace}
\usepackage{graphicx}
\usepackage{multicol}
\usepackage{siunitx}
\usepackage{vwcol} 
\usepackage{varwidth}
\usepackage{tikz}
\usepackage{graphicx}
\graphicspath{{/home/nicole/Documents/presentations/ipac2018/logos}{/home/nicole/Documents/presentations/ipac2018/tba-ipac18}}
\usetikzlibrary{calc}
%\usepackage{times}
%\usepackage{helvet}
%\usepackage{bookman}
\usepackage{palatino}
\newcommand\Tstrut{\rule{0pt}{2.6ex}}         % = `top' strut

\newcommand{\captionfont}{\footnotesize}

\graphicspath{{images/}{../images/}}
\usepackage{tikz}
\usetikzlibrary{shapes.misc}
\usetikzlibrary{shapes,arrows,decorations.markings,shadows,positioning}
\usetikzlibrary{snakes,backgrounds,spy}
\usepackage{tcolorbox}
\newcommand{\SET}[1]  {\ensuremath{\mathcal{#1}}}
\newcommand{\MAT}[1]  {\ensuremath{\boldsymbol{#1}}}
\newcommand{\VEC}[1]  {\ensuremath{\boldsymbol{#1}}}
\newcommand{\Video}{\SET{V}}
\newcommand{\video}{\VEC{f}}
\newcommand{\track}{x}
\newcommand{\Track}{\SET T}
\newcommand{\LMs}{\SET L}
\newcommand{\lm}{l}
\newcommand{\PosE}{\SET P}
\newcommand{\posE}{\VEC p}
\newcommand{\negE}{\VEC n}
\newcommand{\NegE}{\SET N}
\newcommand{\Occluded}{\SET O}
\newcommand{\occluded}{o}
\renewcommand{\baselinestretch}{-1.5} 
%%%%%%%%%%%%%%%%%%%%%%%%%%%%%%%%%%%%%%%%%%%%%%%%%%%%%%%%%%%%%%%%%%%%%%%%%%%%%%%%
%%%% Some math symbols used in the text
%%%%%%%%%%%%%%%%%%%%%%%%%%%%%%%%%%%%%%%%%%%%%%%%%%%%%%%%%%%%%%%%%%%%%%%%%%%%%%%%

%%%%%%%%%%%%%%%%%%%%%%%%%%%%%%%%%%%%%%%%%%%%%%%%%%%%%%%%%%%%%%%%%%%%%%%%%%%%%%%%
% Multicol Settings
%%%%%%%%%%%%%%%%%%%%%%%%%%%%%%%%%%%%%%%%%%%%%%%%%%%%%%%%%%%%%%%%%%%%%%%%%%%%%%%%
\setlength{\columnsep}{1.5em}
\setlength{\columnseprule}{0mm}

%%%%%%%%%%%%%%%%%%%%%%%%%%%%%%%%%%%%%%%%%%%%%%%%%%%%%%%%%%%%%%%%%%%%%%%%%%%%%%%%
% Save space in lists. Use this after the opening of the list
%%%%%%%%%%%%%%%%%%%%%%%%%%%%%%%%%%%%%%%%%%%%%%%%%%%%%%%%%%%%%%%%%%%%%%%%%%%%%%%%
\newcommand{\compresslist}{%
	\setlength{\itemsep}{1pt}%
	\setlength{\parskip}{0pt}%
	\setlength{\parsep}{0pt}%
}

%%%%%%%%%%%%%%%%%%%%%%%%%%%%%%%%%%%%%%%%%%%%%%%%%%%%%%%%%%%%%%%%%%%%%%%%%%%%%%
%%% Begin of Document
%%%%%%%%%%%%%%%%%%%%%%%%%%%%%%%%%%%%%%%%%%%%%%%%%%%%%%%%%%%%%%%%%%%%%%%%%%%%%%
\begin{document}

%%%%%%%%%%%%%%%%%%%%%%%%%%%%%%%%%%%%%%%%%%%%%%%%%%%%%%%%%%%%%%%%%%%%%%%%%%%%%%
%%% Here starts the poster
%%%---------------------------------------------------------------------------
%%% Format it to your taste with the options
%%%%%%%%%%%%%%%%%%%%%%%%%%%%%%%%%%%%%%%%%%%%%%%%%%%%%%%%%%%%%%%%%%%%%%%%%%%%%%
\definecolor{silver}{cmyk}{0,0,0,0.3}
\definecolor{black}{cmyk}{0,0,0.0,1.0}
\definecolor{darkYellow}{cmyk}{0,0,1.0,0.5}
\definecolor{darkSilver}{cmyk}{0,0,0,0.1}

\definecolor{KTHBlue}{cmyk}{.71,.37,0.07,0}
\definecolor{KTHsilver}{cmyk}{0,0,0,0.35}
\definecolor{KTHbeige}{cmyk}{0,0.03,0.19,0.04}


\begin{poster}{
  % Poster Options
	% Show grid to help with alignment
	grid=false,
	columns=6,
	% Column spacing
	colspacing=1em,
	% Color style
	bgColorOne=white,
	bgColorTwo=white,
	borderColor=KTHBlue,
	headerColorOne=black,
	headerColorTwo=KTHBlue,
	headerFontColor=white,
	boxColorOne=white,
	boxColorTwo=KTHBlue,
	% Format of textbox
	textborder=roundedleft,
	% Format of text header
	eyecatcher=true,
	headerborder=closed,
	headerheight=0.13\textheight,
	%  textfont=\sc, An example of changing the text font
	headershape=roundedright,
	headershade=shadelr,
	headerfont=\Large\bf\textsc, %Sans Serif
	textfont={\setlength{\parindent}{0em}},
	boxshade=plain,
	%  background=shade-tb,
	background=plain,
	linewidth=2pt
}
% Eye Catcher
{
	
	%\makebox[3em][l]{
		
} %}
% Title
{\bf\textsc{Status of TBA Layout at the AWA }}%\vspace{10em}}
% Authors
{\vspace{1em}
	N. Neveu\textsuperscript{1}, L. Spentzouris, Illinois Institute of Technology, Chicago, IL, USA \\
	J. G. Power, W. Gai, M. Conde \textsuperscript{1}Argonne National  Laboratory, Lemont, IL, USA \\
	C. Jing, Euclid Techlabs LLC, Solon, OH, USA \\
    }
% University logo
{% The makebox allows the title to flow into the logo, this is a hack because of the L shaped logo.
	%\makebox[3em][r]{
		\includegraphics[height=9em]{../logos/combo_logo}%}
}


%%%%%%%%%%%%%%%%%%%%%%%%%%%%%%%%%%%%%%%%%%%%%%%%%%%%%%%%%%%%%%%%%%%%%%%%%%%%%%
\headerbox{Abstract}{name=problem,column=0,row=0, span=2}{
%%%%%%%%%%%%%%%%%%%%%%%%%%%%%%%%%%%%%%%%%%%%%%%%%%%%%%%%%%%%%%%%%%%%%%%%%%%%%%
The Argonne Wakefield Accelerator (AWA) Facility has demonstrated Two Beam Acceleration (TBA) using metallic and dielectric structures. A two stage demonstration was accomplished recently (i.e. staging), although the stages were not independently powered. Design efforts are underway for an additional beam line with a kicker, septum magnet, and dipole in a dogleg configuration. The additional line will allow bunch trains to be independently routed so that each train powers only one accelerating structure. This will push the gradient in each stage.
}


%%%%%%%%%%%%%%%%%%%%%%%%%%%%%%%%%%%%%%%%%%%%%%%%%%%%%%%%%%%%%%%%%%%%%%%%%%%%%%
\headerbox{AWA FAcility}{name=sims,column=2,row=0,span=2, bottomaligned=problem}{
%%%%%%%%%%%%%%%%%%%%%%%%%%%%%%%%%%%%%%%%%%%%%%%%%%%%%%%%%%%%%%%%%%%%%%%%%%%%%%
The AWA facility houses two \SI{1.3}{GHz} rf photoinjectors.
Typical operating charges are 1, 4, 10, and \SI{40}{nC}. 
Recent experiments include emittance exchange, 
high gradient structure tests, thermal emittance measurements, 
and two beam acceleration (TBA). 
The AWA facility lends well to TBA experiments due to the 
close proximity of both operational photoinjectors. 
For TBA experiments, the witness line is operated at \SI{1}{nC} 
and the drive line is operated at \SI{40}{nC} per bunch.
The witness line is operated in single bunch mode, and 
the drive line supplies high charge bunch trains. 
}


%%%%%%%%%%%%%%%%%%%%%%%%%%%%%%%%%%%%%%%%%%%%%%%%%%%%%%%%%%%%%%%%%%%%%%%%%%%%%%
\headerbox{Two Beam Acceleration Layout}{name=beamline,column=0,below=sims, span=4}{%, bottomaligned=conclusion}{
%%%%%%%%%%%%%%%%%%%%%%%%%%%%%%%%%%%%%%%%%%%%%%%%%%%%%%%%%%%%%%%%%%%%%%%%%%%%%%
%\noindent
\begin{centering}
	%\begin{tikzpicture}[scale=\textwidth/33cm, text=black]
	\begin{tikzpicture}[scale=0.57, text=black]
	\input{/home/nicole/Documents/thesis/beamer/long_talk/full_staging.tex}
	\end{tikzpicture}
\end{centering}

\vspace{1em}
The arrows indicate what direction the beams travels.
The guns are located at opposite ends of the bunker and 
the propagation direction of the beam lines are opposing.
PETS stands for Power Extraction and Transfer Structure, and ACC stands for Accelerating structure. 
The subscript on each structure refers to which stage the structures belong to (first or second). 
The drive line has six accelerating cavities with a maximum beam energy of \SI{70}{MeV}. 
The witness line has one accelerating cavity with a max beam energy of \SI{15}{MeV}.
}

%%%%%%%%%%%%%%%%%%%%%%%%%%%%%%%%%%%%%%%%%%%%%%%%%%%%%%%%%%%%%%%%%%%%%%%%%%%%%%
\headerbox{Experimental Stage}{name=stage3,column=4, span=2}{
%%%%%%%%%%%%%%%%%%%%%%%%%%%%%%%%%%%%%%%%%%%%%%%%%%%%%%%%%%%%%%%%%%%%%%%%%%%%%
\centering
\begin{tikzpicture}[every node/.style={anchor=south west,inner sep=0pt},x=1mm, y=1mm,]   
\node (fig1) at (0,0)
{\includegraphics[width=1\textwidth]{/home/nicole/Documents/presentations/space_charge_2017/stage}};
\node[fill=white, inner sep=2pt] (txt2) at (10,10) {ACC};
\node[fill=white, inner sep=2pt] (txt2) at (57,30) {PETS};
\node[fill=white, inner sep=2pt, rotate=32] (txt2) at (27,29) {Waveguide};
\end{tikzpicture}

\vspace{1em}
Example of a TBA stage at the AWA. This picture was taken during 
a past TBA experiment.  

High charge bunch trains are supplied to Power Extraction
and Transfer Structures (PETS) downstream. These are decelerating 
structures that extract power from the bunches through wakefield generation.
Supperposition combines the wake from each bunch into one rf pulse.
The high power pulse generated by the combination of wakes
is transfered through a waveguide to the witness line. 
The accelerating structures, ACC$_{1,2}$ are only 
powered by the PETS on the drive line. There is no  
external power source.
	
}

%%%%%%%%%%%%%%%%%%%%%%%%%%%%%%%%%%%%%%%%%%%%%%%%%%%%%%%%%%%%%%%%%%%%%%%%%%%%%%
\headerbox{Kicker Test}{name=kicker,column=1,row=0,span=2, below=beamline}{%, bottomaligned=stage3}{
%%%%%%%%%%%%%%%%%%%%%%%%%%%%%%%%%%%%%%%%%%%%%%%%%%%%%%%%%%%%%%%%%%%%%%%%%%%%%%

		A kicker was specifically fabricated 
		for this experiment. The initial design was adapted from 
		work done at Indiana University. The plates were lengthened
		to increase the angle and the gap adjusted based on 
		beam size simulations and mechanical constraints at the AWA.
		The plates will be 
		operated in differential mode, to get the highest field
		possible from the available pulsar.


}


%%%%%%%%%%%%%%%%%%%%%%%%%%%%%%%%%%%%%%%%%%%%%%%%%%%%%%%%%%%%%%%%%%%%%%%%%%%%%%
\headerbox{Optimization}{name=kicker,column=2,row=0,span=4, below=stage3}{
%%%%%%%%%%%%%%%%%%%%%%%%%%%%%%%%%%%%%%%%%%%%%%%%%%%%%%%%%%%%%%%%%%%%%%%%%%%%%%
The parameters on the drive line are further complicated
by strong space charge forces and bending elements.
A first round multi-objective optimization of the drive line has been performed 
using the built in genetic algorithm (GA) in OPAL-T . 
These simulations included all elements leading up 
to the entrance of the septum.
The objectives were beam size and energy spread at the wakefield structure.


}


%%%%%%%%%%%%%%%%%%%%%%%%%%%%%%%%%%%%%%%%%%%%%%%%%%%%%%%%%%%%%%%%%%%%%%%%%%%%%%
\headerbox{Acknowlegements}{name=ref,column=4,above=bottom, span=2}{%, aligned=ref}{%
%%%%%%%%%%%%%%%%%%%%%%%%%%%%%%%%%%%%%%%%%%%%%%%%%%%%%%%%%%%%%%%%%%%%%%%%%%%%%%
\noindent 
We gratefully acknowledge the computing resources
provided on Bebop, a HPC cluster operated by the LCRC at ANL.
This work is supported by the U.S. DOE, OS,  
contract DE-AC02-06CH11357 and grant DE-SC0015479. 
Travel to IPAC'18 supported by the Division of Physics 
of the U.S. NSF 
Accelerator Science Program and the DBP of the APS.
\vspace{-1em}
\hfill
\begin{center}
%	\includegraphics[width=0.3\textwidth]{/home/nicole/Documents/presentations/logos/DOE_logo_color_cmyk}
%	\includegraphics[width=0.1\textwidth]{/home/nicole/Documents/presentations/logos/aps-logo}
\end{center}


}


\end{poster}%
%
\end{document}
