\documentclass[portrait,final,paperwidth=111.8cm, paperheight=86.4cm,  fontscale=0.277]{baposter}
%a0paper,
\usepackage{calc}
\usepackage{graphicx}
\usepackage{amsmath}
\usepackage{amssymb}
\usepackage{relsize}
\usepackage{multirow}
\usepackage{rotating}
\usepackage{bm}
\usepackage{url}
\usepackage{setspace}
\usepackage{graphicx}
\usepackage{multicol}
\usepackage{siunitx}
\usepackage{vwcol} 
\usepackage{varwidth}
\usepackage{tikz}
\usepackage{graphicx}
\graphicspath{{/home/nicole/Documents/presentations/ipac2018/logos}{/home/nicole/Documents/presentations/ipac2018/tba-ipac18}}
\usetikzlibrary{calc}
%\usepackage{times}
%\usepackage{helvet}
%\usepackage{bookman}
\usepackage{palatino}
\newcommand\Tstrut{\rule{0pt}{2.6ex}}         % = `top' strut

\newcommand{\captionfont}{\footnotesize}

\graphicspath{{images/}{../images/}}
\usepackage{tikz}
\usetikzlibrary{shapes.misc}
\usetikzlibrary{shapes,arrows,decorations.markings,shadows,positioning}
\usetikzlibrary{snakes,backgrounds,spy}
\usepackage{tcolorbox}
\newcommand{\SET}[1]  {\ensuremath{\mathcal{#1}}}
\newcommand{\MAT}[1]  {\ensuremath{\boldsymbol{#1}}}
\newcommand{\VEC}[1]  {\ensuremath{\boldsymbol{#1}}}
\newcommand{\Video}{\SET{V}}
\newcommand{\video}{\VEC{f}}
\newcommand{\track}{x}
\newcommand{\Track}{\SET T}
\newcommand{\LMs}{\SET L}
\newcommand{\lm}{l}
\newcommand{\PosE}{\SET P}
\newcommand{\posE}{\VEC p}
\newcommand{\negE}{\VEC n}
\newcommand{\NegE}{\SET N}
\newcommand{\Occluded}{\SET O}
\newcommand{\occluded}{o}
\renewcommand{\baselinestretch}{-1.5} 
%%%%%%%%%%%%%%%%%%%%%%%%%%%%%%%%%%%%%%%%%%%%%%%%%%%%%%%%%%%%%%%%%%%%%%%%%%%%%%%%
%%%% Some math symbols used in the text
%%%%%%%%%%%%%%%%%%%%%%%%%%%%%%%%%%%%%%%%%%%%%%%%%%%%%%%%%%%%%%%%%%%%%%%%%%%%%%%%

%%%%%%%%%%%%%%%%%%%%%%%%%%%%%%%%%%%%%%%%%%%%%%%%%%%%%%%%%%%%%%%%%%%%%%%%%%%%%%%%
% Multicol Settings
%%%%%%%%%%%%%%%%%%%%%%%%%%%%%%%%%%%%%%%%%%%%%%%%%%%%%%%%%%%%%%%%%%%%%%%%%%%%%%%%
\setlength{\columnsep}{1.5em}
\setlength{\columnseprule}{0mm}

%%%%%%%%%%%%%%%%%%%%%%%%%%%%%%%%%%%%%%%%%%%%%%%%%%%%%%%%%%%%%%%%%%%%%%%%%%%%%%%%
% Save space in lists. Use this after the opening of the list
%%%%%%%%%%%%%%%%%%%%%%%%%%%%%%%%%%%%%%%%%%%%%%%%%%%%%%%%%%%%%%%%%%%%%%%%%%%%%%%%
\newcommand{\compresslist}{%
	\setlength{\itemsep}{1pt}%
	\setlength{\parskip}{0pt}%
	\setlength{\parsep}{0pt}%
}

%%%%%%%%%%%%%%%%%%%%%%%%%%%%%%%%%%%%%%%%%%%%%%%%%%%%%%%%%%%%%%%%%%%%%%%%%%%%%%
%%% Begin of Document
%%%%%%%%%%%%%%%%%%%%%%%%%%%%%%%%%%%%%%%%%%%%%%%%%%%%%%%%%%%%%%%%%%%%%%%%%%%%%%
\begin{document}

%%%%%%%%%%%%%%%%%%%%%%%%%%%%%%%%%%%%%%%%%%%%%%%%%%%%%%%%%%%%%%%%%%%%%%%%%%%%%%
%%% Here starts the poster
%%%---------------------------------------------------------------------------
%%% Format it to your taste with the options
%%%%%%%%%%%%%%%%%%%%%%%%%%%%%%%%%%%%%%%%%%%%%%%%%%%%%%%%%%%%%%%%%%%%%%%%%%%%%%
\definecolor{silver}{cmyk}{0,0,0,0.3}
\definecolor{black}{cmyk}{0,0,0.0,1.0}
\definecolor{darkYellow}{cmyk}{0,0,1.0,0.5}
\definecolor{darkSilver}{cmyk}{0,0,0,0.1}

\definecolor{KTHBlue}{cmyk}{.71,.37,0.07,0}
\definecolor{KTHsilver}{cmyk}{0,0,0,0.35}
\definecolor{KTHbeige}{cmyk}{0,0.03,0.19,0.04}


\begin{poster}{
  % Poster Options
	% Show grid to help with alignment
	grid=false,
	columns=6,
	% Column spacing
	colspacing=1em,
	% Color style
	bgColorOne=white,
	bgColorTwo=white,
	borderColor=KTHBlue,
	headerColorOne=black,
	headerColorTwo=KTHBlue,
	headerFontColor=white,
	boxColorOne=white,
	boxColorTwo=KTHBlue,
	% Format of textbox
	textborder=roundedleft,
	% Format of text header
	eyecatcher=true,
	headerborder=closed,
	headerheight=0.13\textheight,
	%  textfont=\sc, An example of changing the text font
	headershape=roundedright,
	headershade=shadelr,
	headerfont=\Large\bf\textsc, %Sans Serif
	textfont={\setlength{\parindent}{0em}},
	boxshade=plain,
	%  background=shade-tb,
	background=plain,
	linewidth=2pt
}
% Eye Catcher
{
	
	%\makebox[3em][l]{
		
} %}
% Title
{\bf\textsc{Status of TBA Layout at the AWA }}%\vspace{10em}}
% Authors
{\vspace{1em}
	N. Neveu\textsuperscript{1}, L. Spentzouris, Illinois Institute of Technology, Chicago, IL, USA \\
	J. G. Power, W. Gai, M. Conde \textsuperscript{1}Argonne National  Laboratory, Lemont, IL, USA \\
	C. Jing, Euclid Techlabs LLC, Solon, OH, USA \\
    }
% University logo
{% The makebox allows the title to flow into the logo, this is a hack because of the L shaped logo.
	%\makebox[3em][r]{
		\includegraphics[height=9em]{../logos/combo_logo}%}
}


%%%%%%%%%%%%%%%%%%%%%%%%%%%%%%%%%%%%%%%%%%%%%%%%%%%%%%%%%%%%%%%%%%%%%%%%%%%%%%
\headerbox{Abstract}{name=problem,column=0,row=0, span=3}{
%%%%%%%%%%%%%%%%%%%%%%%%%%%%%%%%%%%%%%%%%%%%%%%%%%%%%%%%%%%%%%%%%%%%%%%%%%%%%%
The Argonne Wakefield Accelerator (AWA) Facility has demonstrated Two Beam Acceleration (TBA) using metallic and dielectric structures. A two stage demonstration was accomplished recently (i.e. staging), although the stages were not independently powered. Design efforts are underway for an additional beam line with a kicker, septum magnet, and dipole in a dogleg configuration. The additional line will allow bunch trains to be independently routed so that each train powers only one accelerating structure. This will push the gradient in each stage.

The AWA facility houses two \SI{1.3}{GHz} rf photoinjectors.
Typical operating charges are 1, 4, 10, and \SI{40}{nC}. 
Recent experiments include emittance exchange, 
high gradient structure tests, thermal emittance measurements, 
and two beam acceleration (TBA). 
The AWA facility lends well to TBA experiments due to the 
close proximity of both operational photoinjectors. 
For TBA experiments, the witness line is operated at \SI{1}{nC} 
and the drive line is operated at \SI{40}{nC} per bunch.
The witness line is operated in single bunch mode, and 
the drive line supplies high charge bunch trains. 
}



%%%%%%%%%%%%%%%%%%%%%%%%%%%%%%%%%%%%%%%%%%%%%%%%%%%%%%%%%%%%%%%%%%%%%%%%%%%%%%
\headerbox{Two Beam Acceleration Layout}{name=beamline,column=0, span=3, below=problem}{%, bottomaligned=conclusion}{
%%%%%%%%%%%%%%%%%%%%%%%%%%%%%%%%%%%%%%%%%%%%%%%%%%%%%%%%%%%%%%%%%%%%%%%%%%%%%%
%\noindent
\begin{centering}
	%\begin{tikzpicture}[scale=\textwidth/33cm, text=black]
	\begin{tikzpicture}[scale=0.43, text=black]
	\input{full_staging_aac.tex}
	%\node[fill=white, inner sep=2pt] (txt2) at (10,10) {ACC};
	%\node[fill=white, inner sep=2pt] (txt2) at (57,30) {PETS};
	%\node[fill=white, inner sep=2pt] (txt2) at (27,29) {Waveguide};
	\end{tikzpicture}
\end{centering}

\vspace{1em}
The arrows indicate what direction the beams travels.
The guns are located at opposite ends of the bunker and 
the the beams propogate in opposite directions.
PETS stands for Power Extraction and Transfer Structure, and ACC stands for Accelerating structure. 
The subscript on each structure refers to which stage the structures belong to (first or second). 
The drive line has six accelerating cavities with a maximum beam energy of \SI{70}{MeV}. 
The witness line has one accelerating cavity with a max beam energy of \SI{15}{MeV}.
}

%%%%%%%%%%%%%%%%%%%%%%%%%%%%%%%%%%%%%%%%%%%%%%%%%%%%%%%%%%%%%%%%%%%%%%%%%%%%%%
\headerbox{Experimental TBA Stage}{name=stage3,column=3, span=3, bottomaligned=problem}{
%%%%%%%%%%%%%%%%%%%%%%%%%%%%%%%%%%%%%%%%%%%%%%%%%%%%%%%%%%%%%%%%%%%%%%%%%%%%%
\begin{minipage}{0.47\textwidth}
	\vspace{14.5em}
	\begin{tikzpicture}[text=white]   
	{\includegraphics[width=\textwidth]{/home/nicole/Documents/presentations/space_charge_2017/stage}};
	%\node[fill=white, inner sep=2pt] (txt2) at (10,10) {ACC};
	%\node[fill=white, inner sep=2pt] (txt2) at (20,30) {PETS};
	%\node[fill=white, inner sep=2pt, rotate=32] (txt2) at (15,20) {Waveguide};
	\end{tikzpicture}
\end{minipage}\hspace{1em}
\begin{minipage}{0.5\textwidth}
	
	Example of a TBA stage at the AWA. This picture was taken during 
	a past TBA experiment.  
	
	High charge bunch trains are supplied to Power Extraction
	and Transfer Structures (PETS) downstream. These are decelerating 
	structures that extract power from the bunches through wakefield generation.
	Supperposition combines the wake from each bunch into one rf pulse.
	The high power pulse generated by the combination of wakes
	is transfered through a waveguide to the witness line. 
	The accelerating structures, ACC$_{1,2}$ are only 
	powered by the PETS on the drive line. There is no  
	external power source.
\end{minipage}
	
}

%%%%%%%%%%%%%%%%%%%%%%%%%%%%%%%%%%%%%%%%%%%%%%%%%%%%%%%%%%%%%%%%%%%%%%%%%%%%%%
\headerbox{Kicker Test}{name=kicker,column=3,row=0,span=3, below=problem}{%, bottomaligned=stage3}{
%%%%%%%%%%%%%%%%%%%%%%%%%%%%%%%%%%%%%%%%%%%%%%%%%%%%%%%%%%%%%%%%%%%%%%%%%%%%%%

		A kicker was fast rise time kicker was fabricated 
		for this experiment. The initial design was adapted from 
		work done at Indiana University. The plates were lengthened
		to increase the angle and the gap adjusted based on 
		beam size simulations and mechanical constraints at the AWA.
		It was recently installed and tested. Experimental resutls are shown below.
		
		\vspace{1em}
		
			Voltages in kV:
		
		\centering
		\includegraphics[width=0.3\textwidth]{/home/nicole/Documents/thesis_code/data_analysis/kicker_july-2018/output2/yag6_kicker_voltage0}%
		\includegraphics[width=0.3\textwidth]{/home/nicole/Documents/thesis_code/data_analysis/kicker_july-2018/output2/yag6_kicker_voltage18}%
		\includegraphics[width=0.3\textwidth]{/home/nicole/Documents/thesis_code/data_analysis/kicker_july-2018/output2/yag6_kicker_voltage26}

\centering
\vspace{1em}

Kicker off \hspace{5em} Kicker On, 18 kV  \hspace{5em} Kicker On, 26 kV

}


%%%%%%%%%%%%%%%%%%%%%%%%%%%%%%%%%%%%%%%%%%%%%%%%%%%%%%%%%%%%%%%%%%%%%%%%%%%%%%
\headerbox{Optimization}{name=opt,column=0,row=0,span=3, below=beamline}{
%%%%%%%%%%%%%%%%%%%%%%%%%%%%%%%%%%%%%%%%%%%%%%%%%%%%%%%%%%%%%%%%%%%%%%%%%%%%%%
A first round multi-objective optimization of the drive line has been performed 
using the built in genetic algorithm (GA) in OPAL-T . 
These simulations included all elements leading up 
to the entrance of the wakefield structure on the bent drive line. The charge was 40 nC, as in TBA experiments.
The objectives were beam size and energy spread at the wakefield structure.

\vspace{1em}

\begin{minipage}{0.33\textwidth}
	
	\centering	
	2D Field Maps only
	
	\includegraphics[width=\textwidth]{{/home/nicole/Documents/awa-tba/pareto_stat_plots/xyrms-optLinac-40nC_KQ3=3.2_KQ5=-1.25_KQ6=-0.25_KQ7=0_KQ8=0}.pdf}	
\end{minipage}
\begin{minipage}{0.33\textwidth}
	
	\centering
	3D Maps and CSR included
	
	\includegraphics[width=\textwidth]{{/home/nicole/Documents/awa-tba/pareto_stat_plots/xyrms-csr_fields}.pdf}		
\end{minipage}
\begin{minipage}{0.33\textwidth}
	\centering
	Adjusted Solenoid, M=225 A
	
	\includegraphics[width=\textwidth]{{/home/nicole/Documents/awa-tba/pareto_stat_plots/KQ3=3.2/xy-max-min-optLinac-40nC_KQ3=3.2_IM=225}.pdf}%
\end{minipage}

}


%%%%%%%%%%%%%%%%%%%%%%%%%%%%%%%%%%%%%%%%%%%%%%%%%%%%%%%%%%%%%%%%%%%%%%%%%%%%%%
\headerbox{Acknowlegements}{name=ref,column=3,above=bottom, below=kicker,span=3}{%, aligned=ref}{%
%%%%%%%%%%%%%%%%%%%%%%%%%%%%%%%%%%%%%%%%%%%%%%%%%%%%%%%%%%%%%%%%%%%%%%%%%%%%%%
%\noindent 
\begin{minipage}{0.5\textwidth}
	\vspace{0.5em}
	We gratefully acknowledge the computing resources
	provided on Bebop, a HPC cluster operated by the LCRC at ANL.
	This work is supported by the U.S. DOE, OS,  
	contract DE-AC02-06CH11357 and grant DE-SC0015479. 
	Travel to AAC'18 supported by 
\end{minipage}\hspace{1em}
\begin{minipage}{0.45\textwidth}
	\begin{center}
		\includegraphics[width=\textwidth]{DOE_logo_color_cmyk-eps-converted-to}
		%\includegraphics[width=0.1\textwidth]{/home/nicole/Documents/presentations/logos/aps-logo}
	\end{center}
\end{minipage}




}


\end{poster}%
%
\end{document}
