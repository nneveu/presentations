%%%%%%%%%%%%%%%%%%%%%%%%%%%%%%%%%%%%%%%%%%%%%%%%%%%%%%%%%%%%%%%%%%%%%%
%%%%%%%%% Select one of the options, and comment the rest of them

%%%%%%%%%% Option 1:  to compile with pdflatex : parameter "t" - to align to the top
\documentclass[professionalfonts,t]{beamer}
%sans font?

%%%%%%%%%% Option 3: to create handout for print
%\documentclass[t,handout]{beamer}
%\usepackage{pgfpages}              % to put several slides on one page
%\pgfpagesuselayout{2 on 1}[a4paper, border shrink=5mm]             % 2 slides on 1 page
%\pgfpagesuselayout{4 on 1}[a4paper,landscape, border shrink=5mm]   % 4 slides on 1 page, and landscaped


%%%%%%%%%%%%%%%%%%%%%%%%%%%%%%%%%%%%%%%%%%%%%%%%%%%%%%%%%%%%%%%%%%%%
%%%%%%%%%%%%%% Select the Theme %%%%%%%%%%%%%%%%%%%%%%%%%%%%%%%%%%%
\usetheme{Dresden}     % OK
%\usetheme{Berlin}
%\usetheme{Bergen}      % NO
%\usetheme{Boadilla}    % NO
%\usetheme{Copenhagen}  % NO
%\usetheme{Hannover}    % NO
%\usetheme{Luebeck}     % NO
%\usetheme{Marburg}     % NO
%\usetheme{Pittsburgh}  % NO
%\usetheme{default}
%\usetheme{Singapore}   % OK
%\usetheme{boxes}
%\usecolortheme{structure}
%\usecolortheme{rose}
%\usecolortheme{beaver}


\definecolor{mymaroon}{cmyk}{0.0, 1.0, 1.0, 0.498}
\definecolor{myblue}{cmyk}{1.0, 1, 0, 0.5}
\definecolor{mygreen}{cmyk}{100, 0, 100, 50}
\setbeamercolor*{palette secondary}{use=structure,fg=white,bg=myblue}
\setbeamercolor*{palette tertiary}{use=structure,fg=white,bg=mymaroon}

%\usepackage{beamerthemesplit}              %
\beamertemplateballitem % fancy bullets and numbering

\setbeamertemplate{navigation symbols}{}   % suppress navigation symbols
\addtobeamertemplate{frametitle}{}{%
	\logo{../images/IIT_logo}
	\iffalse
	
	\begin{tikzpicture}[remember picture,overlay]
	\node[anchor=center, yshift=-13pt, xshift=-5pt] at (current page.north) 
	{\includegraphics[height=1.1cm]{../images/Argonne_cmyk_black-eps-converted-to}\hspace{10cm}};
	
	\node[anchor=north east, yshift=3pt, xshift=0pt] at (current page.north east) 
	{\includegraphics[height=0.7cm]{../images/IIT_Logo_blk}};
	\end{tikzpicture}
    
     \fi
}
% other possibilities to include LOGO. it puts it in RLC

%
%\pgfdeclareimage[width=1cm]{logo}{../images/IIT_Logo}
%\logo{\pgfuseimage{logo}}


% load additional packages

\usepackage{xcolor}
\usepackage{graphicx}
\usepackage{amsmath}
\usepackage{amssymb}
\usepackage{amsthm}
\usepackage{graphicx}
\usepackage{url}
\usepackage{color}
\usepackage{booktabs} % Allows the use of \toprule, \midrule and \bottomrule in tables
\usepackage{pifont}% http://ctan.org/pkg/pifont
\usepackage{epstopdf}
\usepackage[export]{adjustbox}
\usepackage{tikz}
\usetikzlibrary{shapes.misc}
\usetikzlibrary{shapes,arrows,decorations.markings,shadows,positioning}

% Your Abbreviations
\newcommand\bE{{\mathbb{E}}}
\newcommand\bR{{\mathbb{R}}}
\newcommand\bH{{\mathbf{H}}}
% End abbreviations

\newcommand\Wider[2][3em]{%
	\makebox[\linewidth][c]{%
		\begin{minipage}{\dimexpr\textwidth+#1\relax}
			\raggedright#2
		\end{minipage}%
	}%\textbf{}
}

%%%%%%%%%%%%%%%%%%%%% to edit the main text below
%NOTES ON SOME TECHNICS
%%%% Box %%%%%%%%%%%%%%%%%%%%%%%%%%%%%%%%%%%%%%%%%%%%%%%
%{\fbox{ \parbox[t]{10cm}{ SOME TEXT }}}

%%% include a picture. The file should be with extention EPS, e.g. FILENAME.EPS
%\begin{figure}[h]
%\centering
%\includegraphics[width=.7\linewidth]{FILENAME}
%\caption{{\footnotesize PUT_CAPTION }}
%\end{figure}

%\subtitle{}
%\institute[ANL/IIT]{Argonne National Laboratory\\Illinois Institute of Technology}

\title[Space Charge 2017]{Study of space charge dominated beams at the AWA rf photoinjector}
\author[N.Neveu]{{\Large Nicole Neveu}}
\institute[ANL, IIT] % (optional, but mostly needed)
{   Illinois Institute of Technology \\
	Argonne National Laboratory \\
    \url{nneveu@anl.gov} 
}
% - Use the \inst command only if there are several affiliations.
% - Keep it simple, no one is interested in your street address.
\date{ \today \\
\includegraphics[width=3cm,keepaspectratio]{../images/Argonne_cmyk_black}%
\hfill \hfill \hfill%
\includegraphics[width=4cm,keepaspectratio]{../images/IIT_Logo_blk-eps-converted-to}%
}

%\date[IIT, April 2009]{
%           Space Charge 2017 \\ Oc 18, 2009  }



\begin{document}


\begin{frame}
  \titlepage
\end{frame}
\begin{frame}
	\frametitle{Outline}
	\tableofcontents
\end{frame}

%\begin{frame}{Outline of the talk}
%  \tableofcontents
%  % You might wish to add the option [pausesections]
%\end{frame}


% Structuring a talk is a difficult task and the following structure
% may not be suitable. Here are some rules that apply for this
% solution:

% - Exactly two or three sections (other than the summary).
% - At *most* three subsections per section.
% - Talk about 30s to 2min per frame. So there should be between about
%   15 and 30 frames, all told.

% - A conference audience is likely to know very little of what you
%   are going to talk about. So *simplify*!
% - In a 20min talk, getting the main ideas across is hard
%   enough. Leave out details, even if it means being less precise than
%   you think necessary.
% - If you omit details that are vital to the proof/implementation,
%   just say so once. Everybody will be happy with that.

\section{Facility Introduction}
%%%%%%%%%%%%%%%%%%%%%%%%%%%%%%%%%%%%%%%%%%%%%%%%%%%%%%%%%%%%%%%%%%%%%%%%%%%%%%%%
\subsection{Photoinjectors}
\begin{frame}[t]
\frametitle{Argonne Wakefield Accelerator Facility}
\vspace{1em}
\begin{columns}[T] % align columns
	\begin{column}{.48\textwidth}
		%\color{red}\rule{\linewidth}{4pt}		
					Two photocathode guns and accompanying linacs:
			\begin{itemize}
				\item{\underline{\textbf{Drive Line}}: $Cs_2Te$ cathode, 6 linac cavities}
				\begin{itemize}
					\item{Charge 0.1-100nC}
					\item{Energy $\leq$ 65 MeV}
					
				\end{itemize}
				\item{\underline{\textbf{Witness Line}}: $Mg$ cathode, 1 linac cavity}
				\begin{itemize}
					\item{Charge 0.1-10nC}
					\item{Energy $\leq$ 15 MeV}
				\end{itemize}
			\end{itemize}
	\end{column}%
	\hfill%
	\begin{column}{.5\textwidth}
		\vspace{1em}
		%\color{blue}\rule{\linewidth}{4pt}
		\includegraphics[width=1.0\linewidth, right]{../images/drive_gun}
	\end{column}%
\end{columns}
\end{frame}
%%%%%%%%%%%%%%%%%%%%%%%%%%%%%%%%%%%%%%%%%%%%%%%%%%%%%%%%%%%%%%%%%%%%%%%%%%%%%%%%
\subsection{Ongoing Experiments}
\begin{frame}
	\frametitle{AWA Facility}
	%\vspace{-1em}
	Current experiments include:
	\begin{itemize}
		\item{Emittance Exchange (EEX)}
		\item{Electron Radiography Imaging (ERI)}
		\item{Cathode Studies}
	\end{itemize}
	\vspace{0.3cm}
	\centering
	\includegraphics[width=0.5\linewidth]{../images/EEX}\includegraphics[width=0.5\linewidth]{../images/cathode1}
\end{frame}
%%%%%%%%%%%%%%%%%%%%%%%%%%%%%%%%%%%%%%%%%%%%%%%%%%%%%%%%%%%%%%%%%%%%%%%%%%%%%%%%
\begin{frame}[t]
	\frametitle{AWA Facility}
	%\vspace{-1em}
	Current experiments include:
	\begin{itemize}
		\item{Two Beam Acceleration (TBA)}
		\item{Beam line design for TBA = my thesis}
		\item{Dielectric accelerating and decelerating structure tests}	
	\end{itemize}
    \vspace{1em}
    % for trim = left, lower, right, upper
	\includegraphics[width=0.5\linewidth, trim={0 0 0 1.65cm},clip]{../images/stage}\hfill%
	\includegraphics[width=0.5\linewidth]{../images/dielectrics}	
\end{frame}
%%%%%%%%%%%%%%%%%%%%%%%%%%%%%%%%%%%%%%%%%%%%%%%%%%%%%%%%%%%%%%%%%%%%%%%%%%%%%%%%
\section{Simulations}
\subsection{Code}
\begin{frame}[containsverbatim]
\frametitle{Code}
OPAL-T: \begin{verbatim}
	https://gitlab.psi.ch/OPAL/src/wikis/home
\end{verbatim}
\begin{itemize}
\item Free, open source 
\item Developed at PSI, easy to work with developers
\item Parallel (weak scaling)
\item Features include 3D space charge and wakefields
\item Can output data in beam or global reference frame
\end{itemize}
RF photoinjector benchmark:
\begin{verbatim}
https://gitlab.psi.ch/OPAL/src/wikis/RFPhotoInjector
\end{verbatim}
\end{frame}
%%%%%%%%%%%%%%%%%%%%%%%%%%%%%%%%%%%%%%%%%%%%%%%%%%%%%%%%%%%%%%%%%%%%%%%%%%%%%%%%
\subsection{Optimization}
\begin{frame}
	\frametitle{Initial Optimization Goals at 40 nC}
	\Wider[4em]{
		\begin{minipage}{0.6\textwidth}			
			\begin{itemize}
				\item Linac only, used BOBYQA algorithm
				\item{Determine optimum settings for TBA experiments}
				\item {metrics = emittance \& bunch length}
				\item{Varied 10 parameters:}
			\end{itemize}
		\end{minipage}%
		\begin{minipage}{0.4\textwidth}
			\def \gunleft {-1.0}
			\def \gunright {0.3}
			\def \loneright {1.0}
			\def \ltworight {2.0}
			\def \lthreeright {3.0}
			\def \lfourright {4.0}
			\def \lfiveright {5.0}
			\def \lsixright {6.0}
			\centering
			\begin{center}
				\begin{tikzpicture}[scale=0.55]%,use optics
				%Gun drawings
				\draw[fill=orange, very thick, rounded corners =0.1cm] (\gunleft-0.2,0.5)rectangle (\gunright,1.5) node[pos=.5, white] {\textbf{Gun}} ;
				
				%S1
				\node[] at (-1.5,2.9) {$S_1$};
				\draw[ultra thick, fill=black!60!green] (-1.4,-0.5)rectangle  (-1.0,0.5) node[pos=.5, white] {} ;
				\draw[black, ultra thick] (-1.4,-0.5) -- (-1.0,0.5);
				\draw[black, ultra thick] (-1.4,0.5) -- (-1.0,-0.5);
				\draw[ultra thick, fill=black!60!green] (-1.4,1.5)rectangle  (-1.0,2.5) node[pos=.5, white] {} ;
				\draw[black, ultra thick] (-1.4,1.5) -- (-1.0,2.5);
				\draw[black, ultra thick] (-1.4,2.5) -- (-1.0,1.5);
				%S2
				\node[] at (-0.8,2.9) {$S_2$};
				\draw[ultra thick, fill=black!60!green] (-1.0,-0.5)rectangle  (-0.6,0.5) node[pos=.5, white] {} ;
				\draw[black, ultra thick] (-1.0,-0.5) -- (-0.6,0.5);
				\draw[black, ultra thick] (-1.0,0.5) -- (-0.6,-0.5);
				\draw[ultra thick, fill=black!60!green] (-1.0,1.5)rectangle  (-0.6,2.5) node[pos=.5, white] {} ;
				\draw[black, ultra thick] (-1.0,1.5) -- (-0.6,2.5);
				\draw[black, ultra thick] (-1.0,2.5) -- (-0.6,1.5);
				
				%S3
				\node[] at (0.2,2.9) {$S_3$};
				\draw[ultra thick, fill=black!60!green] (-0.1,-0.5) rectangle  (0.3,0.5) node[pos=.5, white] {};
				\draw[black, ultra thick] (-0.1,-0.5) -- (0.3,0.5);
				\draw[black, ultra thick] (-0.1,0.5) -- (0.3,-0.5);
				\draw[ultra thick, fill=black!60!green] (-0.1,1.5) rectangle  (0.3,2.5) node[pos=.5, white] {};
				\draw[black, ultra thick] (-0.1,1.5) -- (0.3,2.5);
				\draw[black, ultra thick] (-0.1,2.5) -- (0.3,1.5);
				%Linac drawings 
				\draw[fill=blue, ultra thick, rounded corners =0.1cm] (\loneright,0)rectangle  ({\loneright+0.84},2) node[pos=.5, white] {$L_1$} ;
				\draw[fill=blue, ultra thick, rounded corners =0.1cm] (\ltworight,0)rectangle  ({\ltworight+0.84},2) node[pos=.5, white] {$L_2$};
				\draw[fill=blue, ultra thick, rounded corners =0.1cm] (\lthreeright,0)rectangle ({\lthreeright+0.84},2) node[pos=.5, white] {$L_3$};
				\draw[fill=blue, ultra thick, rounded corners =0.1cm] (\lfourright,0)rectangle ({\lfourright+0.84},2) node[pos=.5, white] {$L_4$};
				\draw[fill=blue, ultra thick, rounded corners =0.1cm] (\lfiveright,0)rectangle ({\lfiveright+0.84},2) node[pos=.5, white] {$L_5$};
				\draw[fill=blue, ultra thick, rounded corners =0.1cm] (\lsixright,0)rectangle ({\lsixright+0.84},2) node[pos=.5, white] {$L_6$};
				\end{tikzpicture}
			\end{center}
		\end{minipage}%
		\begin{center}
			\setcounter{mpfootnote}{\value{footnote}}%
			\renewcommand{\thempfootnote}{\arabic{mpfootnote}}%	
			\begin{tabular}{ l *{3}{c}}
				%\toprule
				\textbf{Variable} & \textbf{Range} & \textbf{Unit} \\
				\midrule
				Solenoid Strength & $ 150 \le S_3 \le 440$  & amps \\
				Phase of Gun & $-40 \le \phi_g \le 40$  & degrees \\
				Laser Radius  & $3 \le R \le 9$  & mm \\
				Laser FWHM  & $2 \le T \le $10  & ps \\
				Cavity Phase & $-40 \le \phi_L \le 40$\footnote[1]{$\phi_L=[\phi_{L_1},\ldots,\phi_{L_6}]$} & degrees
				%\bottomrule    
			\end{tabular}
		\end{center}
	}
\end{frame}
\begin{frame}
	\frametitle{Initial Optimization Results at 40 nC}
\begin{columns}[T] % align columns
	\begin{column}{.56\textwidth}
		%\vspace{1em}
		%\color{red}\rule{\linewidth}{4pt}		
		%\color{blue}\rule{\linewidth}{4pt}
		\includegraphics[width=1.0\linewidth, right]{../images/pareto_emittance_vs_zrms}
	    \begin{gather*}
	    f(v,w) = w \,\bar{\epsilon}_x(v,z_1) + (1-w)\, \bar{\sigma}_z(v,z_1)
	    \end{gather*}
	\end{column}%
	\hfill%
	\begin{column}{.44\textwidth}
		%\vspace{1em}
		Code verifies:
		\begin{itemize}
		\item Larger laser radius is always better
		\item Shorter laser pulse length $\rightarrow$ shorter $\sigma_z$
		\item Longer laser pulse $\rightarrow$ lower $\epsilon_{x,y}$
		\item Running off crest in linac mitigates energy spread out of gun
	    \end{itemize}	
	\end{column}%
\end{columns}
\end{frame}
%%%%%%%%%%%%%%%%%%%%%%%%%%%%%%%%%%%%%%%%%%%%%%%%%%%%%%%%%%%%%%%%%%%%%%%%%%%%%%%%
\section{Experimental Measurements}
\subsection{Overview}
\begin{frame}
	\frametitle{Overview}
Took data exactly 2 weeks ago!

Tried to dial in machine settings based on simulations:
\begin{itemize}
	\item Initial results did not match simulations - not a surprise
	\item Identified issues:
	\begin{itemize}
		\item Energy lower than expected
		\item Solenoid strength
		\item Shot to shot charge fluctuation
	\end{itemize}
	
	
	\item Adjusted settings to approach simulation values
	\item Took four types of data:
	\begin{itemize}
		\item Energy measurements
		\item \textbf{Beam size data - YAG screens}
		\item Emittance - scanning slit
		\item Bunch length - CTR (interferometer, and bolometer) 
	\end{itemize}
\end{itemize}
%\includegraphics[width=0.25\linewidth]{YAG1}\includegraphics[width=0.25\linewidth]{YAG2}%
%\includegraphics[width=0.25\linewidth]{YAG3}\includegraphics[width=0.25\linewidth]{YAG3}

\end{frame}
%%%%%%%%%%%%%%%%%%%%%%%%%%%%%%%%%%%%%%%%%%%%%%%%%%%%%%%%%%%%%%%%%%%%%%%%%%%%%%%%
\subsection{Beam Size Measurements}
\begin{frame}
	%\vspace{-0.5em}
	%\frametitle{Beam Size Measurements}
	\vspace{-0.5em}
	\centering
    \includegraphics[width=0.415\textheight]{../images/laser}%
	\includegraphics[width=0.415\textheight]{../images/YAG1}%
	\includegraphics[width=0.415\textheight]{../images/YAG2}\\
	\includegraphics[width=0.415\textheight]{../images/YAG3}
	\includegraphics[width=0.415\textheight]{../images/YAG6}%
	\includegraphics[width=0.415\textheight]{../images/YAGCTR}\\
     Hot spot on bottom left corner...origin laser?  
\end{frame}


\begin{frame}
	%\frametitle{Beam Size Measurements}
	%\vspace{-0.5em}
\begin{columns}[T] % align columns
	\begin{column}{.5\textwidth}
		%\color{red}\rule{\linewidth}{4pt}		
		2D Results:
    \includegraphics[width=0.75\linewidth]{../images/beamsizes_2Dx} \\ \vspace{0.25em}
    \includegraphics[width=0.75\linewidth]{../images/beamsizes_2Dy}
	\end{column}%
	\hfill%
	\begin{column}{.5\textwidth}
		%\vspace{1em}
		%\color{blue}\rule{\linewidth}{4pt}
		3D Results:
		\includegraphics[width=0.75\linewidth]{../images/beamsizes_3Dx} \\ \vspace{0.25em}
		\includegraphics[width=0.75\linewidth]{../images/beamsizes_3Dy}
	\end{column}%
\end{columns}
\end{frame}
%%%%%%%%%%%%%%%%%%%%%%%%%%%%%%%%%%%%%%%%%%%%%%%%%%%%%%%%%%%%%%%%%%%%%%%%%%%%%%%%
\begin{frame}
	\frametitle{Summary}
	\vspace{3em}
	End Goal: Use simulations to optimize beam parameters. 
	\begin{itemize}		
		\item AWA beam line configuration is dynamic and variable
		\item Space charge {\tiny {\tiny }}drives the limitations for TBA experiments
		\item We need a set of tools to quickly optimize extremely different parameters
		\item Agreement between simulations and measurements should guide experiments
	
	\end{itemize}
\end{frame}
%%%%%%%%%%%%%%%%%%%%%%%%%%%%%%%%%%%%%%%%%%%%%%%%%%%%%%%%%%%%%%%%%%%%%%%%%%%%%%%%
\begin{frame}
\vspace{5em}
\Huge {Thanks for your time!} \\ 
%\normalfont{And thanks to my funding sources:}
\includegraphics[width=0.5\linewidth]{../images/IIT_Logo_blk-eps-converted-to} %\\%
\includegraphics[width=0.5\linewidth]{../images/Argonne_cmyk_black}%
%\includegraphics[width=0.5\linewidth]{../images/DOE_logo_color_cmyk}
\\
\end{frame}


\end{document}
















