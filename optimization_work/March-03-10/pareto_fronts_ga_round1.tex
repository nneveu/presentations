%%%%%%%%%%%%%%%%%%%%%%%%%%%%%%%%%%%%%%%%%%%%%%%%%%%%%%%%%%%%%%%%%%%%%%
%%%%%%%%% Select one of the options, and comment the rest of them

%%%%%%%%%% Option 1:  to compile with pdflatex : parameter "t" - to align to the top
\documentclass[professionalfonts,t]{beamer}
%sans font?

%%%%%%%%%% Option 3: to create handout for print
%\documentclass[t,handout]{beamer}
%\usepackage{pgfpages}              % to put several slides on one page
%\pgfpagesuselayout{2 on 1}[a4paper, border shrink=5mm]             % 2 slides on 1 page
%\pgfpagesuselayout{4 on 1}[a4paper,landscape, border shrink=5mm]   % 4 slides on 1 page, and landscaped


%%%%%%%%%%%%%%%%%%%%%%%%%%%%%%%%%%%%%%%%%%%%%%%%%%%%%%%%%%%%%%%%%%%%
%%%%%%%%%%%%%% Select the Theme %%%%%%%%%%%%%%%%%%%%%%%%%%%%%%%%%%%
\usetheme{Dresden}     % OK
%\usetheme{Berlin}
%\usetheme{Bergen}      % NO
%\usetheme{Boadilla}    % NO
%\usetheme{Copenhagen}  % NO
%\usetheme{Hannover}    % NO
%\usetheme{Luebeck}     % NO
%\usetheme{Marburg}     % NO
%\usetheme{Pittsburgh}  % NO
%\usetheme{default}
%\usetheme{Singapore}   % OK
%\usetheme{boxes}
%\usecolortheme{structure}
%\usecolortheme{rose}
%\usecolortheme{beaver}


\definecolor{mymaroon}{cmyk}{0.0, 1.0, 1.0, 0.498}
\definecolor{myblue}{cmyk}{1.0, 1, 0, 0.5}
\definecolor{mygreen}{cmyk}{100, 0, 100, 50}
\setbeamercolor*{palette secondary}{use=structure,fg=white,bg=myblue}
\setbeamercolor*{palette tertiary}{use=structure,fg=white,bg=mymaroon}

%\usepackage{beamerthemesplit}              %
\beamertemplateballitem % fancy bullets and numbering

\setbeamertemplate{navigation symbols}{}   % suppress navigation symbols
\addtobeamertemplate{frametitle}{}{%
	\logo{../images/IIT_logo}
	\iffalse
	
	\begin{tikzpicture}[remember picture,overlay]
	\node[anchor=center, yshift=-13pt, xshift=-5pt] at (current page.north) 
	{\includegraphics[height=1.1cm]{../images/Argonne_cmyk_black-eps-converted-to}\hspace{10cm}};
	
	\node[anchor=north east, yshift=3pt, xshift=0pt] at (current page.north east) 
	{\includegraphics[height=0.7cm]{../images/IIT_Logo_blk}};
	\end{tikzpicture}
    
     \fi
}
% other possibilities to include LOGO. it puts it in RLC

%
%\pgfdeclareimage[width=1cm]{logo}{../images/IIT_Logo}
%\logo{\pgfuseimage{logo}}


% load additional packages

\usepackage{xcolor}
\usepackage{graphicx}
\usepackage{amsmath}
\usepackage{amssymb}
\usepackage{amsthm}
\usepackage{graphicx}
\usepackage{url}
\usepackage{color}
\usepackage{booktabs} % Allows the use of \toprule, \midrule and \bottomrule in tables
\usepackage{pifont}% http://ctan.org/pkg/pifont
\usepackage{epstopdf}
\usepackage[export]{adjustbox}
\usepackage{tikz}
\usetikzlibrary{shapes.misc}
\usetikzlibrary{shapes,arrows,decorations.markings,shadows,positioning}

% Your Abbreviations
\newcommand\bE{{\mathbb{E}}}
\newcommand\bR{{\mathbb{R}}}
\newcommand\bH{{\mathbf{H}}}
% End abbreviations

\newcommand\Wider[2][3em]{%
	\makebox[\linewidth][c]{%
		\begin{minipage}{\dimexpr\textwidth+#1\relax}
			\raggedright#2
		\end{minipage}%
	}%\textbf{}
}

%%%%%%%%%%%%%%%%%%%%% to edit the main text below
%NOTES ON SOME TECHNICS
%%%% Box %%%%%%%%%%%%%%%%%%%%%%%%%%%%%%%%%%%%%%%%%%%%%%%
%{\fbox{ \parbox[t]{10cm}{ SOME TEXT }}}

%%% include a picture. The file should be with extention EPS, e.g. FILENAME.EPS
%\begin{figure}[h]
%\centering
%\includegraphics[width=.7\linewidth]{FILENAME}
%\caption{{\footnotesize PUT_CAPTION }}
%\end{figure}

%\subtitle{}
%\institute[ANL/IIT]{Argonne National Laboratory\\Illinois Institute of Technology}

\title[]{Optimization and \\ Surrogate Models Work}
\author[N.Neveu]{{\Large Nicole Neveu}}
\institute[ANL, IIT] % (optional, but mostly needed)
{   Illinois Institute of Technology \\
	Argonne National Laboratory \\
    %\url{nneveu@anl.gov} 
}
% - Use the \inst command only if there are several affiliations.
% - Keep it simple, no one is interested in your street address.
\date{ \today \\
\includegraphics[width=3cm,keepaspectratio]{../logos/Argonne_cmyk_black}%
\hfill \hfill \hfill%
\includegraphics[width=4cm,keepaspectratio]{../logos/IIT_Logo_blk-eps-converted-to}%
}

%\date[IIT, April 2009]{
%           Space Charge 2017 \\ Oc 18, 2009  }



\begin{document}
%\section{Intro}

\begin{frame}
  \titlepage
\end{frame}
\begin{frame}
	\frametitle{Outline}
	\tableofcontents
\end{frame}

% Structuring a talk is a difficult task and the following structure
% may not be suitable. Here are some rules that apply for this
% solution:

% - Exactly two or three sections (other than the summary).
% - At *most* three subsections per section.
% - Talk about 30s to 2min per frame. So there should be between about
%   15 and 30 frames, all told.

% - A conference audience is likely to know very little of what you
%   are going to talk about. So *simplify*!
% - In a 20min talk, getting the main ideas across is hard
%   enough. Leave out details, even if it means being less precise than
%   you think necessary.
% - If you omit details that are vital to the proof/implementation,
%   just say so once. Everybody will be happy with that.

\section{TBA Optimization Work}
%%%%%%%%%%%%%%%%%%%%%%%%%%%%%%%%%%%%%%%%%%%%%%%%%%%%%%%%%%%%%%%%%%%%%%%%%%%%%%%%
\begin{frame}
	\frametitle{TBA Optimization}
	Goal 1: Finish this arxiv paper in progress: \url{https://arxiv.org/pdf/1302.2889.pdf}
		\begin{itemize}
			\item Use TBA beam line as model (up to the septum)
			\item Results on next two slides, Ex-1-4 stand for various GA parameters
			\item I will add AWA section to paper
			\item Hopefully submit in a few weeks, given few rounds of editing
		\end{itemize} 
	\vspace{1em}
	Goal 2: Finish whole TBA optimization
		\begin{itemize}
			\item Framework in place and working to do this
			\item Needs a few more big runs on Bebop 
			\item Will be done by/before IPAC
		\end{itemize}

\end{frame}

\begin{frame}[t]
\frametitle{Pareto front at entrance of kicker, z = 16.5 m}
		\begin{minipage}{0.48\textwidth}
			Ex-1
			\includegraphics[width=1.0\linewidth]{/home/nicole/Documents/opal_stuff/awa-optim/ex-1_gen96}
			Ex-2	\includegraphics[width=1.0\linewidth]{/home/nicole/Documents/opal_stuff/awa-optim/ex-2_gen81}
		\end{minipage}
		\begin{minipage}{0.48\textwidth}
			Ex-3
			\includegraphics[width=1.0\linewidth]{/home/nicole/Documents/opal_stuff/awa-optim/ex-3_gen53}
			Ex-4
			\includegraphics[width=1.0\linewidth]{/home/nicole/Documents/opal_stuff/awa-optim/ex-4_gen95}
		\end{minipage}

\end{frame}

\begin{frame}[t]
	\frametitle{Pareto front at entrance of septum, z = 18.5 m}
	\begin{minipage}{0.48\textwidth}
		Ex-1
		\includegraphics[width=1.0\linewidth]{/home/nicole/Documents/opal_stuff/awa-optim/ex-1_gen96_z2}
		Ex-2
		\includegraphics[width=1.0\linewidth]{/home/nicole/Documents/opal_stuff/awa-optim/ex-2_gen81_z2}
	\end{minipage}
	\begin{minipage}{0.48\textwidth}
		Ex-3
		\includegraphics[width=1.0\linewidth]{/home/nicole/Documents/opal_stuff/awa-optim/ex-3_gen53_z2}
		Ex-4
		\includegraphics[width=1.0\linewidth]{/home/nicole/Documents/opal_stuff/awa-optim/ex-4_gen95_z2}
	\end{minipage}	
\end{frame}
%%%%%%%%%%%%%%%%%%%%%%%%%%%%%%%%%%%%%%%%%%%%%%%%%%%%%%%%%%%%%%%%%%%%%%%%%%%%%%%%

%%%%%%%%%%%%%%%%%%%%%%%%%%%%%%%%%%%%%%%%%%%%%%%%%%%%%%%%%%%%%%%%%%%%%%%%%%%%%%%%
\section{Surrogate Models}
\begin{frame}
	\frametitle{Surrogate Models and Machine Learning}
	Andreas and Auralee invited me to collaboration on a Surrogate Model and Machine learning (ML) paper that is an extension of the work we did for the ML workshop at SLAC.
	\vspace{1em}	
	
	Goal: Generate 3 sets of data to be used for the paper:
	\begin{enumerate}
		\item Optimized data using genetic algorithm in OPAL 
			\begin{itemize}
				\item about 33k simulations completed
			\end{itemize}

		\item Random sample 
		\begin{itemize}
			\item 1200 simulations completed
		\end{itemize}
		\item Polynomial-chaos set 
		\begin{itemize}
			\item 3rd order, about 16.5k simulations completed
		\end{itemize}
	\end{enumerate}
	\vspace{1em}
	We may need to re-run 1 or 3 depending on post processing results.
\end{frame}

\begin{frame}
	\frametitle{Surrogate Models continued...}
	Software used:
	\begin{itemize}
		\item OPAL as forward simulation
		\item runOPAL python script (from PSI)
		\item libensemble, library that Jeff Larson from MCS is developing
		\item Generating functions to go with libensemble (Nicole)
		\item UQ python code  (Andreas)
		\item Neural Net python code (Auralee)
	\end{itemize}
		
\end{frame}

\begin{frame}[t]
	\frametitle{Surrogate Models: GA optimization}
	\centering
				\begin{itemize}
				\item Model includes gun and first linac cavity
				\item 7 optimization variables: 2 solenoids, laser radius, 2 gradients, 2 phases.  
			\end{itemize}
		\includegraphics[width=0.8\linewidth]{/home/nicole/Documents/opal_stuff/surrogate-ga/zoomed_in_100gen}
\end{frame}


\section{Towards INCITE}
\begin{frame}
	\frametitle{Scaling and 1:1 Work }
	Theta is an intermediate cluster between Bebop and Mira.
	Goal here is to get OPAL ready for a competitive INCITE proposal. 
	Project would be 1:1 simulations for plasma wakefield experiment.
	Work done includes (mostly done by Andreas):
	
	\begin{itemize}
		\item Worked with Kevin at ALCF to install OPAL on Theta
		\item Did scaling study on FFT, passed data to ALCF
		\item ALCF folks (Kevin) are working on understanding the scalability of the FFT.
		\item Identified issues with the particle update (already know)
		\item 4E10 simulations are possible with the current version of OPAL.
	\end{itemize}
\end{frame}

\begin{frame}[t]
	\frametitle{Preliminary Scaling Data}
	\centering
		\includegraphics[width=.45\linewidth]{scaling-theta-1}
		\includegraphics[width=.45\linewidth]{scaling-theta-2}\includegraphics[width=.45\linewidth]{scaling-theta-3}
\end{frame}

\begin{frame}
	\frametitle{To Do:}
	\begin{enumerate}
		\item Finish GA paper
		\item Finish TBA optimization
		\item Post process surrogate models data
		\item Make 4th order PC data set, if needed
		\item Work on getting surrogate models draft done by end of month.
		\item Continue scaling studies for INCITE
	\end{enumerate}
\end{frame}
\end{document}
















