%%%%%%%%%%%%%%%%%%%%%%%%%%%%%%%%%%%%%%%%%%%%%%%%%%%%%%%%%%%%%%%%%%%%%%
%%%%%%%%% Select one of the options, and comment the rest of them

%%%%%%%%%% Option 1:  to compile with pdflatex : parameter "t" - to align to the top
\documentclass[professionalfonts,t]{beamer}
%sans font?

%%%%%%%%%% Option 3: to create handout for print
%\documentclass[t,handout]{beamer}
%\usepackage{pgfpages}              % to put several slides on one page
%\pgfpagesuselayout{2 on 1}[a4paper, border shrink=5mm]             % 2 slides on 1 page
%\pgfpagesuselayout{4 on 1}[a4paper,landscape, border shrink=5mm]   % 4 slides on 1 page, and landscaped


%%%%%%%%%%%%%%%%%%%%%%%%%%%%%%%%%%%%%%%%%%%%%%%%%%%%%%%%%%%%%%%%%%%%
%%%%%%%%%%%%%% Select the Theme %%%%%%%%%%%%%%%%%%%%%%%%%%%%%%%%%%%
\usetheme{Dresden}     % OK
%\usetheme{Berlin}
%\usetheme{Bergen}      % NO
%\usetheme{Boadilla}    % NO
%\usetheme{Copenhagen}  % NO
%\usetheme{Hannover}    % NO
%\usetheme{Luebeck}     % NO
%\usetheme{Marburg}     % NO
%\usetheme{Pittsburgh}  % NO
%\usetheme{default}
%\usetheme{Singapore}   % OK
%\usetheme{boxes}
%\usecolortheme{structure}
%\usecolortheme{rose}
%\usecolortheme{beaver}


\definecolor{mymaroon}{cmyk}{0.0, 1.0, 1.0, 0.498}
\definecolor{myblue}{cmyk}{1.0, 1, 0, 0.5}
\definecolor{mygreen}{cmyk}{100, 0, 100, 50}
\setbeamercolor*{palette secondary}{use=structure,fg=white,bg=myblue}
\setbeamercolor*{palette tertiary}{use=structure,fg=white,bg=mymaroon}

%\usepackage{beamerthemesplit}              %
\beamertemplateballitem % fancy bullets and numbering

\setbeamertemplate{navigation symbols}{}   % suppress navigation symbols
\addtobeamertemplate{frametitle}{}{%
	\logo{../../images/IIT_logo}
	\iffalse
	
	\begin{tikzpicture}[remember picture,overlay]
	\node[anchor=center, yshift=-13pt, xshift=-5pt] at (current page.north) 
	{\includegraphics[height=1.1cm]{../images/Argonne_cmyk_black-eps-converted-to}\hspace{10cm}};
	
	\node[anchor=north east, yshift=3pt, xshift=0pt] at (current page.north east) 
	{\includegraphics[height=0.7cm]{../images/IIT_Logo_blk}};
	\end{tikzpicture}
    
     \fi
}
% other possibilities to include LOGO. it puts it in RLC

%
%\pgfdeclareimage[width=1cm]{logo}{../images/IIT_Logo}
%\logo{\pgfuseimage{logo}}


% load additional packages

\usepackage{xcolor}
\usepackage{graphicx}
\usepackage{amsmath}
\usepackage{amssymb}
\usepackage{amsthm}
\usepackage{graphicx}
\usepackage{url}
\usepackage{color}
\usepackage{booktabs} % Allows the use of \toprule, \midrule and \bottomrule in tables
\usepackage{pifont}% http://ctan.org/pkg/pifont
\usepackage{epstopdf}
\usepackage[export]{adjustbox}
\usepackage{tikz}
\usetikzlibrary{shapes.misc}
\usetikzlibrary{shapes,arrows,decorations.markings,shadows,positioning}

% Your Abbreviations
\newcommand\bE{{\mathbb{E}}}
\newcommand\bR{{\mathbb{R}}}
\newcommand\bH{{\mathbf{H}}}
% End abbreviations

\newcommand\Wider[2][3em]{%
	\makebox[\linewidth][c]{%
		\begin{minipage}{\dimexpr\textwidth+#1\relax}
			\raggedright#2
		\end{minipage}%
	}%\textbf{}
}

%%%%%%%%%%%%%%%%%%%%% to edit the main text below
%NOTES ON SOME TECHNICS
%%%% Box %%%%%%%%%%%%%%%%%%%%%%%%%%%%%%%%%%%%%%%%%%%%%%%
%{\fbox{ \parbox[t]{10cm}{ SOME TEXT }}}

%%% include a picture. The file should be with extention EPS, e.g. FILENAME.EPS
%\begin{figure}[h]
%\centering
%\includegraphics[width=.7\linewidth]{FILENAME}
%\caption{{\footnotesize PUT_CAPTION }}
%\end{figure}

%\subtitle{}
%\institute[ANL/IIT]{Argonne National Laboratory\\Illinois Institute of Technology}

\title[May 2018]{Preparations for TBA}
\author[N.Neveu]{{\Large Nicole Neveu}}
\institute[ANL, IIT] % (optional, but mostly needed)
{   Illinois Institute of Technology \\
	Argonne National Laboratory \\
    \url{nneveu@anl.gov} 
}
% - Use the \inst command only if there are several affiliations.
% - Keep it simple, no one is interested in your street address.
\date{ \today \\
\includegraphics[width=3cm,keepaspectratio]{../logos/Argonne_cmyk_black}%
\hfill \hfill \hfill%
\includegraphics[width=4cm,keepaspectratio]{../logos/IIT_Logo_blk-eps-converted-to}%
}

%\date[IIT, April 2009]{
%           Space Charge 2017 \\ Oc 18, 2009  }



\begin{document}


\begin{frame}
  \titlepage
\end{frame}
\begin{frame}
	\frametitle{Outline}
	\tableofcontents
\end{frame}

\begin{frame}
\frametitle{Beam Studies}
	To accomplish TBA:
	\vspace{0.5em}

		\begin{itemize}
			\item Kicker test
			\item Comparison of beam measurements to optimized simulations
			\item Phase test
			\item Quad settings comparison
			\item Septum test
			\item Power Measurements
		\end{itemize}
	
\end{frame}

%%%%%%%%%%%%%%%%%%%%%%%%%%%%%%%%%%%%%%%%%%%%%%%%%%%%%%%%%%%%%%%%%%%%%%%%%%%%%%%%
\section{Kicker Test}
\begin{frame}
\frametitle{Kicker Test}
Verify what kicker voltage gives correct deflection. 
This check will depend heavily on beam size entering the kicker.
If beam is scraping, measurement will not be clear. Then, 
preliminary upstream work would be required.
\vspace{0.2em}



\end{frame}


\begin{frame}
	\begin{itemize}
	\item Kicker gap = 40 mm
	\item $\sigma_x \approxeq 1.4$ mm
	\begin{itemize}
		\item With no phase control
		\item Full size 8.4 mm
	\end{itemize}
	\item Deflection exiting the kicker = XXX mm
\end{itemize}
\begin{minipage}{0.45\textwidth}
		\centering
\includegraphics[width=\textwidth]{/home/nicole/Documents/thesis_code/tba_traj/XoffsetVsEnergy}
	
\end{minipage}
	\begin{minipage}{0.45\textwidth}
		\centering
		\includegraphics[width=\textwidth]{/home/nicole/Documents/thesis_code/tba_traj/AngleVsEnergy}
	\end{minipage}
\centering
*Black line indicates $2^{\circ}$ kick
\end{frame}

\begin{frame}
\frametitle{Kicker Test}
Goal: Enter and exit kicker without scraping and achieve good beam quality.
\vspace{1em}

Measurement Steps:
\begin{itemize}
	\item Verify energy and beam size upstream of kicker. 
	\begin{itemize}
		\item i.e: quad settings give correct beam size.
		\item Leave YAG before kicker in until beam size is correct.
	\end{itemize}
	
	\item Step voltage up on kicker and measure
	deflection (angle) for several settings.
	\item Take measurements on scanning slit after kicker.  
	\item Beam size on all possible YAGs. 
	\item Measure energy before and after kicker. 
\end{itemize}
\end{frame}
%%%%%%%%%%%%%%%%%%%%%%%%%%%%%%%%%%%%%%%%%%%%%%%%%%%%%%%%%%%%%%%%%%%%%%%%%%%%%%%%

\section{Phase Control}
\begin{frame}
\frametitle{Phase Scans}
Test Liu's phase scan program.
Two linacs at a time if possible.
Take corresponding energy measurements to verify phase.
\vspace{1em}

Goal: Minimize emittance and maximize energy into tba section. 

\vspace{1em}
Measurements:
\begin{itemize}
	\item Phase (via Lui's program)
	\item Energy - compare to simulation for various phases
	\item Record beam size on all possible YAGs
	\item Do above bullets for low and high charge if possible
\end{itemize}
\end{frame}

%%%%%%%%%%%%%%%%%%%%%%%%%%%%%%%%%%%%%%%%%%%%%%%%%%%%%%%%%%%%%%%%%%%%%%%%%%%%%%%%

\section{Quad Comparison}
\begin{frame}
\frametitle{Quad Comparison}
	Compare focusing strengths of quads to simulations.
	
	\vspace{1em}
	Goal: Achieve desired beam size at entrance of kicker.

\vspace{1em}	
Measurements
	\begin{itemize}
		\item Verify energy and beam size upstream of Quad. 
		\item Based on optimization settings: Quad settings vs. beam sizes
		\item Current readings on power supplies
		\item Beam size on all possible YAGs
		\item Low and high charge if possible.
	\end{itemize}

\end{frame}

\begin{frame}
\frametitle{Quad Comparison}
	** Add a slide of quad settings vs. beam size.
	Add a table*****
\end{frame}


%%%%%%%%%%%%%%%%%%%%%%%%%%%%%%%%%%%%%%%%%%%%%%%%%%%%%%%%%%%%%%%%%%%%%%%%%%%%%%%%
\section{Power}
\begin{frame}
	\frametitle{Power Measurements}
	Gun - Solenoid scan indicates energy is not matching. \\
	Linac - No documentation of power/energy gain in each linac tank. 
	
	\vspace{0.5em}
	Goal: Redo measurement of power in gun and linacs (only the 2 with calibration numbers). 
	
	\vspace{1em}
	\begin{itemize}
		\item Power measurements from rf pick ups using power meter.
		\item Record gun traces from bi-directional coupler (Eric's thesis)
		\item Energy measurements gun alone, add linacs. 
	\end{itemize} 


\end{frame}


%%%%%%%%%%%%%%%%%%%%%%%%%%%%%%%%%%%%%%%%%%%%%%%%%%%%%%%%%%%%%%%%%%%%%%%%%%%%%%%%
\begin{frame}
	\frametitle{Summary}
	End goals:
	\begin{itemize}
		\item Verify kicker is working properly
		\item Compare optimization simulation parameters
		\item Explore quadruple settings
	\end{itemize}
	
	
	
\end{frame}
%%%%%%%%%%%%%%%%%%%%%%%%%%%%%%%%%%%%%%%%%%%%%%%%%%%%%%%%%%%%%%%%%%%%%%%%%%%%%%%%



\section{Backup}
\begin{frame}
\frametitle{Kicker Run Plan}
	\begin{itemize}
		\item 
	\end{itemize}

\end{frame}

\begin{frame}
	\frametitle{Power Measurements Run Plan}
	Have procedural backup slide
\end{frame}

\begin{frame}
	Equipment and people needed for each experiment.
\end{frame}
\end{document}
















