%%%%%%%%%%%%%%%%%%%%%%%%%%%%%%%%%%%%%%%%%%%%%%%%%%%%%%%%%%%%%%%%%%%%%%
%%%%%%%%% Select one of the options, and comment the rest of them

%%%%%%%%%% Option 1:  to compile with pdflatex : parameter "t" - to align to the top
\documentclass[professionalfonts,t]{beamer}
%sans font?

%%%%%%%%%% Option 3: to create handout for print
%\documentclass[t,handout]{beamer}
%\usepackage{pgfpages}              % to put several slides on one page
%\pgfpagesuselayout{2 on 1}[a4paper, border shrink=5mm]             % 2 slides on 1 page
%\pgfpagesuselayout{4 on 1}[a4paper,landscape, border shrink=5mm]   % 4 slides on 1 page, and landscaped


%%%%%%%%%%%%%%%%%%%%%%%%%%%%%%%%%%%%%%%%%%%%%%%%%%%%%%%%%%%%%%%%%%%%
%%%%%%%%%%%%%% Select the Theme %%%%%%%%%%%%%%%%%%%%%%%%%%%%%%%%%%%
\usetheme{Dresden}     % OK
%\usetheme{Berlin}
%\usetheme{Bergen}      % NO
%\usetheme{Boadilla}    % NO
%\usetheme{Copenhagen}  % NO
%\usetheme{Hannover}    % NO
%\usetheme{Luebeck}     % NO
%\usetheme{Marburg}     % NO
%\usetheme{Pittsburgh}  % NO
%\usetheme{default}
%\usetheme{Singapore}   % OK
%\usetheme{boxes}
%\usecolortheme{structure}
%\usecolortheme{rose}
%\usecolortheme{beaver}


\definecolor{mymaroon}{cmyk}{0.0, 1.0, 1.0, 0.498}
\definecolor{myblue}{cmyk}{1.0, 1, 0, 0.5}
\definecolor{mygreen}{cmyk}{100, 0, 100, 50}
\setbeamercolor*{palette secondary}{use=structure,fg=white,bg=myblue}
\setbeamercolor*{palette tertiary}{use=structure,fg=white,bg=mymaroon}

%\usepackage{beamerthemesplit}              %
\beamertemplateballitem % fancy bullets and numbering

\setbeamertemplate{navigation symbols}{}   % suppress navigation symbols
\addtobeamertemplate{frametitle}{}{%
	\logo{../../images/IIT_logo}
	\iffalse
	
	\begin{tikzpicture}[remember picture,overlay]
	\node[anchor=center, yshift=-13pt, xshift=-5pt] at (current page.north) 
	{\includegraphics[height=1.1cm]{../images/Argonne_cmyk_black-eps-converted-to}\hspace{10cm}};
	
	\node[anchor=north east, yshift=3pt, xshift=0pt] at (current page.north east) 
	{\includegraphics[height=0.7cm]{../images/IIT_Logo_blk}};
	\end{tikzpicture}
    
     \fi
}
% other possibilities to include LOGO. it puts it in RLC

%
%\pgfdeclareimage[width=1cm]{logo}{../images/IIT_Logo}
%\logo{\pgfuseimage{logo}}


% load additional packages

\usepackage{xcolor}
\usepackage{graphicx}
\usepackage{amsmath}
\usepackage{amssymb}
\usepackage{amsthm}
\usepackage{graphicx}
\usepackage{url}
\usepackage{color}
\usepackage{booktabs} % Allows the use of \toprule, \midrule and \bottomrule in tables
\usepackage{pifont}% http://ctan.org/pkg/pifont
\usepackage{epstopdf}
\usepackage[export]{adjustbox}
\usepackage{tikz}
\usetikzlibrary{shapes.misc}
\usetikzlibrary{shapes,arrows,decorations.markings,shadows,positioning}

% Your Abbreviations
\newcommand\bE{{\mathbb{E}}}
\newcommand\bR{{\mathbb{R}}}
\newcommand\bH{{\mathbf{H}}}
% End abbreviations

\newcommand\Wider[2][3em]{%
	\makebox[\linewidth][c]{%
		\begin{minipage}{\dimexpr\textwidth+#1\relax}
			\raggedright#2
		\end{minipage}%
	}%\textbf{}
}

%%%%%%%%%%%%%%%%%%%%% to edit the main text below
%NOTES ON SOME TECHNICS
%%%% Box %%%%%%%%%%%%%%%%%%%%%%%%%%%%%%%%%%%%%%%%%%%%%%%
%{\fbox{ \parbox[t]{10cm}{ SOME TEXT }}}

%%% include a picture. The file should be with extention EPS, e.g. FILENAME.EPS
%\begin{figure}[h]
%\centering
%\includegraphics[width=.7\linewidth]{FILENAME}
%\caption{{\footnotesize PUT_CAPTION }}
%\end{figure}

%\subtitle{}
%\institute[ANL/IIT]{Argonne National Laboratory\\Illinois Institute of Technology}

\title[May 2018]{Beam Studies}
\author[N.Neveu]{{\Large Nicole Neveu}}
\institute[ANL, IIT] % (optional, but mostly needed)
{   Illinois Institute of Technology \\
	Argonne National Laboratory \\
    \url{nneveu@anl.gov} 
}
% - Use the \inst command only if there are several affiliations.
% - Keep it simple, no one is interested in your street address.
\date{ \today \\
\includegraphics[width=3cm,keepaspectratio]{../logos/Argonne_cmyk_black}%
\hfill \hfill \hfill%
\includegraphics[width=4cm,keepaspectratio]{../logos/IIT_Logo_blk-eps-converted-to}%
}

%\date[IIT, April 2009]{
%           Space Charge 2017 \\ Oc 18, 2009  }



\begin{document}


\begin{frame}
  \titlepage
\end{frame}
\begin{frame}
	\frametitle{Outline}
	\tableofcontents
\end{frame}

\begin{frame}
	Steps to accomplish TBA:
	\vspace{1em}

		\begin{itemize}
			\item Kicker test
			\item Prepare of optimization, compare to simulations:
			\item Phase test
			\item Quad comparison
			\item Optimize:
			\item Quad parameters
			\item Septum test
			\item Power Measurements
		\end{itemize}
	
\end{frame}

%%%%%%%%%%%%%%%%%%%%%%%%%%%%%%%%%%%%%%%%%%%%%%%%%%%%%%%%%%%%%%%%%%%%%%%%%%%%%%%%
\section{Kicker Test}
\begin{frame}
\frametitle{Kicker Test}
Verify what kicker voltage gives correct deflection. 
This check will depend heavily having a manageable beam size entering the kicker.
If beam is scraping, measurement will not be clear.

\vspace{0.5em}
Measurements:
\begin{itemize}
	\item Verify energy and beam size upstream of kicker. If not scraping go on, else go back to adjust upstream settings. 
	\item Verify quad settings
	\item Step voltage up on kicker and measure
	deflection (angle) for several settings.
	Take measurements on scanning slit.  
	\item Beam size on all possible YAGs. 
	\item Measure energy after kicker. 
\end{itemize}

Add estimate of kicker deflection****.
Add nominal beam size, nominal beam enregy, kicker aperture, kicker voltage.
Add table of kicker voltage vs. deflection.
\end{frame}

%%%%%%%%%%%%%%%%%%%%%%%%%%%%%%%%%%%%%%%%%%%%%%%%%%%%%%%%%%%%%%%%%%%%%%%%%%%%%%%%

\section{Phase Control}
\begin{frame}
\frametitle{Phase Scans}
Test Liu's phase scan program.
Two linacs at a time if possible.
Take corresponding energy measurements to verify phase.
Important for minimization of emittance into tba section. 

\vspace{1em}
Measurements:
\begin{itemize}
	\item Phase 
	\item Energy 
	\item Beam size on all possible YAGs
	\item Low and high charge if possible
\end{itemize}

\vspace{1em}
Goal: Minimize emittance and maximize energy
\end{frame}

%%%%%%%%%%%%%%%%%%%%%%%%%%%%%%%%%%%%%%%%%%%%%%%%%%%%%%%%%%%%%%%%%%%%%%%%%%%%%%%%

\section{Quad Comparison}
\begin{frame}
\frametitle{Quad Comparison}
	Compare focusing strengths of quads to simulations.
	Important for initial beam size into kicker.

\vspace{1em}	
Measurements
	\begin{itemize}
		\item Verify energy and beam size upstream of Quad. 
		\item Based on optimization settings: Quad settings vs. beam sizes
		\item Current readings on power supplies
		\item Beam size on all possible YAGs
		\item Low and high charge if possible.
	\end{itemize}

** Add a slide of quad settings vs. beam size.
Add a table*****
\end{frame}


%%%%%%%%%%%%%%%%%%%%%%%%%%%%%%%%%%%%%%%%%%%%%%%%%%%%%%%%%%%%%%%%%%%%%%%%%%%%%%%%
\section{Power}
\begin{frame}
	\frametitle{Power Measurements}
	Gun - Solenoid scan indicates energy is not matching. \\
	Linac - No documentation of gradient/power in linac tanks. 
	\vspace{1em}
	
	Goal: Redo measurement of power in gun and linacs (only the 2 with calibration numbers). 
	Use pick up probe and cross check with bi-directional coupler as in Eric's thesis.
	
	\vspace{1em}
	\begin{itemize}
		\item Power measurements from rf pick ups using power meter.
		\item Record traces from bi-directional coupler?
		\item Energy measurements
	\end{itemize} 

Have procedural backup slide
\end{frame}


%%%%%%%%%%%%%%%%%%%%%%%%%%%%%%%%%%%%%%%%%%%%%%%%%%%%%%%%%%%%%%%%%%%%%%%%%%%%%%%%
\begin{frame}
	\frametitle{Summary}
	End goals:
	\begin{itemize}
		\item Verify kicker is working properly
		\item Compare optimization simulation parameters
		\item Explore quadruple settings
	\end{itemize}
	
	
	
\end{frame}
%%%%%%%%%%%%%%%%%%%%%%%%%%%%%%%%%%%%%%%%%%%%%%%%%%%%%%%%%%%%%%%%%%%%%%%%%%%%%%%%



\section{Backup}
\begin{frame}
	Equipment and people needed for each experiment.
\end{frame}
\end{document}
















