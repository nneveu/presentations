%%%%%%%%%%%%%%%%%%%%%%%%%%%%%%%%%%%%%%%%%%%%%%%%%%%%%%%%%%%%%%%%%%%%%%
%%%%%%%%% Select one of the options, and comment the rest of them

%%%%%%%%%% Option 1:  to compile with pdflatex : parameter "t" - to align to the top
\documentclass[professionalfonts,t]{beamer}
%sans font?

%%%%%%%%%% Option 3: to create handout for print
%\documentclass[t,handout]{beamer}
%\usepackage{pgfpages}              % to put several slides on one page
%\pgfpagesuselayout{2 on 1}[a4paper, border shrink=5mm]             % 2 slides on 1 page
%\pgfpagesuselayout{4 on 1}[a4paper,landscape, border shrink=5mm]   % 4 slides on 1 page, and landscaped


%%%%%%%%%%%%%%%%%%%%%%%%%%%%%%%%%%%%%%%%%%%%%%%%%%%%%%%%%%%%%%%%%%%%
%%%%%%%%%%%%%% Select the Theme %%%%%%%%%%%%%%%%%%%%%%%%%%%%%%%%%%%
\usetheme{Dresden}     % OK
%\usetheme{Berlin}
%\usetheme{Bergen}      % NO
%\usetheme{Boadilla}    % NO
%\usetheme{Copenhagen}  % NO
%\usetheme{Hannover}    % NO
%\usetheme{Luebeck}     % NO
%\usetheme{Marburg}     % NO
%\usetheme{Pittsburgh}  % NO
%\usetheme{default}
%\usetheme{Singapore}   % OK
%\usetheme{boxes}
%\usecolortheme{structure}
%\usecolortheme{rose}
%\usecolortheme{beaver}


\definecolor{mymaroon}{cmyk}{0.0, 1.0, 1.0, 0.498}
\definecolor{myblue}{cmyk}{1.0, 1, 0, 0.5}
\definecolor{mygreen}{cmyk}{100, 0, 100, 50}
\setbeamercolor*{palette secondary}{use=structure,fg=white,bg=myblue}
\setbeamercolor*{palette tertiary}{use=structure,fg=white,bg=mymaroon}

%\usepackage{beamerthemesplit}              %
\beamertemplateballitem % fancy bullets and numbering

\setbeamertemplate{navigation symbols}{}   % suppress navigation symbols
\addtobeamertemplate{frametitle}{}{%
	\logo{../images/IIT_logo}
	\iffalse
	
	\begin{tikzpicture}[remember picture,overlay]
	\node[anchor=center, yshift=-13pt, xshift=-5pt] at (current page.north) 
	{\includegraphics[height=1.1cm]{../images/Argonne_cmyk_black-eps-converted-to}\hspace{10cm}};
	
	\node[anchor=north east, yshift=3pt, xshift=0pt] at (current page.north east) 
	{\includegraphics[height=0.7cm]{../images/IIT_Logo_blk}};
	\end{tikzpicture}
    
     \fi
}
% other possibilities to include LOGO. it puts it in RLC

%
%\pgfdeclareimage[width=1cm]{logo}{../images/IIT_Logo}
%\logo{\pgfuseimage{logo}}


% load additional packages

\usepackage{xcolor}
\usepackage{graphicx}
\usepackage{amsmath}
\usepackage{amssymb}
\usepackage{amsthm}
\usepackage{graphicx}
\usepackage{url}
\usepackage{color}
\usepackage{booktabs} % Allows the use of \toprule, \midrule and \bottomrule in tables
\usepackage{pifont}% http://ctan.org/pkg/pifont
\usepackage{epstopdf}
\usepackage[export]{adjustbox}
\usepackage{tikz}
\usetikzlibrary{shapes.misc}
\usetikzlibrary{shapes,arrows,decorations.markings,shadows,positioning}

% Your Abbreviations
\newcommand\bE{{\mathbb{E}}}
\newcommand\bR{{\mathbb{R}}}
\newcommand\bH{{\mathbf{H}}}
% End abbreviations

\newcommand\Wider[2][3em]{%
	\makebox[\linewidth][c]{%
		\begin{minipage}{\dimexpr\textwidth+#1\relax}
			\raggedright#2
		\end{minipage}%
	}%\textbf{}
}

%%%%%%%%%%%%%%%%%%%%% to edit the main text below
%NOTES ON SOME TECHNICS
%%%% Box %%%%%%%%%%%%%%%%%%%%%%%%%%%%%%%%%%%%%%%%%%%%%%%
%{\fbox{ \parbox[t]{10cm}{ SOME TEXT }}}

%%% include a picture. The file should be with extention EPS, e.g. FILENAME.EPS
%\begin{figure}[h]
%\centering
%\includegraphics[width=.7\linewidth]{FILENAME}
%\caption{{\footnotesize PUT_CAPTION }}
%\end{figure}

%\subtitle{}
%\institute[ANL/IIT]{Argonne National Laboratory\\Illinois Institute of Technology}

\title[USPAS Fundamentals, June 4-15, 2018]{Accelerating Structures and Linear Machines}
\author[N.Neveu]{{\Large Nicole Neveu}}
\institute[ANL, IIT] % (optional, but mostly needed)
{   Illinois Institute of Technology \\
	Argonne National Laboratory \\
    \url{nneveu@anl.gov} 
}
% - Use the \inst command only if there are several affiliations.
% - Keep it simple, no one is interested in your street address.
\date{ June 8, 2018 \\ \vspace{0.5em}
\includegraphics[width=3.5cm,keepaspectratio]{../../logos/Argonne_cmyk_black}%
\hfill \hfill \hfill%
\includegraphics[width=4cm,keepaspectratio]{../../logos/IIT_Logo_blk-eps-converted-to}%
}

%\date[IIT, April 2009]{
%           Space Charge 2017 \\ Oc 18, 2009  }



\AtBeginSection[]
{
\begin{frame}[noframenumbering]
	\frametitle{Outline}
\begin{minipage}{0.4\textwidth}
	\tableofcontents[currentsection]
\end{minipage}\hfill
\begin{minipage}{0.5\textwidth}
	\centering
	\includegraphics[width=0.85\textwidth]{linda_linac}
	\vspace{-1em}
	Source: L. Spentzouris
\end{minipage}
\end{frame}
}

\begin{document}


\begin{frame}
  \titlepage
\end{frame}
\begin{frame}
	\frametitle{Outline}
	\begin{minipage}{0.4\textwidth}
		\tableofcontents
	\end{minipage}\hfill
\begin{minipage}{0.5\textwidth}
	\centering
	\includegraphics[width=0.85\textwidth]{linda_linac}
	Source: L. Spentzouris
\end{minipage}
\end{frame}

%\begin{frame}{Outline of the talk}
%  \tableofcontents
%  % You might wish to add the option [pausesections]
%\end{frame}


% Structuring a talk is a difficult task and the following structure
% may not be suitable. Here are some rules that apply for this
% solution:

% - Exactly two or three sections (other than the summary).
% - At *most* three subsections per section.
% - Talk about 30s to 2min per frame. So there should be between about
%   15 and 30 frames, all told.

% - A conference audience is likely to know very little of what you
%   are going to talk about. So *simplify*!
% - In a 20min talk, getting the main ideas across is hard
%   enough. Leave out details, even if it means being less precise than
%   you think necessary.
% - If you omit details that are vital to the proof/implementation,
%   just say so once. Everybody will be happy with that.


\section{Linacs}
%\subsection{Introduction}

\begin{frame}
	\frametitle{Linear Accelerators (Linacs)}
	
	\vspace{1em}
	Goals for this talk:
	
	\begin{minipage}{0.43\textwidth}
		\begin{itemize}
			\item What is a linac?
			\item Why do we need them?
			\item How do they work?
		\end{itemize}
	\end{minipage}
\begin{minipage}{0.55\textwidth}
	\centering
	\includegraphics[width=0.95\textwidth]{awa_linac}
	
	Source: AWA-ANL and 
\end{minipage}

\vspace{1em}
Note: this is not the end all be all talk on linacs!
It's a general overview of some common machine types and techniques.

\end{frame}

\begin{frame}
\frametitle{Some Uses for Linacs}
\vspace{-0.5em}
\begin{minipage}{0.45\textwidth}
	\begin{itemize}
		\item Colliders (SLAC...(?))
		\item Sources for circular machines
		\item Light sources produce x-rays for experiments:
		\begin{itemize}
			\item Chemistry, Biology, Material Science, Engineering
		\end{itemize}
		\item Medical accelerators 
		\begin{itemize}
			\item Cancer therapy 
			\item Isotope production
		\end{itemize}
	\end{itemize}
\end{minipage}\hfill
\begin{minipage}{0.5\textwidth}
	\centering
	\vspace{-1.25em}
	
	\small Proton Improvement Plan (PIP-II)
	\includegraphics[width=0.9\textwidth]{pip2}
	
	\vspace{-0.25em}
	Picture Source: E. Prebys
\end{minipage}
\end{frame}


\subsection{Types of Linacs}
\begin{frame}[containsverbatim]
\frametitle{Electron Linacs}
\begin{itemize}
	\item Basically the speed of light after gun (large $\beta$)
	\item Commonly use copper and superconducting cavities
	\item Less radiation compared to protons/ions
	\item Used for light sources (synchrotrons, FEL)
	\item FEL = Free Electron Laser
\end{itemize}

\includegraphics[width=\textwidth]{awa-drawing}
%\vspace{0.25em}

\hfill Picture Source: AWA-ANL
\end{frame}


\begin{frame}
	\frametitle{Proton/Ion Linacs}

\begin{minipage}{0.45\textwidth}
	\begin{itemize}
		\item Low $\beta$
		\item Copper and superconducting
		\item FRIB - here at MSU
	\end{itemize}
	\centering

	\includegraphics[width=\textwidth]{outside_proton_linac}
	Outside of Fermi proton linac.
\end{minipage}\hfill
\begin{minipage}{0.45\textwidth}
	\vspace{-1em}
	\centering
	Inside of Fermi proton linac.
	\includegraphics[width=\textwidth]{inside_proton_linac}
	Picture(s) Source: E. Preybs
\end{minipage}
\end{frame}


%%%%%%%%%%%%%%%%%%%%%%%%%%%%%%%%%%%%%%%%%%%%%%%%%%%%%%%%%%%%%%%%%%%%%%%%%%%%%%%%

\section{Waveguides}
%\subsection{Rectangular Waveguide}
\begin{frame}
\frametitle{Some terms that might be useful...}
\begin{itemize}
	\item Radio Frequency (RF): 3Hz - 3 THz
	\item Microwaves: 300 MHz to 300 GHz 
	\begin{itemize}
		\item L band: 1-3 GHz
		\item S band: 2-4 GHz
		\item C band: 4-8 GHz
		\item X band: 8-12 GHz
	\item Waveguide: used for high power transmission to cavities
	\end{itemize}

\vspace{1em}

Most electron copper or superconducting linacs 
operate in L~and S band (in my experience). 

\vspace{1em}
\url{https://en.wikipedia.org/wiki/Radio_spectrum}
\url{https://en.wikipedia.org/wiki/Microwave}
\end{itemize}
\end{frame}


\begin{frame}
	\frametitle{Rectangular Waveguides}
	The power cable for cavities!
	*Include picture here*
\end{frame}
\begin{frame}
\frametitle{Rectangular Waveguide}
\begin{center}
	\includegraphics[width=0.8\textwidth]{rect_wg}
\end{center}

Source L. Spentzouris and:

\tiny\url{http://www.kathrynindiana.com/pages/science/Physics/waveguides.html}
\end{frame}

\begin{frame}
	Start with plane waves (electric and magnetic):
	\begin{equation}
		E\left(z,t\right) =E_0\,e^{j\left(kz-\omega t\right)}
	\end{equation}
	\begin{equation}
		B\left(z,t\right) =B_0\,e^{j\left(kz-\omega t\right)}
	\end{equation}
	We have to introduce boundary conditions (metallic walls):
		\begin{equation}
			B_x\left(x=0,y\right) = B_x\left(x=a,y\right)= 0
		\end{equation}
		\begin{equation}
			B_y\left(x,y=0\right) = B_y\left(x,y=b\right)=0
		\end{equation}
	
\end{frame}

\subsection{Circular Waveguides}
\begin{frame}
\frametitle{Alireza Nassiri and Geoff Waldschmidt}
\centering
	\includegraphics[width=0.8\textwidth]{circ_modes}
	
	Source: L. Spentzouris
\end{frame}
\begin{frame}
\frametitle{Alireza Nassiri and Geoff Waldschmidt}
\centering
	\includegraphics[width=0.6\textwidth]{coax_modes}
	\vspace{-1em}
	
	Source: L. Spentzouris
\end{frame}


%%%%%%%%%%%%%%%%%%%%%%%%%%%%%%%%%%%%%%%%%%%%%%%%%%%%%%%%%%%%%%%%%%%%%%%%%%%%%%%%
\section{Accelerating Structures}
%\subsection{Pill Box}
\begin{frame}
\frametitle{Accelerating Structures}
Shape and material of accelerating structures depends heavily on: 
\begin{itemize}
	\item particle type (electron, proton, ions)
	\item Beta, $\beta$ after source
	\begin{itemize}
		\item large $\beta$ for electrons
		\item low $\beta$ for protons and ions
	\end{itemize}
	\item Final energy requirement
\end{itemize}
\end{frame}

\begin{frame}
	\frametitle{Pillbox Cavity}
	similarity to wg, fundamental mode (first zero of the Bessel), 
	circuit equation, 
\end{frame}


\subsection{Iris Loaded}
\begin{frame}
	\frametitle{}
	Give example of common copper structures. 
\end{frame}

\begin{frame}
\frametitle{Proton, Ion}
Spoke cavities used for low Beta ($\beta$) particles:
\vspace{0.25em}

\centering
\begin{minipage}{0.45\textwidth}
	\centering
	\includegraphics[width=0.2\textheight]{spoke1}
	
	Half-wave resonator
\end{minipage}\hspace{-1em}
\begin{minipage}{0.5\textwidth}
	\centering
	\includegraphics[width=\textwidth]{spoke2}
	
	(Triple) spoke resonator
\end{minipage}

\vspace{1em}
\hfill Source: E. Prebys
\end{frame}
%%%%%%%%%%%%%%%%%%%%%%%%%%%%%%%%%%%%%%%%%%%%%%%%%%%%%%%%%%%%%%%%%%%%%%%%%%%%%%%%



%%%%%%%%%%%%%%%%%%%%%%%%%%%%%%%%%%%%%%%%%%%%%%%%%%%%%%%%%%%%%%%%%%%%%%%%%%%%%%%%
\section{Power and Energy }
%\subsection{Power Calcs}
\begin{frame}
	\frametitle{Quality Factor}
	\begin{equation}
		Q= \frac{\omega U}{P}
	\end{equation}
\end{frame}

\begin{frame}
	\frametitle{Gradient}
	The electric field strength (*on axis) in a cavity 
	due to an externally applied power is:
	\begin{equation}
		E^2_z = \frac{P\omega}{v_g} \frac{R}{Q}
	\end{equation}
	\begin{itemize}
		\item $E =$ electric field on axis
		\item $P =$ power supplied to cavity
		\item $v_g =$ group velocity
		\item $R =$ Shunt impedance 
		\item $Q =$ Quality factor of cavity
	\end{itemize}

\vspace{0.5em}
This field contributes to the gradient of a cavity.
\end{frame}

\subsection{Energy Calculation}
\begin{frame}
\frametitle{Energy Gain}
Given a gradient based on the information in the last slides...
Calculating the expected energy gain is proportional to the gradient and length of the accelerating structure. 

\begin{equation}
	\Delta W = qE_zTL
\end{equation}

\begin{itemize}
	\item $\Delta W = $ change in beam energy
	\item $q =$ charge of particles
	\item $E_zT = $ accelerating gradient
	\item $L = $ length of accelerating structure or cell 
\end{itemize}

\vspace{1em}
Source: Chp. 2, "RF Linear Accelerators", T. Wangler
\end{frame}


\begin{frame}
\frametitle{Transit Factor - Protons/Ions}
\vspace{1em}
Source: Chp. 2, "RF Linear Accelerators", T. Wangler
\end{frame}

\begin{frame}
	\frametitle{TBA Example}

	
	Two beam acceleration is a candidate technology for future machines.
	CLIC, FELs, etc...
	
	TBA scheme, how measurements are done. 
How energy measurements are done, spectrometer.
Calc energy from RF gradient
calc shunt impedance and Q
Do TBA on the board.	

	Note, deceleration can occur as well...
	
\end{frame}



\section{Energy Measurements}
\begin{frame}
	\frametitle{Energy Measurements}
	Trajectory of a beam through a dipole is proportional to it's energy:
	\begin{equation}
		B \rho = 0.2998 \, \beta \, E \, [GeV]
	\end{equation}
	\begin{itemize}
		\item $B = $ magnetic field in Tesla
		\item $\rho =$ bending radius through magnet in meters
		\item $\beta =$ velocity of beam
		\item $E = $ energy of beam in GeV 
	\end{itemize}

\vspace{1em}
Reminder from Monday:

$B\rho$ is called "beam rigidity"

\vspace{1em}
Source: "Particle Accelerator Physics", H. Wiedemann
\end{frame}

\subsection{Experimental}
\begin{frame}
\frametitle{Energy Measurements}
\centering
\vspace{1em}

\begin{minipage}{0.35\textwidth}
Hardware needed to do measurement:
\begin{itemize}
	\item Dipole
	\item Imaging screen(s)
	\item Actuators
	\item Current monitors
\end{itemize}	
\end{minipage}\hfill \hfill
\begin{minipage}{0.6\textwidth}
\begin{tikzpicture}[every node/.style={anchor=south west,inner sep=0pt},x=1mm, y=1mm,]   
\node (fig1) at (0,0)
{\includegraphics[width=\textwidth]{awa_dipole}};
\node[fill=white, inner sep=2pt] (txt2) at (32,45) {Actuator};
\node[fill=white, inner sep=2pt] (txt2) at (30,11) {Window};	
\node[fill=white, inner sep=2pt] (txt2) at (12,32) {Dipole};
\end{tikzpicture}
\end{minipage}

\vspace{1em}
\hfill Source Picture: AWA-ANL
\end{frame}

%%%%%%%%%%%%%%%%%%%%%%%%%%%%%%%%%%%%%%%%%%%%%%%%%%%%%%%%%%%%%%%%%%%%%%%%%%%%%%%%
\begin{frame}
	\frametitle{Summary}

	\large{Thanks for your attention!}
\end{frame}
%%%%%%%%%%%%%%%%%%%%%%%%%%%%%%%%%%%%%%%%%%%%%%%%%%%%%%%%%%%%%%%%%%%%%%%%%%%%%%%%


\end{document}
















