%%%%%%%%%%%%%%%%%%%%%%%%%%%%%%%%%%%%%%%%%%%%%%%%%%%%%%%%%%%%%%%%%%%%%%
%%%%%%%%% Select one of the options, and comment the rest of them

%%%%%%%%%% Option 1:  to compile with pdflatex : parameter "t" - to align to the top
\documentclass[professionalfonts,t]{beamer}
%sans font?

%%%%%%%%%% Option 3: to create handout for print
%\documentclass[t,handout]{beamer}
%\usepackage{pgfpages}              % to put several slides on one page
%\pgfpagesuselayout{2 on 1}[a4paper, border shrink=5mm]             % 2 slides on 1 page
%\pgfpagesuselayout{4 on 1}[a4paper,landscape, border shrink=5mm]   % 4 slides on 1 page, and landscaped


%%%%%%%%%%%%%%%%%%%%%%%%%%%%%%%%%%%%%%%%%%%%%%%%%%%%%%%%%%%%%%%%%%%%
%%%%%%%%%%%%%% Select the Theme %%%%%%%%%%%%%%%%%%%%%%%%%%%%%%%%%%%
\usetheme{Dresden}     % OK
%\usetheme{Berlin}
%\usetheme{Bergen}      % NO
%\usetheme{Boadilla}    % NO
%\usetheme{Copenhagen}  % NO
%\usetheme{Hannover}    % NO
%\usetheme{Luebeck}     % NO
%\usetheme{Marburg}     % NO
%\usetheme{Pittsburgh}  % NO
%\usetheme{default}
%\usetheme{Singapore}   % OK
%\usetheme{boxes}
%\usecolortheme{structure}
%\usecolortheme{rose}
%\usecolortheme{beaver}


\definecolor{mymaroon}{cmyk}{0.0, 1.0, 1.0, 0.498}
\definecolor{myblue}{cmyk}{1.0, 1, 0, 0.5}
\definecolor{mygreen}{cmyk}{100, 0, 100, 50}
\setbeamercolor*{palette secondary}{use=structure,fg=white,bg=myblue}
\setbeamercolor*{palette tertiary}{use=structure,fg=white,bg=mymaroon}

%\usepackage{beamerthemesplit}              %
\beamertemplateballitem % fancy bullets and numbering

\setbeamertemplate{navigation symbols}{}   % suppress navigation symbols
\addtobeamertemplate{frametitle}{}{%
	\logo{../images/IIT_logo}
	\iffalse
	
	\begin{tikzpicture}[remember picture,overlay]
	\node[anchor=center, yshift=-13pt, xshift=-5pt] at (current page.north) 
	{\includegraphics[height=1.1cm]{../images/Argonne_cmyk_black-eps-converted-to}\hspace{10cm}};
	
	\node[anchor=north east, yshift=3pt, xshift=0pt] at (current page.north east) 
	{\includegraphics[height=0.7cm]{../images/IIT_Logo_blk}};
	\end{tikzpicture}
    
     \fi
}
% other possibilities to include LOGO. it puts it in RLC

%
%\pgfdeclareimage[width=1cm]{logo}{../images/IIT_Logo}
%\logo{\pgfuseimage{logo}}


% load additional packages

\usepackage{xcolor}
\usepackage{graphicx}
\usepackage{amsmath}
\usepackage{amssymb}
\usepackage{amsthm}
\usepackage{graphicx}
\usepackage{url}
\usepackage{color}
\usepackage{booktabs} % Allows the use of \toprule, \midrule and \bottomrule in tables
\usepackage{pifont}% http://ctan.org/pkg/pifont
\usepackage{epstopdf}
\usepackage{xmpmulti}
\usepackage[export]{adjustbox}
\usepackage{tikz}
\usetikzlibrary{shapes.misc}
\usetikzlibrary{shapes,arrows,decorations.markings,shadows,positioning}

% Your Abbreviations
\newcommand\bE{{\mathbb{E}}}
\newcommand\bR{{\mathbb{R}}}
\newcommand\bH{{\mathbf{H}}}
% End abbreviations

\newcommand\Wider[2][3em]{%
	\makebox[\linewidth][c]{%
		\begin{minipage}{\dimexpr\textwidth+#1\relax}
			\raggedright#2
		\end{minipage}%
	}%\textbf{}
}

%%%%%%%%%%%%%%%%%%%%% to edit the main text below
%NOTES ON SOME TECHNICS
%%%% Box %%%%%%%%%%%%%%%%%%%%%%%%%%%%%%%%%%%%%%%%%%%%%%%
%{\fbox{ \parbox[t]{10cm}{ SOME TEXT }}}

%%% include a picture. The file should be with extention EPS, e.g. FILENAME.EPS
%\begin{figure}[h]
%\centering
%\includegraphics[width=.7\linewidth]{FILENAME}
%\caption{{\footnotesize PUT_CAPTION }}
%\end{figure}

%\subtitle{}
%\institute[ANL/IIT]{Argonne National Laboratory\\Illinois Institute of Technology}

\title[USPAS Fundamentals, June 4-15, 2018]{Accelerating Structures and Linear Machines}
\author[N.Neveu]{{\Large Nicole Neveu}}
\institute[ANL, IIT] % (optional, but mostly needed)
{   Illinois Institute of Technology \\
	Argonne National Laboratory \\
    \url{nneveu@anl.gov} 
}
% - Use the \inst command only if there are several affiliations.
% - Keep it simple, no one is interested in your street address.
\date{ June 8, 2018 \\ \vspace{0.5em}
\includegraphics[width=3.5cm,keepaspectratio]{../../logos/Argonne_cmyk_black}%
\hfill \hfill \hfill%
\includegraphics[width=4cm,keepaspectratio]{../../logos/IIT_Logo_blk-eps-converted-to}%
}

%\date[IIT, April 2009]{
%           Space Charge 2017 \\ Oc 18, 2009  }



%%%%%%%%%%%%%%%%%%%%%%%%%%%%%%Section title frame 
\AtBeginSection[]{
	\begin{frame}
	\vfill
	\centering

	\begin{minipage}{0.45\textwidth}
	%\begin{beamercolorbox}[sep=8pt,center,shadow=true,rounded=true]{title}
		\tableofcontents[currentsection]
		%\usebeamerfont{title}\insertsectionhead\par%
	%\end{beamercolorbox}	
	\end{minipage}\hfill
	\begin{minipage}{0.5\textwidth}
	\includegraphics[width=4cm]{\secimage}
	
	%Source: Fermilab Media
	\end{minipage}
	%\vfill 
	\end{frame}
}
%%%%%%%%%%%%%%%%%%%%%%%%%%%%%%Section title frame

\newcommand{\secimage}{linda_linac}

\begin{document}


\iftrue




\begin{frame}
  \titlepage
\end{frame}
\begin{frame}
	\frametitle{About me...}
	\vspace{-1em}
	
	\begin{itemize}
		\item 20013: BS Electrical Engineering, University of Houston 
		\item 2013-2018: PhD Student, Illinois Institute of Technology
		\begin{itemize}
			\item Thesis work on beam line design
			\item Part of the Argonne Wakefield Accelerator Group (ANL)
			\item Lots of simulation work
			\item Comparison to experimental measurements
		\end{itemize}
	\end{itemize}

\centering

\Wider{
\includegraphics[width=0.8\textheight]{awa_group}\includegraphics[width=0.8\textheight]{cyclotron}
}
\end{frame}


\begin{frame}
	\frametitle{Outline}
	\begin{minipage}{0.5\textwidth}
		\tableofcontents
	\end{minipage}\hfill
\begin{minipage}{0.45\textwidth}
	\centering
	\includegraphics[width=0.85\textwidth]{linda_linac}
	Source: Fermilab Media
\end{minipage}
\end{frame}

%\begin{frame}{Outline of the talk}
%  \tableofcontents
%  % You might wish to add the option [pausesections]
%\end{frame}


% Structuring a talk is a difficult task and the following structure
% may not be suitable. Here are some rules that apply for this
% solution:

% - Exactly two or three sections (other than the summary).
% - At *most* three subsections per section.
% - Talk about 30s to 2min per frame. So there should be between about
%   15 and 30 frames, all told.

% - A conference audience is likely to know very little of what you
%   are going to talk about. So *simplify*!
% - In a 20min talk, getting the main ideas across is hard
%   enough. Leave out details, even if it means being less precise than
%   you think necessary.
% - If you omit details that are vital to the proof/implementation,
%   just say so once. Everybody will be happy with that.


\section{Linacs}
%\subsection{Introduction}

\begin{frame}
	\frametitle{Linear Accelerators (Linacs)}
	
	\vspace{1em}
	
	\begin{minipage}{0.43\textwidth}
		Goals for this talk:
		
		\begin{itemize}
			\item What is a linac?
			\item Why do we need them?
			\item How do they work (conceptually)?
		\end{itemize}
	\end{minipage}
\begin{minipage}{0.55\textwidth}
	\centering
	\includegraphics[width=0.95\textwidth]{awa_linac}
	
	Source: AWA-ANL 
\end{minipage}

\vspace{1em}
Note: It's a general overview of some common machine types and techniques.
Not a complete and thorough review of all machines!

\end{frame}

\begin{frame}
\frametitle{Some Uses for Linacs}
\vspace{-1em}
\begin{minipage}{0.45\textwidth}
	\begin{itemize}
		\item Colliders 
		\item Injectors for circular machines
		\item Light sources to produce x-rays for experiments:
		\begin{itemize}
			\item Chemistry, Biology, Material Science, Engineering, etc.
		\end{itemize}
		\item Medical accelerators 
		\begin{itemize}
			\item Cancer therapy 
			\item Isotope production
		\end{itemize}
		\item Semiconductor industry
	\end{itemize}
\end{minipage}\hfill
\begin{minipage}{0.5\textwidth}
	\centering
	\vspace{-1.5em}
	
	\small Proton Improvement Plan (PIP-II)
	\includegraphics[width=0.9\textwidth]{pip2}
	
	\vspace{-0.25em}
	Source: Fermilab, E. Prebys
\end{minipage}
\end{frame}


\subsection{Types of Linacs}
\begin{frame}[containsverbatim]
\frametitle{Electron Linacs}
\begin{itemize}
	\item Usually the speed of light after gun (large $\beta$)
	\item Commonly use copper and superconducting cavities
	\item Usually less radiation compared to protons/ions
	\item Used for light sources (synchrotrons, FEL)
	\item FEL = Free Electron Laser
\end{itemize}

\includegraphics[width=\textwidth]{awa-drawing}
%\vspace{0.25em}

\hfill Source: AWA-ANL
\end{frame}


\begin{frame}
	\frametitle{Proton/Ion Linacs}

\begin{minipage}{0.45\textwidth}
	\begin{itemize}
		\item Low $\beta$
		\item Copper and superconducting
		\item FRIB - here at MSU
	\end{itemize}
	\centering

	\includegraphics[width=\textwidth]{outside_proton_linac}
	Low energy Fermi proton linac.
\end{minipage}\hfill
\begin{minipage}{0.45\textwidth}
	\vspace{-1em}
	\centering
	Inside of Fermi proton linac.
	\includegraphics[width=\textwidth]{inside_proton_linac}
	Source: Fermilab, E. Preybs
\end{minipage}
\end{frame}

\fi
%%%%%%%%%%%%%%%%%%%%%%%%%%%%%%%%%%%%%%%%%%%%%%%%%%%%%%%%%%%%%%%%%%%%%%%%%%%%%%%%

\iftrue


{
	\renewcommand{\secimage}{gun_wg}


\section{Waveguides}
%\subsection{Rectangular Waveguide}
\begin{frame}
\frametitle{Some terms that might be useful...}
\begin{itemize}
	\item Radio Frequency (RF): 3Hz - 3 THz
	\item Microwaves: 300 MHz to 300 GHz 
	\begin{itemize}
		\item L band: 1-3 GHz
		\item S band: 2-4 GHz
		\item C band: 4-8 GHz
		\item X band: 8-12 GHz
	\item Waveguide: used for high power transmission to cavities
	\end{itemize}

\vspace{1em}

Most electron copper or superconducting linacs 
operate in L~and S band (in my experience). 

\vspace{1em}
\url{https://en.wikipedia.org/wiki/Radio_spectrum}
\url{https://en.wikipedia.org/wiki/Microwave}
\end{itemize}
\end{frame}


\begin{frame}
	\frametitle{Waveguides in real life...}
	The power cable for cavities! Why not use cable (coax)?
	
	\centering
	\vspace{0.5em}
	
	\centering
	\Wider{
		\includegraphics[width=0.5\textheight]{linac_wg}\hspace{0.5em}%
		\includegraphics[width=0.5\textheight]{waveguide_and_cavity}\hspace{0.5em}%
		\includegraphics[width=0.5\textheight]{gun_wg}
	}

\vspace{-19em}
\hfill Source: AWA-ANL
\end{frame}

\begin{frame}
	\frametitle{More waveguides...}
	We use L band, 1.3 GHz klystrons. 
	
	This is the roof of the AWA-ANL bunker.
	\vspace{1em}
	
	\Wider{
		\centering
	\includegraphics[width=0.49\linewidth]{roof_wg}\hspace{0.5em}%
	\includegraphics[width=0.49\linewidth]{roof_wg2}
}

\hfill Source: AWA-ANL
\end{frame}


\begin{frame}
	\frametitle{Klystrons}
	Mini accelerators that generate high power RF waves.
	\Wider{
	\begin{minipage}{0.5\linewidth}
		\includegraphics[width=\linewidth]{klystron3}
		
		
	\end{minipage}
	\begin{minipage}{0.45\textwidth}
		\includegraphics[width=0.5\textwidth]{klystron2}
	\end{minipage}
}

Video of how klystrons work:

\url{https://www.youtube.com/watch?v=TsBTI3tO5-8}

\small Source: \url{https://en.wikipedia.org/wiki/Klystron} , E. Prebys
\end{frame}

\subsection{Rectangular Waveguides}
\begin{frame}
\frametitle{Rectangular Waveguide}
\begin{center}
	\includegraphics[width=0.8\textwidth]{rect_wg}
\end{center}

Source: L. Spentzouris, and...

\tiny\url{http://www.kathrynindiana.com/pages/science/Physics/waveguides.html}
\end{frame}

\begin{frame}
\frametitle{Rectangular Waveguide}
\begin{minipage}{0.6\linewidth}
	The waves that propagate inside the waveguide depend on the size, 
	shape, and filling material.
	Some waveguides are pumped with gas to prevent electric breakdown. 
	\begin{itemize}
		\item TE$_{nm}$ = Transverse electric field
		\item TM$_{nm}$ = Transverse magnetic field
		\item Several modes can propagate at a time (usually not good)
	\end{itemize}
\end{minipage}\hfill
\begin{minipage}{0.35\linewidth}
	\includegraphics[width=\linewidth]{rect_modes}
	Source: L. Spentzouris
\end{minipage}
\end{frame}



\begin{frame}
\frametitle{Rectangular Waveguide Derivation}
		Start with plane waves (electric and magnetic):
	\begin{equation}
	\vec{E}\left(x,y,z,t\right) =\vec{E}\left(x,y\right)\,e^{j\left(kz-\omega t\right)}
	\end{equation}
	\begin{equation}
	\vec{B}\left(x,y,z,t\right) =\vec{B}\left(x,y\right)\,e^{j\left(kz-\omega t\right)}
	\end{equation}
	
	These equations can be written
	\begin{equation}
		\vec{B}\left(x,y,z,t\right) =X\left(x\right)Y\left(y\right)\,e^{j\left(kz-\omega t\right)}
	\end{equation}
	
	Get the wave equation from Maxwell's equations...
	\begin{equation}
		\left(\frac{\partial^2}{\partial x^2} + \frac{\partial^2}{\partial y^2} + k^2_c\right)
		B_z\left(x,y\right)=0
	\end{equation}
\end{frame}


\begin{frame}
\frametitle{TE Boundary Conditions}
\vspace{-0.5em}
	Due to metallic walls:
		\begin{equation}
			B_x\left(x=0,y\right) = B_x\left(x=a,y\right)= 0
		\end{equation}
		\begin{equation}
			B_y\left(x,y=0\right) = B_y\left(x,y=b\right)=0
		\end{equation}
		
\vspace{-1em}

\begin{center}
\includegraphics[width=0.5\textwidth]{waveguide}
\end{center}

\vspace{-1.5em}

Figure and derivation of resulting fields here:	
\small\url{http://uspas.fnal.gov/materials/10MIT/Lecture5.pdf}
\end{frame}

\begin{frame}
	\frametitle{Cutoff Frequencies}
	Cut off frequencies tell us what waves will propagate in the waveguides. 
	Derived from Maxwell's equations and boundary conditions.
	\begin{equation}
		f_{nm} = \frac{1}{2\sqrt{\mu_0 \epsilon_0}}\sqrt{\left(\frac{n}{a}\right)^2+\left(\frac{m}{b}\right)^2}
	\end{equation}
	
	In class exercise:
	\begin{enumerate}
		\item Calculate the cutoff frequencies for the following modes: a=0.02, b=0.06
		\begin{itemize}
			\item $f_{01}$, $f_{10}$, $f_{20}$, $f_{11}$
		\end{itemize}
		\item In what frequency range is only one mode propagating?
		\item In what frequency range are three modes propagating?
		\item Is it good or bad to have more than one mode propagating?
	\end{enumerate}	
\end{frame}


\subsection{Circular Waveguides}
\begin{frame}
\frametitle{Alireza Nassiri and Geoff Waldschmidt}
This theory applies to circular geometries too.

\centering
	\includegraphics[width=0.8\textwidth]{circ_modes}
	\vspace{-1em}
\end{frame}
\begin{frame}
\frametitle{Alireza Nassiri and Geoff Waldschmidt}
Coaxial (coax) cables are a commonly used example of this.

\centering
	\includegraphics[width=0.6\textwidth]{coax_modes}
	\vspace{-1em}
\end{frame}

}

\fi
%%%%%%%%%%%%%%%%%%%%%%%%%%%%%%%%%%%%%%%%%%%%%%%%%%%%%%%%%%%%%%%%%%%%%%%%%%%%%%%%


\iftrue

\section{Accelerating Structures}
\begin{frame}
\frametitle{Accelerating Structures}

\Wider{

\begin{minipage}{0.5\textwidth}
		Shape and material of acc. structures depends heavily on: 
	\begin{itemize}
		\item particle type (electron, proton, ions)
		\item Beta, $\beta$ after source
		\begin{itemize}
			\item large $\beta$ for electrons
			\item low $\beta$ for protons/ions
		\end{itemize}
		\item Final energy requirement
		\item Continuous or pulsed operation?
		\vspace{-1em}
		\begin{itemize}
			\item This determines superconducting or not!
			\item SC saves power, but cryoplant is expensive.
		\end{itemize}
	\end{itemize}
\end{minipage}\hfill
\begin{minipage}{0.45\textwidth}
	\includegraphics[width=0.8\linewidth]{cryo}
	Source: Fermilab, E. Harms
\end{minipage}

}
\end{frame}


\subsection{Common Structures}
\begin{frame}
	\frametitle{Normal Conducting}
	\vspace{-0.5em}
	\includegraphics[width=1\textwidth]{copper1}\\%
	\includegraphics[width=0.5\textwidth]{tw}\hfill%
	\includegraphics[width=0.45\textwidth]{copper2}

Source: Fermilab, SLAC
\end{frame}
\begin{frame}
	\frametitle{Superconducting}
		\vspace{-0.5em}
	LCLS-II, Fermilab, Jlab, Europe, Japan, etc...

	
	\centering
	
	\includegraphics[width=0.75\textheight]{sc_elvin}%
	\includegraphics[width=0.5\textwidth]{pi_cavity}\vfill 
	\includegraphics[width=0.53\textwidth]{cryo2}\hspace{1em}%
	\includegraphics[width=0.25\textwidth]{cryo3}\\
	
	Sources: Fermilab, E. Harms, E. Prebys
\end{frame}


\begin{frame}
\frametitle{Proton, Ion}
Spoke cavities used for low Beta ($\beta$) particles:
\vspace{0.25em}

\centering
\begin{minipage}{0.45\textwidth}
	\centering
	\includegraphics[width=0.2\textheight]{spoke1}
	
	Half-wave resonator
\end{minipage}\hspace{-1em}
\begin{minipage}{0.5\textwidth}
	\centering
	\includegraphics[width=\textwidth]{spoke2}
	
	(Triple) spoke resonator
\end{minipage}

\vspace{1em}
\hfill Source: E. Prebys
\end{frame}


\begin{frame}
\frametitle{Pillbox Cavity}
\vspace{-2em}
\hfill Source: T. Wangler, E. Prebys

\begin{minipage}{0.6\textwidth}
	\includegraphics[width=\textwidth]{pillbox}
	
	Boundary conditions +
\end{minipage}
\begin{minipage}{0.34\textwidth}
	\begin{equation}
	\vec{E} = \vec{E}\left(r,t\right)\hat{z}
	\end{equation}
	\begin{equation}
	\vec{B} = \vec{B}\left(r,t\right)\hat{\phi}
	\end{equation}
	\begin{equation}
		E_z = E \left(r\right) e^{i \omega t} \label{p1}
	\end{equation}
\end{minipage}

Using Maxwell's and a wave equation again:

\begin{equation}
\frac{\partial^2 E_z}{\partial r^2} + \frac{1}{r} \frac{\partial E_z}{\partial r} = \frac{1}{c^2} \frac{\partial^2 E_z}{\partial t^2} \label{p2}
\end{equation}

Assume Ez is the form of eq. \ref{p1} to solve PDE in eq. \ref{p2}.  
\end{frame}


\begin{frame}
\frametitle{Bessel Functions}
\vspace{-1em}
\hfill Source: T. Wangler, E. Prebys
	
	Solutions include Bessel functions. 
	
	0th order gives the first mode of the cavity.

	\begin{minipage}{0.5\textwidth}
		\includegraphics[width=\textwidth]{bessel}
	\end{minipage}
	\begin{minipage}{0.4\textwidth}
	\begin{equation}
		E_z\left(r\right) = E_0 J_0 \left(\frac{\omega}{c} r\right)
	\end{equation}
	
	First zero at J(2.405):
	\begin{equation}
		f_0 = 2.405 \frac{c}{2 \pi R}
	\end{equation}
	\end{minipage}

\end{frame}

\fi
%%%%%%%%%%%%%%%%%%%%%%%%%%%%%%%%%%%%%%%%%%%%%%%%%%%%%%%%%%%%%%%%%%%%%%%%%%%%%%%%



%%%%%%%%%%%%%%%%%%%%%%%%%%%%%%%%%%%%%%%%%%%%%%%%%%%%%%%%%%%%%%%%%%%%%%%%%%%%%%%%
\iftrue

\section{Power and Energy in Cavities }
%\subsection{Power Calcs}

\begin{frame}
	\frametitle{Quality Factor}
	A measure of how fast power is dissipated. 
	High Q means slower power loss, i.e. oscillations in resonator die out more slowly. 
	\url{https://en.wikipedia.org/wiki/Q_factor}
	\begin{equation}
		Q= \frac{\omega U}{P}
	\end{equation}
	\begin{itemize}
		\item $Q =$ quality factor
		\item $\omega =$ frequency of cavity
		\item $U =$ stored energy in the cavity 
	\end{itemize}

\vspace{1em}

Source: Chp. 2, "RF Linear Accelerators", T. Wangler
\end{frame}

\begin{frame}
	\frametitle{Gradient}
	The electric field strength (*on axis) in a cavity 
	due to an externally applied power is:
	\begin{equation}
		E^2_z = \frac{P\omega}{v_g} \frac{R}{Q}
	\end{equation}
	\begin{itemize}
		\item $E =$ electric field on axis
		\begin{itemize}
			\item This field contributes to the gradient of a cavity.
		\end{itemize}
		\item $P =$ power supplied to cavity
		\item $v_g =$ group velocity
		\item $R =$ Shunt impedance 
		\item $Q =$ Quality factor of cavity
	\end{itemize}

\end{frame}

\subsection{Energy Calculation}
\begin{frame}
\frametitle{Energy Gain}
Given a gradient based on the information in the last slides...
Calculating the expected energy gain is proportional to the gradient and length of the accelerating structure. 

\begin{equation}
	\Delta W = qE_zTL\, cos\phi
\end{equation}

\begin{itemize}
	\item $\Delta W = $ change in beam energy (also $\Delta E$ sometimes)
	\begin{itemize}
		\item "on crest": $ \phi=0$
	\end{itemize}
	\item $q =$ charge of particles
	\item $E_zT = $ accelerating gradient
	\item $L = $ length of accelerating structure or cell 
\end{itemize}

\vspace{0.25em}
Source: Chp. 2, "RF Linear Accelerators", T. Wangler
\end{frame}


\begin{frame}
\frametitle{Transit Factor (T) - Protons/Ions}

\Wider{
\begin{itemize}
	\item Need to account for this in drift tube linacs (protons, ions), etc.
	\item Usually not an issue for high $\beta$ electron machines 
	\begin{itemize}
		\item Energy gain equation reduces to: $\Delta W = qE_zL$
		\item On crest energy gain
	\end{itemize}

\url{http://uspas.fnal.gov/materials/09VU/Lecture4.pdf}
\end{itemize}

\centering
\begin{minipage}{0.6\textwidth}
	\includegraphics[width=0.8\textheight]{drift_transit}
\end{minipage}
\begin{minipage}{0.3\textwidth}
	\begin{equation}
	d = \frac{v}{f}
	\end{equation}
\end{minipage}
}

\vspace{1em}
Let's watch a gif!

\vspace{1em}
Source: Chp. 2, "RF Linear Accelerators", T. Wangler
\end{frame}

\begin{frame}
	\frametitle{RLC Circuit Model of Cavity}
	%\vspace{-1em}
		\begin{minipage}{0.5\textwidth}
		\includegraphics[width=\textwidth]{rlc}
	\end{minipage}\hfill
	\begin{minipage}{0.45\textwidth}
		We can model a resonant (accelerating) cavity as a RLC circuit. 
		Circuit analysis tells us the impedance (similar to resistance) is:
	\end{minipage}

\vspace{1em}

\Wider{
\begin{equation}
Z =  \left| \frac{V_0}{I_0} \right| = \frac{1}{\sqrt{\left(\frac{1}{R}\right)^2
	+ \left(\omega C - \frac{1}{\omega L}\right)}} 
\end{equation}	
}
\vspace{1em}

	In class practice problem: R=8, L=0.2, C=0.8 
	
	What does a plot of $\left| \frac{ V_0 }{I_0} \right| $
	vs. $\omega$ look like? What does this mean?
\end{frame}

\fi

\iftrue
\section{Energy Measurements}
\begin{frame}
	\frametitle{Energy Measurements}
	Trajectory of a beam through a dipole is proportional to it's energy:
	\begin{equation}
		B \rho = 0.2998 \, \beta \, E \, [GeV]
	\end{equation}
	\begin{itemize}
		\item $B = $ magnetic field in Tesla
		\item $\rho =$ bending radius through magnet in meters
		\item $\beta =$ velocity of beam
		\item $E = $ energy of beam in GeV 
	\end{itemize}

\vspace{1em}
Reminder from Monday:

$B\rho$ is called "beam rigidity"

\vspace{1em}
Source: "Particle Accelerator Physics", H. Wiedemann
\end{frame}


\subsection{Experimental}
\begin{frame}
\frametitle{Energy Measurements}
\centering
\vspace{1em}

\begin{minipage}{0.35\textwidth}
Hardware needed to do measurement:
\begin{itemize}
	\item Dipole
	\item Imaging screen(s)
	\item Actuators
	\item Current monitors
\end{itemize}	
\end{minipage}\hfill \hfill
\begin{minipage}{0.6\textwidth}
\begin{tikzpicture}[every node/.style={anchor=south west,inner sep=0pt},x=1mm, y=1mm,]   
\node (fig1) at (0,0)
{\includegraphics[width=\textwidth]{awa_dipole}};
\node[fill=white, inner sep=2pt] (txt2) at (32,45) {Actuator};
\node[fill=white, inner sep=2pt] (txt2) at (30,11) {Window};	
\node[fill=white, inner sep=2pt] (txt2) at (12,32) {Dipole};
\end{tikzpicture}
\end{minipage}

\vspace{1em}
\hfill Source Picture: AWA-ANL
\end{frame}


%%%%%%%%%%%%%%%%%%%%%%%%%%%%%%%%%%%%%%%%%%%%%%%%%%%%%%%%%%%%%%%%%%%%%%%%%%%%%%%%
\begin{frame}
	\frametitle{Summary}
	\begin{itemize}
		\item Linacs
		\item Rectangular waveguides
		\item Accelerating Structures
		\item Q, gradient, and energy calculations
		\item Klystron $\rightarrow$ waveguide $\rightarrow$ cavity $\rightarrow$ beam
	\end{itemize}

\vspace{1em}

	\centering
	\large{Thanks for your attention!}
\end{frame}
%%%%%%%%%%%%%%%%%%%%%%%%%%%%%%%%%%%%%%%%%%%%%%%%%%%%%%%%%%%%%%%%%%%%%%%%%%%%%%%%
\fi

\end{document}
















