%%%%%%%%%%%%%%%%%%%%%%%%%%%%%%%%%%%%%%%%%%%%%%%%%%%%%%%%%%%%%%%%%%%%%%
%%%%%%%%% Select one of the options, and comment the rest of them

%%%%%%%%%% Option 1:  to compile with pdflatex : parameter "t" - to align to the top
\documentclass[professionalfonts,t]{beamer}
%sans font?

%%%%%%%%%% Option 3: to create handout for print
%\documentclass[t,handout]{beamer}
%\usepackage{pgfpages}              % to put several slides on one page
%\pgfpagesuselayout{2 on 1}[a4paper, border shrink=5mm]             % 2 slides on 1 page
%\pgfpagesuselayout{4 on 1}[a4paper,landscape, border shrink=5mm]   % 4 slides on 1 page, and landscaped


%%%%%%%%%%%%%%%%%%%%%%%%%%%%%%%%%%%%%%%%%%%%%%%%%%%%%%%%%%%%%%%%%%%%
%%%%%%%%%%%%%% Select the Theme %%%%%%%%%%%%%%%%%%%%%%%%%%%%%%%%%%%
\usetheme{Dresden}     % OK
%\usetheme{Berlin}
%\usetheme{Bergen}      % NO
%\usetheme{Boadilla}    % NO
%\usetheme{Copenhagen}  % NO
%\usetheme{Hannover}    % NO
%\usetheme{Luebeck}     % NO
%\usetheme{Marburg}     % NO
%\usetheme{Pittsburgh}  % NO
%\usetheme{default}
%\usetheme{Singapore}   % OK
%\usetheme{boxes}
%\usecolortheme{structure}
%\usecolortheme{rose}
%\usecolortheme{beaver}


\definecolor{mymaroon}{cmyk}{0.0, 1.0, 1.0, 0.498}
\definecolor{myblue}{cmyk}{1.0, 1, 0, 0.5}
\definecolor{mygreen}{cmyk}{100, 0, 100, 50}
\setbeamercolor*{palette secondary}{use=structure,fg=white,bg=myblue}
\setbeamercolor*{palette tertiary}{use=structure,fg=white,bg=mymaroon}

%\usepackage{beamerthemesplit}              %
\beamertemplateballitem % fancy bullets and numbering

\setbeamertemplate{navigation symbols}{}   % suppress navigation symbols
\addtobeamertemplate{frametitle}{}{%
	\logo{../images/IIT_logo}
	\iffalse
	
	\begin{tikzpicture}[remember picture,overlay]
	\node[anchor=center, yshift=-13pt, xshift=-5pt] at (current page.north) 
	{\includegraphics[height=1.1cm]{../images/Argonne_cmyk_black-eps-converted-to}\hspace{10cm}};
	
	\node[anchor=north east, yshift=3pt, xshift=0pt] at (current page.north east) 
	{\includegraphics[height=0.7cm]{../images/IIT_Logo_blk}};
	\end{tikzpicture}
    
     \fi
}
% other possibilities to include LOGO. it puts it in RLC

%
%\pgfdeclareimage[width=1cm]{logo}{../images/IIT_Logo}
%\logo{\pgfuseimage{logo}}


% load additional packages

\usepackage{xcolor}
\usepackage{graphicx}
\usepackage{amsmath}
\usepackage{amssymb}
\usepackage{amsthm}
\usepackage{graphicx}
\usepackage{url}
\usepackage{color}
\usepackage{booktabs} % Allows the use of \toprule, \midrule and \bottomrule in tables
\usepackage{pifont}% http://ctan.org/pkg/pifont
\usepackage{epstopdf}
\usepackage[export]{adjustbox}
\usepackage{tikz}
\usetikzlibrary{shapes.misc}
\usetikzlibrary{shapes,arrows,decorations.markings,shadows,positioning}

% Your Abbreviations
\newcommand\bE{{\mathbb{E}}}
\newcommand\bR{{\mathbb{R}}}
\newcommand\bH{{\mathbf{H}}}
% End abbreviations

\newcommand\Wider[2][3em]{%
	\makebox[\linewidth][c]{%
		\begin{minipage}{\dimexpr\textwidth+#1\relax}
			\raggedright#2
		\end{minipage}%
	}%\textbf{}
}

%%%%%%%%%%%%%%%%%%%%% to edit the main text below
%NOTES ON SOME TECHNICS
%%%% Box %%%%%%%%%%%%%%%%%%%%%%%%%%%%%%%%%%%%%%%%%%%%%%%
%{\fbox{ \parbox[t]{10cm}{ SOME TEXT }}}

%%% include a picture. The file should be with extention EPS, e.g. FILENAME.EPS
%\begin{figure}[h]
%\centering
%\includegraphics[width=.7\linewidth]{FILENAME}
%\caption{{\footnotesize PUT_CAPTION }}
%\end{figure}

%\subtitle{}
%\institute[ANL/IIT]{Argonne National Laboratory\\Illinois Institute of Technology}

\title[USPAS 2018]{Accelerating Structures and Linear Machines}
\author[N.Neveu]{{\Large Nicole Neveu}}
\institute[ANL, IIT] % (optional, but mostly needed)
{   Illinois Institute of Technology \\
	Argonne National Laboratory \\
    \url{nneveu@anl.gov} 
}
% - Use the \inst command only if there are several affiliations.
% - Keep it simple, no one is interested in your street address.
\date{ June 8, 2018 \\ \vspace{0.5em}
\includegraphics[width=3.5cm,keepaspectratio]{../../logos/Argonne_cmyk_black}%
\hfill \hfill \hfill%
\includegraphics[width=4cm,keepaspectratio]{../../logos/IIT_Logo_blk-eps-converted-to}%
}

%\date[IIT, April 2009]{
%           Space Charge 2017 \\ Oc 18, 2009  }



\AtBeginSection[]
{
	\begin{frame}{Outline}
	\tableofcontents[currentsection]
\end{frame}
}

\begin{document}


\begin{frame}
  \titlepage
\end{frame}
\begin{frame}
	\frametitle{Outline}
	\tableofcontents
\end{frame}

%\begin{frame}{Outline of the talk}
%  \tableofcontents
%  % You might wish to add the option [pausesections]
%\end{frame}


% Structuring a talk is a difficult task and the following structure
% may not be suitable. Here are some rules that apply for this
% solution:

% - Exactly two or three sections (other than the summary).
% - At *most* three subsections per section.
% - Talk about 30s to 2min per frame. So there should be between about
%   15 and 30 frames, all told.

% - A conference audience is likely to know very little of what you
%   are going to talk about. So *simplify*!
% - In a 20min talk, getting the main ideas across is hard
%   enough. Leave out details, even if it means being less precise than
%   you think necessary.
% - If you omit details that are vital to the proof/implementation,
%   just say so once. Everybody will be happy with that.


\section{Linacs}
\subsection{Introduction}

\begin{frame}
	\frametitle{Linear Accelerators (Linacs)}
	\vspace{1em}
	
	\begin{minipage}{0.5\textwidth}
		\begin{itemize}
			\item What is a linac?
			\item Why do we need them?
			\item How do they work?
		\end{itemize}
	\end{minipage}
\begin{minipage}{0.45\textwidth}
	linac pic
\end{minipage}

\vspace{1em}
Another linac pic
	
	

\end{frame}

\begin{frame}
\frametitle{Linacs}
Copper or superconducting cavities stacked together.
Power from the wall (klyston) is pumped into the cavities.
If timed right, each cavity gives the beam some energy.




\end{frame}

\begin{frame}[containsverbatim]
\frametitle{Electron Linacs}
Basically at the speed of light already.
Can use resonators, copper, iris loaded.
\end{frame}


\begin{frame}
\frametitle{Proton, Ion}
Spoke cavities used for low Beta particles.
\end{frame}


%%%%%%%%%%%%%%%%%%%%%%%%%%%%%%%%%%%%%%%%%%%%%%%%%%%%%%%%%%%%%%%%%%%%%%%%%%%%%%%%

\section{Waveguides}
\subsection{Rectangular Waveguide}
\begin{frame}
\frametitle{Electrical Engineering(?) Jargon}
\begin{itemize}
	\item Radio Frequency (RF)
	\item Microwaves: 300 MHz to 300 GHz 
	\begin{itemize}
		\item L band: 1-3 GHz
		\item S band: 2-4 GHz
		\item C band: 4-8 GHz
		\item 
	\end{itemize}

Most copper linacs operate in L and S band. 
\end{itemize}
\end{frame}

\begin{frame}
\frametitle{Rectangular Waveguide}
\end{frame}
%%%%%%%%%%%%%%%%%%%%%%%%%%%%%%%%%%%%%%%%%%%%%%%%%%%%%%%%%%%%%%%%%%%%%%%%%%%%%%%%
\section{Accelerating Structures}
\subsection{Pill Box}
\begin{frame}
\frametitle{Accelerating Structures}

\end{frame}

\begin{frame}
	\frametitle{Pillbox Cavity}
	similarity to wg, fundamental mode (first zero of the Bessel), 
	circuit equation, 
\end{frame}

\begin{frame}
	\frame{Transit Factor}
\end{frame}

\subsection{Iris Loaded}
\begin{frame}
	\frametitle{}
	Give example of common copper structures. 
\end{frame}

%%%%%%%%%%%%%%%%%%%%%%%%%%%%%%%%%%%%%%%%%%%%%%%%%%%%%%%%%%%%%%%%%%%%%%%%%%%%%%%%



%%%%%%%%%%%%%%%%%%%%%%%%%%%%%%%%%%%%%%%%%%%%%%%%%%%%%%%%%%%%%%%%%%%%%%%%%%%%%%%%
\section{Power and Energy}
\subsection{Power Calcs}
\begin{frame}
	\frametitle{Gradient}
	The electric field strength (*on axis) in cavity due to an externally applied power is:
	\begin{equation}
		E^2_a = 
	\end{equation}

\end{frame}

\subsection{Energy Calculation}
\begin{frame}

\end{frame}


\begin{frame}
\frametitle{Calculate Energy Gain}
Given a gradient based on the information in the last slides...
Calculating the expected energy gain is proportional to the gradient and length of the accelerating structure. 

\begin{equation}
	\Delta E = qGL_a
\end{equation}

\begin{itemize}
	\item $\Delta E = $ change (gain) in energy
	\item $q =$ charge of particles
	\item $G = $ gradient of cavity in $\frac{MV}{m}$
	\item $L_a = $ length of accelerating structure 
\end{itemize}

\vspace{1em}
Note, deceleration can occur as well...
\end{frame}

\begin{frame}
	\frametitle{TBA Example}
	Two beam acceleration is a candidate technology for future machines.
	CLIC, FELs, etc...
	
	TBA scheme, how measurements are done. 
How energy measurements are done, spectrometer.
Calc energy from RF gradient
calc shunt impedance and Q
Do TBA on the board.	
	
\end{frame}



\section{Energy Measurement}
\begin{frame}
	\frametitle{Energy Measurements}
	Hardware needed to do measurement:

		\begin{itemize}
			\item Dipole
			\item Imaging screen(s)
			\item Actuators
			\item Current monitors
		\end{itemize}
\end{frame}


\begin{frame}
	\frametitle{Energy Measurements}
	\centering
	
	\begin{tikzpicture}[every node/.style={anchor=south west,inner sep=0pt},x=1mm, y=1mm,]   
	\node (fig1) at (0,0)
	{\includegraphics[width=0.8\textwidth]{awa_dipole}};
	\node[fill=white, inner sep=2pt] (txt2) at (40,35) {YAG};
	\node[fill=white, inner sep=2pt] (txt2) at (42,17) {Window};	
	\node[fill=white, inner sep=2pt] (txt2) at (13,41) {Dipole};
	\end{tikzpicture}
	
Picture courtesy AWA-ANL.	
\end{frame}

%%%%%%%%%%%%%%%%%%%%%%%%%%%%%%%%%%%%%%%%%%%%%%%%%%%%%%%%%%%%%%%%%%%%%%%%%%%%%%%%
\begin{frame}
	\frametitle{Summary}

\end{frame}
%%%%%%%%%%%%%%%%%%%%%%%%%%%%%%%%%%%%%%%%%%%%%%%%%%%%%%%%%%%%%%%%%%%%%%%%%%%%%%%%


\end{document}
















