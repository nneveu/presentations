%%%%%%%%%%%%%%%%%%%%%%%%%%%%%%%%%%%%%%%%%%%%%%%%%%%%%%%%%%%%%%%%%%%%%%
%%%%%%%%% Select one of the options, and comment the rest of them

%%%%%%%%%% Option 1:  to compile with pdflatex : parameter "t" - to align to the top
\documentclass[professionalfonts,t]{beamer}
%sans font?

%%%%%%%%%% Option 3: to create handout for print
%\documentclass[t,handout]{beamer}
%\usepackage{pgfpages}              % to put several slides on one page
%\pgfpagesuselayout{2 on 1}[a4paper, border shrink=5mm]             % 2 slides on 1 page
%\pgfpagesuselayout{4 on 1}[a4paper,landscape, border shrink=5mm]   % 4 slides on 1 page, and landscaped


%%%%%%%%%%%%%%%%%%%%%%%%%%%%%%%%%%%%%%%%%%%%%%%%%%%%%%%%%%%%%%%%%%%%
%%%%%%%%%%%%%% Select the Theme %%%%%%%%%%%%%%%%%%%%%%%%%%%%%%%%%%%
\usetheme{Dresden}     % OK
%\usetheme{Berlin}
%\usetheme{Bergen}      % NO
%\usetheme{Boadilla}    % NO
%\usetheme{Copenhagen}  % NO
%\usetheme{Hannover}    % NO
%\usetheme{Luebeck}     % NO
%\usetheme{Marburg}     % NO
%\usetheme{Pittsburgh}  % NO
%\usetheme{default}
%\usetheme{Singapore}   % OK
%\usetheme{boxes}
%\usecolortheme{structure}
%\usecolortheme{rose}
%\usecolortheme{beaver}


\definecolor{mymaroon}{cmyk}{0.0, 1.0, 1.0, 0.498}
\definecolor{myblue}{cmyk}{1.0, 1, 0, 0.5}
\definecolor{mygreen}{cmyk}{100, 0, 100, 50}
\setbeamercolor*{palette secondary}{use=structure,fg=white,bg=myblue}
\setbeamercolor*{palette tertiary}{use=structure,fg=white,bg=mymaroon}

%\usepackage{beamerthemesplit}              %
\beamertemplateballitem % fancy bullets and numbering

\setbeamertemplate{navigation symbols}{}   % suppress navigation symbols
\addtobeamertemplate{frametitle}{}{%
	\logo{../../images/IIT_logo}
	\iffalse
	
	\begin{tikzpicture}[remember picture,overlay]
	\node[anchor=center, yshift=-13pt, xshift=-5pt] at (current page.north) 
	{\includegraphics[height=1.1cm]{../images/Argonne_cmyk_black-eps-converted-to}\hspace{10cm}};
	
	\node[anchor=north east, yshift=3pt, xshift=0pt] at (current page.north east) 
	{\includegraphics[height=0.7cm]{../images/IIT_Logo_blk}};
	\end{tikzpicture}
    
     \fi
}
% other possibilities to include LOGO. it puts it in RLC

%
%\pgfdeclareimage[width=1cm]{logo}{../images/IIT_Logo}
%\logo{\pgfuseimage{logo}}


% load additional packages

\usepackage{xcolor}
\usepackage{graphicx}
\usepackage{amsmath}
\usepackage{amssymb}
\usepackage{amsthm}
\usepackage{graphicx}
\usepackage{url}
\usepackage{color}
\usepackage{booktabs} % Allows the use of \toprule, \midrule and \bottomrule in tables
\usepackage{pifont}% http://ctan.org/pkg/pifont
\usepackage{epstopdf}
\usepackage[export]{adjustbox}
\usepackage{tikz}
\usetikzlibrary{shapes.misc}
\usetikzlibrary{shapes,arrows,decorations.markings,shadows,positioning}

% Your Abbreviations
\newcommand\bE{{\mathbb{E}}}
\newcommand\bR{{\mathbb{R}}}
\newcommand\bH{{\mathbf{H}}}
\newcommand{\cmark}{\ding{51}}%
\newcommand{\xmark}{\ding{55}}%
% End abbreviations

\newcommand\Wider[2][3em]{%
	\makebox[\linewidth][c]{%
		\begin{minipage}{\dimexpr\textwidth+#1\relax}
			\raggedright#2
		\end{minipage}%
	}%\textbf{}
}

%%%%%%%%%%%%%%%%%%%%% to edit the main text below
%NOTES ON SOME TECHNICS
%%%% Box %%%%%%%%%%%%%%%%%%%%%%%%%%%%%%%%%%%%%%%%%%%%%%%
%{\fbox{ \parbox[t]{10cm}{ SOME TEXT }}}

%%% include a picture. The file should be with extention EPS, e.g. FILENAME.EPS
%\begin{figure}[h]
%\centering
%\includegraphics[width=.7\linewidth]{FILENAME}
%\caption{{\footnotesize PUT_CAPTION }}
%\end{figure}

%\subtitle{}
%\institute[ANL/IIT]{Argonne National Laboratory\\Illinois Institute of Technology}

\title[USPAS Fundamentals, June 4-15, 2018]{Guns and Codes}
\author[N.Neveu]{{\Large Nicole Neveu}}
\institute[ANL, IIT] % (optional, but mostly needed)
{   Illinois Institute of Technology \\
	Argonne National Laboratory \\
    \url{nneveu@anl.gov} 
}
% - Use the \inst command only if there are several affiliations.
% - Keep it simple, no one is interested in your street address.
\date{ \today \\
\includegraphics[width=3cm,keepaspectratio]{../../logos/Argonne_cmyk_black}%
\hfill \hfill \hfill%
\includegraphics[width=4cm,keepaspectratio]{../../logos/IIT_Logo_blk-eps-converted-to}%
}

%\date[IIT, April 2009]{
%           Space Charge 2017 \\ Oc 18, 2009  }

\AtBeginSection[]{
	\begin{frame}
	\vfill
	\centering
	
	\begin{minipage}{0.45\textwidth}
		%\begin{beamercolorbox}[sep=8pt,center,shadow=true,rounded=true]{title}
		\tableofcontents[currentsection]
		%\usebeamerfont{title}\insertsectionhead\par%
		%\end{beamercolorbox}	
	\end{minipage}\hfill
	\begin{minipage}{0.5\textwidth}
		\includegraphics[width=4cm]{\secimage}
		
		%Source: Fermilab Media
	\end{minipage}
	%\vfill 
\end{frame}
}

\newcommand{\secimage}{awa_gun}

\begin{document}


\begin{frame}
  \titlepage
\end{frame}


\begin{frame}
	\frametitle{Outline}
	\tableofcontents
\end{frame}

%\begin{frame}{Outline of the talk}
%  \tableofcontents
%  % You might wish to add the option [pausesections]
%\end{frame}


% Structuring a talk is a difficult task and the following structure
% may not be suitable. Here are some rules that apply for this
% solution:

% - Exactly two or three sections (other than the summary).
% - At *most* three subsections per section.
% - Talk about 30s to 2min per frame. So there should be between about
%   15 and 30 frames, all told.

% - A conference audience is likely to know very little of what you
%   are going to talk about. So *simplify*!
% - In a 20min talk, getting the main ideas across is hard
%   enough. Leave out details, even if it means being less precise than
%   you think necessary.
% - If you omit details that are vital to the proof/implementation,
%   just say so once. Everybody will be happy with that.


\section{Guns}

\begin{frame}
	\frametitle{What's a gun?}
	A few key parts are needed:
	\begin{itemize}
		\item Cathode (Source of particles)
		\item Focusing (usually solenoids)
		\item Vacuum (strict for electron machines w/ semiconductors)
		\item Accelerating field
	\end{itemize}

\vspace{1em}
\centering
\includegraphics[width=0.5\textwidth]{proton}

\tiny
\url{https://www.lhc-closer.es/taking_a_closer_look_at_lhc/0.proton_source}
\end{frame}

\subsection{Sources}
\begin{frame}
\frametitle{Sources of particles: Cathodes}
The particles have to come from somewhere:
\begin{itemize}
	\item metals and semiconductors
	\item gas
	\item filament
\end{itemize} 
\hfill \includegraphics[width=0.5\textwidth]{../../space_charge_2017/cathode1}%

\hfill Source: AWA-ANL

\end{frame}



\begin{frame}
	\frametitle{Thermionic Cathodes}
	\vspace{-0.75em}
	
	A filament is heated up until electrons are emitted.
	\includegraphics[width=0.5\textwidth]{therm}%
	\includegraphics[width=0.5\textwidth]{thermcut}
	
	\tiny
	Sources: \url{https://www.sciencedirect.com/science/article/pii/S0168900213014381\#f0005}
	\url{https://nau.edu/cefns/labs/electron-microprobe/glg-510-class-notes/instrumentation/}
	
	
\end{frame}


\begin{frame}
\frametitle{Photocathodes}
\vspace{-1em}
\begin{itemize}
	\item Needs a laser
	\item Photoelectric effect
\end{itemize}

\centering
\includegraphics[width=0.7\textwidth]{photocathode}

\tiny
\url{https://phys.org/news/2013-02-black-gold-enabling-bright-high.html}
\end{frame}

\subsection{Photoinjectors}
\begin{frame}
\frametitle{AWA-ANL Gun}
\centering
RF Photoinjector, 1.5 cell

\vspace{1em}
\includegraphics[width=\linewidth]{gun_cartoon}
\end{frame}

\begin{frame}
	\frametitle{AWA Gun fields}
	\centering
	\includegraphics[width=0.8\linewidth]{/home/nicole/Documents/thesis_code/regtest/benchmark/gun_EM_fields}
\end{frame}

\begin{frame}[containsverbatim]
\frametitle{Gun Types}
\vspace{1em}

\begin{minipage}{0.49\textwidth}
	DC Guns
	\begin{itemize}
		\item Bunch compression usually required
	\end{itemize}
\includegraphics[width=1\textwidth]{dc}
\end{minipage}
\begin{minipage}{0.49\textwidth}
	RF Guns
	\begin{itemize}
		\item Energy chirp, but bunched
	\end{itemize}
\includegraphics[width=\textwidth]{cw}
\includegraphics[width=\textwidth]{pulsed}

\end{minipage}

\vspace{2em}
\tiny
\url{https://www.classe.cornell.edu/~hoff/LECTURES/08S_688/08S_688_080225.pdf}
\end{frame}

%%%%%%%%%%%%%%%%%%%%%%%%%%%%%%%%%%%%%%%%%%%%%%%%%%%%%%%%%%%%%%%%%%%%%%%%%%%%%%%%

\section{Simulations}
\begin{frame}
\frametitle{Simulations}
Simulations give help with design and prediction of accelerators.
Two main types of simulations:

\vspace{1em}
\begin{itemize}
	\item PIC: Particle in Cell
	\begin{itemize}
		\item Model particles individually 
	\end{itemize}
	\item Envelope
	\begin{itemize}
		\item Use matrices to look at envelope behavior
	\end{itemize}
\end{itemize}

\vspace{1em}
Historically, envelope codes were first. 
Now computing resources are good enough to use PIC in more situations.
\end{frame}

\subsection{Codes}
\begin{frame}
	\frametitle{Some Envelope Codes}
	These codes do matrix multiplication, and use input beam parameters to find Twiss parameters around a beam line.
	
	\vspace{1em}
	\begin{itemize}
		\item MADX - \url{http://mad.web.cern.ch/mad/}
		\item Elegant (also a PIC code) - \url{https://beta.sirepo.com/elegant\#/simulations}
		\item Accelerator Toolbox (AT) - \url{https://www.slac.stanford.edu/grp/ssrl/spear/at/}
		\item TRACE - \url{http://laacg.lanl.gov/laacg/services/download_trace.phtml}
	\end{itemize}
\end{frame}


\begin{frame}
	\frametitle{PIC Codes}
	PIC = Particle in Cell 
	\vspace{1em}
	
	Distribute particles on a grid and solve equations numerically.
	\begin{itemize}
		\item Particle-Particle Forces (space charge)
		\item E\&M Forces
		\item Apertures 
	\end{itemize}
	
	\vspace{1em}
	Draw some things on the board....
	
	Meshes, particle push, etc.
\end{frame}

\begin{frame}
	\frametitle{Some PIC codes for beam line modeling...}
	\vspace{-1em}
	\small
	\begin{itemize}
		\item Parmela - \url{http://laacg.lanl.gov/laacg/services/serv_codes.phtml} 
		\item ASTRA - \url{http://www.desy.de/~mpyflo/}
		\item GPT - \url{http://www.pulsar.nl/gpt/}
		\item OPAL - \url{https://gitlab.psi.ch/OPAL/src/wikis/home}
		\item Synergia -  \url{https://web.fnal.gov/sites/Synergia/SitePages/Synergia\%20Home.aspx}
		\item Bmad - \url{https://www.classe.cornell.edu/bmad/}
		\item G4beamline - \url{http://www.muonsinternal.com/muons3/G4beamline}
		\item WARP - \url{http://blast.lbl.gov/blast-codes-warp/}
		\item IMPACT - \url{http://blast.lbl.gov/blast-codes-impact/}
	\end{itemize}
\end{frame}

\begin{frame}
\frametitle{Some Pic Codes cont..}
\begin{itemize}
	\item Elegant (also a PIC code) - \url{https://beta.sirepo.com/elegant\#/simulations}
\end{itemize}

\centering
\includegraphics[width=0.5\textwidth]{elegant}%
\includegraphics[width=0.5\textwidth]{xyelegant}
\end{frame}


\begin{frame}
\frametitle{PIC Code Comparison}
\centering
\vspace{-0.2cm}
\begin{table}
	\begin{minipage}{0.8\textwidth}
		\begin{center}	
			\begin{tabular}{l c c c}
				\toprule
				\textbf{Feature} & \textbf{ASTRA} & \textbf{GPT} & \textbf{OPAL}\\
				\midrule
				Windows     		& \cmark & \cmark & \alert \xmark \\ 
				Mac         		& \cmark & \cmark & \cmark \\
				Linux       		& \cmark & \cmark & \cmark \\
				Open Source 		& \alert \xmark & \alert \xmark & \color{black!30!green}\cmark \\
				Parallel    		& \alert \xmark * & \alert \xmark * & \color{black!30!green}\cmark \\
				Autophase   		& \cmark & \xmark & \cmark \\
				Adaptive Time Step 	& \xmark & \cmark & \xmark \\
				3D Space Charge 	& \cmark & \cmark & \cmark \\
				Wakefields  		& \cmark & \xmark * & \color{black!30!green}\cmark \\
				CSR         		& \alert \xmark & \xmark * & \color{black!30!green}\cmark \\
				\bottomrule
			\end{tabular}
			%\caption{Table caption}
		\end{center}
	\end{minipage}
\end{table}
* features available on request or in specific versions.
\end{frame}




\subsection{PIC Results}
\begin{frame}
\frametitle{PIC results: Statistics, beam parameters w.r.t. bunch.}
\vspace{-0.5em}
\centering

	\includegraphics[width=0.75\linewidth]{/home/nicole/Documents/thesis_code/regtest/benchmark/benchmark_gun}
%Phase space, stats, trajectory
\end{frame}





\section{Big Computers}
\begin{frame}
	Why PIC codes need supercomputers.
\end{frame}

\begin{frame}
Computing resources
\begin{itemize}
	\item LCRC-ANL: Bebop
	\item ALCF-ANl: Theta, Mira
	\item NERSC-LBNL: Edison, Cori
	\item ORNL: Titan, Summit
	\item FNL: Local resources
\end{itemize}
\end{frame}

\begin{frame}
	Computer Jargon / typical hardware / software.
	\begin{itemize}
		\item Linux
		\item KNL's
		\item slurm / batch jobs
	\end{itemize}
\end{frame}



\end{document}
















