\documentclass[letterpaper,  %a4paper
               %boxit,
               %titlepage,   % separate title page
               %refpage      % separate references
              ]{jacow-2_3}   %jacow}
%
% CHANGE SEQUENCE OF GRAPHICS EXTENSION TO BE EMBEDDED
% ----------------------------------------------------
% test for XeTeX where the sequence is by default eps-> pdf, jpg, png, pdf, ...
%    and the JACoW template provides JACpic2v3.eps and JACpic2v3.jpg which
%    might generates errors, therefore PNG and JPG first
%
\makeatletter%
	\ifboolexpr{bool{xetex}}
	 {\renewcommand{\Gin@extensions}{.pdf,%
	                    .png,.jpg,.bmp,.pict,.tif,.psd,.mac,.sga,.tga,.gif,%
	                    .eps,.ps,%
	                    }}{}
\makeatother

% CHECK FOR XeTeX/LuaTeX BEFORE DEFINING AN INPUT ENCODING
% --------------------------------------------------------
%   utf8  is default for XeTeX/LuaTeX 
%   utf8  in LaTeX only realizes a small portion of codes
%
\ifboolexpr{bool{xetex} or bool{luatex}} % test for XeTeX/LuaTeX
 {}                                      % input encoding is utf8 by default
 {\usepackage[utf8]{inputenc}}           % switch to utf8

\usepackage[USenglish]{babel}			 

\usepackage[final]{pdfpages}
\usepackage{multirow}
\usepackage{ragged2e}
\usepackage{tikz}
\usetikzlibrary{shapes,arrows,snakes,backgrounds}
\usetikzlibrary{mindmap,trees}
\usetikzlibrary{decorations.pathreplacing}
\usetikzlibrary{plotmarks}
\newcommand{\lsnote}[1]{\textsf{{\color{violet}{ LS note:}   #1 }}}
%
% if BibLaTeX is used
%
\ifboolexpr{bool{jacowbiblatex}}%
 {%
  \addbibresource{jacow-test.bib}
  \addbibresource{biblatex-examples.bib}
 }{}
\listfiles

%
% command for typesetting a \section like word
%
\newcommand\SEC[1]{\textbf{\uppercase{#1}}}

%%
%%   Lengths for the spaces in the title
%%   \setlength\titleblockstartskip{..}  %before title, default 3pt
%%   \setlength\titleblockmiddleskip{..} %between title + author, default 1em
%%   \setlength\titleblockendskip{..}    %afterauthor, default 1em

%\copyrightspace %default 1cm. arbitrary size with e.g. \copyrightspace[2cm]

% testing to fill the copyright space
%\usepackage{eso-pic}
%\AddToShipoutPictureFG*{\AtTextLowerLeft{\textcolor{red}{COPYRIGHTSPACE}}}

\begin{document}

\title{Photoinjector Optimization Studies at the AWA}

\author{N. Neveu\thanks{nneveu@anl.gov}\textsuperscript{1}, 
	    L. Spentzouris, Illinois Institute of Technology, Chicago, USA \\
	    J. Larson, J. G. Power, \textsuperscript{1}Argonne National Laboratory}
\maketitle

%
\begin{abstract}
With a variable charge range of 0.1 nC - 100 nC, 
the Argonne Wakefield Accelerator facility (AWA) 
has a unique and dynamic set of operating parameters. 
Adjustment of the optics and occasionally the rf phases is 
required each time the bunch charge is changed. 
Presently, these adjustments are done by the operator during each experiment. 
This is time consuming and inefficient, more so at high charge and for complex experimental set ups.
In an attempt to reduce the amount of time spent adjusting parameters by hand, 
several optimization methods in simulation are being explored. 
This includes using the well-known Genetic Algorithm (NSGA-II),
incorporated into OPAL-T. 
We have also investigated a model-based method and novel
structure based algorithms developed at ANL. 
These optimization methods will be implemented to improve operations at the AWA. 
Simulation results will be compared to measured beam parameters at the AWA, 
and one or more optimization methods will be selected for use in guiding operations in the future.
\end{abstract}


\section{AWA Facility}
The AWA Facility houses two rf photoinjectors, both 
operating at \SI{1.3}{GHz}. 
A large range of experiments are performed at various 
charge levels, and several methods for simulating 
the beam lines have been used.
Recent experiments include emittance exchange \cite{eex}, 
structure tests \cite{pets}, thermal emittance measurements \cite{therm}, 
and Two Beam Acceleration (TBA) \cite{tba}. 
Simulation codes used for these and other experiments at
the AWA include: PARMELA \cite{parmela}, GPT \cite{gpt}, 
ASTRA \cite{astra}, and more recently OPAL \cite{opal}. 
The latter is the code used for all simulations in this 
study. 
We also take advantage of the computing resources provided
by the Laboratory Computing Resource Center (LCRC), 
at Argonne National Laboratory (ANL). Access to the 
Blues, and recently installed Bebop clusters has significantly
increased simulation productivity by providing the capability to run 
all simulations in parallel, and large optimization cases
on many cores.

\section{Model Based Method}
Last year, we began with the optimization of the front end of 
the high charge photoinjector at the AWA. A charge
of \SI{40}{nC} was chosen to target the needs 
for upcoming TBA experiments at AWA.
The simulation model included the gun, two solenoids, 
and six accelerating cavities, as shown in Fig.~\ref{beamline}. 
The quadrupoles and bending elements were not included in this first effort, they will be included later.
We choose ten variables for the optimization parameters: 
laser radius, laser full width half maximum, solenoid strength ($S_2$), 
gun cavity phase, and the six accelerating cavity phases ($\phi_{L_1}-\phi_{L_6}$).
\begin{figure*}
	\centering
	\begin{tikzpicture}[scale=0.7, text=black]
	\input{tikz-opt.tex}
	\end{tikzpicture}	
	\caption{Partial beam line layout at the AWA.}
	\label{beamline}
\end{figure*}

The objective was to get an optimal combination of small emittance, and short bunch length at the 
entrance of the quads. 
We choose to use a local optimization method that is 
freely available in the NLOPT package \cite{nlopt}. 
Bounded Optimization By Quadratic Approximation (BOBYA), 
is a model based and derivative-free method 
that builds a quadratic using the results of simulation evaluations. 
The next point to evaluate is chosen by minimizing the quadratic.
The initial probe of the parameter space was done with 1,000 random simulations, 
inside the bounds of Table~\ref{bounds}.
Simulations with the best emittance and bunch length were chosen 
as starting points for ten BOBYQA runs.
These local optimization runs were carried out until they converged.
The results from all ten runs were used to form a Pareto front
comparing transverse emittance and bunch length. 
Details of this work were presented at IPAC'17, and can be found here \cite{denmark}. 

The initial results were promising, and attempts to measure
several points on the Pareto front took place.
Experimentally, some of the parameter values were found to 
be outside the normal operating conditions at the AWA.
Using this experience, a second round of optimization 
was done using the same procedure and method as 
outlined in \cite{denmark}. The boundaries for the second
round of optimization work can be found in Table~\ref{bounds}.
Note that $\phi_L=[\phi_{L_1},\ldots,\phi_{L_6}]$.

\begin{table}[hbt] 
	\centering
	\caption{Parameter Bounds for Linac Optimization}
	\begin{tabular}{ l *{3}{c}}
		\toprule
		\textbf{Variable} & \textbf{Range} & \textbf{Unit} \\
		\midrule
		Solenoid Strength & $ 50 \le S_2 \le 440$  & amps \\
		Phase of Gun & $-45 \le \phi_g \le 45$  & degrees \\
		Laser Radius  & $3.5 \le R \le 9$  & mm \\
		Laser FWHM & $1.5 \le T \le $10  & ps \\
		Cavity Phase & $-40 \le \phi_L \le 40$  & degrees \\
		\bottomrule	
	\end{tabular}	
	\label{bounds}
\end{table}

The parameter values of the new Pareto front in Fig.~\ref{paretob} were analyzed, and 
several key points were learned from this work. 
First, it is clear that varying the laser radius is unnecessary for 
high charge simulations. All points on the Pareto front
had a laser radius of \SI{9}{mm}, the maximum value for this parameter.
Due to the strong space charge forces at the AWA, 
it is beneficial to let the 
laser radius be as large as possible.  This mitigates space charge
and improves the emittance. Ongoing and future high charge optimization 
work will exclude the laser radius as an optimization parameter.
\begin{figure}
	\includegraphics[width=0.5\textwidth]{pareto_emittance_vs_zrms}
	\caption{Pareto front for adjusted variable bounds at \SI{40}{nC}. }
	\label{paretob}
\end{figure}

Second, it was clear that the phase boundaries were also too large.
None of the gun phases were positive. An upper bound of $\phi = 0^{\circ}$
will be used in the future. Some of linac phases were positive, 
but very few. Five out of ten local optimization runs had no positive linac 
phases. Four out of ten local optimization runs had only one positive 
linac \lsnote{phase} ($\phi_{L_6}$), and only two out of the ten runs had two or three
positive linac phases ($\phi_{L_4}-\phi_{L_6}$). Therefore, the upper bounds for 
$\phi_{L_1}-\phi_{L_3}$ can also be set to $0^{\circ}$. 

The last lesson learned from the initial model based work deals with the bunch length
and energy spread. As stated earlier, every gun phase was negative as were 
$\phi_{L_1}$ and $\phi_{L_2}$. These negative phases are reducing the 
energy spread in the bunch. This result is expected based on work by others \cite{lclsii}, 
and is called velocity bunching. Validation of a known behavior adds 
credibility to the the model based work.

\section{Genetic Algorithms}
Genetic Algorithms (GAs) have been used successfully for over two decades
in the accelerator physics community.
They are used to tackle large multiobjective optimization problems on wide variety of machines.  
A nice review of work done in the field using GAs can be found here \cite{hofler}. 
In general, this class of algorithms aims to mimic the natural selection 
process often seen in biology. In this way, there are several
steps that most GAs follow: 
\begin{enumerate}
	\itemsep0em
	\item Start with a finite population (simulation evaluations). This is the first generation.
	\item Decided which individuals (simulations) in the population 
	are the best based on a selection criteria.
	\item Mix the best individuals to generate new individuals (new generation).
	\item Repeat many times.
\end{enumerate}

There has been much research on how to implement these steps, 
and what types of selection and mutation criteria should be used.
A widely used implementation is the \mbox{NSGA-II}~\cite{nsgaii} method.
This is also the algorithm built into the simulation code 
OPAL, which was the code of choice for the initial 
optimization work. It was decided to take advantage of the 
built in optimizer for verification of the model based 
method and future work.

To verify the model based method, a low fidelity model 
of the photoinjector was used to do a quick comparison.
The number of particles and grid size was reduced to 
shorten the simulation time to two minutes. 
GAs require several thousand simulations to converge. 
Reduction in the fidelity 
allowed us to complete a comparison using a reasonable 
amount of computing resources and within a weeks time.
The optimization variables (Table~\ref{bounds}) 
and objectives are these same as those given in the previous section, 
with one important difference. The boundaries were 
reduced based on the information learned in the first
round of optimization.

In total, 2,393 simulations (from the 1,000-point sample plus 
1,393 points of  BOBYQA optimization runs) were used to
generate the model based approximate Pareto front. 
This approach required approximately 80 core-hours of computation. 
We compared this with comparable fronts generated by the GA.
With 128 simulations in each generation, the GA required 3,200 
simulations (107 core-hours) to reach the 25th generation, and 64,000 simulations
(2,133 coure-hours) to reach the 500th generation. 
Comparison of the three Pareto fronts is shown in Fig~\ref{compare}.
As expected by the "no free lunch" theorem, both methods give 
acceptable results, with only the total simulation time being 
the difference. 
\begin{figure}
	\includegraphics[width=0.5\textwidth]{model_vs_ga}
	\caption{Comparison of model based optimization and the GA implemented in OPAL at \SI{40}{nC}. }
	\label{compare}
\end{figure}

The next round of optimization will include the four quadrupoles, 
kicker, and septum shown in Fig.~\ref{beamline}. This work 
will support upcoming TBA experiments at the AWA.

\section{Future Work: Novel Methods}
In the following months, we plan to continue our work 
by investigating novel optimization methods and comparing their efficiency and precision. 
We will take advantage of the 
open source Python library libensemble:
\begin{center}
	\url{https://github.com/Libensemble/libensemble}
\end{center} 
which is being developed at ANL.
This framework allows for massively parallel ensemble simulations in 
combination with a mechanism for easily interchanging optimization methods. 
We will work to deploy Asynchronously Parallel Optimization Solver for finding Multiple Minima (APOSMM).
This multistart algorithm considers all results from previously evaluated simulations when determining where to start or continue a local optimization run \cite{jeff}. This allows for an efficient 
search for a global minimum. This method has already been tested
(with success) on problems where derivatives are unavailable, 
as is the case in many non-linear accelerator physics problems. 


\section{Conclusion}
Photoinjector optimization projects are ongoing at the AWA.
Two optimization methods have been used with success:
BOBYQA and NSGA-II. In the future, novel optimization 
methods developed at ANL will be tested. The first of 
which will be APOSMM. All optimization efforts aim to improve operating conditions at the AWA and 
aid in the completion of the beamline designs needed for upcoming experiments.

\section{acknowledgments}
We gratefully acknowledge the computing resources
provided on Bebop, a high-performance computing cluster
operated by the LCRC at ANL.
This material is based upon work supported by the 
U.S. Department of Energy, Office of Science, under 
contract number DE-AC02-06CH11357 and grant number DE-SC0015479. 
Travel to IPAC'18 supported by the United States National Science Foundation, 
the Division of Physics of Beams of the American Physical Society, and TRIUMF.

\begin{thebibliography}{99}
\bibitem{eex}
G.~Ha \emph{et al.}, “Demonstration of Current Profile 
Shaping using Double Dog-Leg Emittance Exchange Beam 
Line at Argonne Wakefield Accelerator”
in \textit{Proc. IPAC’16}, 
Busan, South Korea, May 2016, 
paper TUOBB01.

\bibitem{pets}
J.~Shao \emph{et al.}, “Recent Progress towards Dielectric Short Pulse Two-Beam Acceleration”
in \textit{Proc. IPAC’18}, 
Vancouver, Canada, May 2018, 
paper TUYGBE3.

\bibitem{therm}
L.~Zheng \emph{et al.}, “Measurements of Thermal Emittance 
for Cesium Telluride Photocathodes in an L-Band RF Gun”
in \textit{Proc. IPAC’17}, 
Copenhagen, Denmark, May 2017, 
paper TUPAB074.

\bibitem{tba}
J.~Shao \emph{et al.}, “Recent Two-Beam 
Acceleration Activities at Argonne Wakefield Accelerator Facility”
in \textit{Proc. IPAC’17}, 
Copenhagen, Denmark, May 2017, 
paper WEPVA022.

\bibitem{parmela}
L. M. Young, “PARMELA,”
Los Alamos National Laboratory, 
Los Alamos, NM, USA,
Rep. LA-UR-96-1835 (Revised April 22, 2003).

\bibitem{astra}
ASTRA Manual, April 2014, 
\url{http://www.desy.de/~mpyflo/Astra_manual/Astra-Manual_V3.2.pdf}

\bibitem{gpt}
GPT,  \url{http://www.pulsar.nl/gpt}

\bibitem{opal}
A.~Adelmann \emph{et al.},
“The OPAL (Object Oriented Parallel Accelerator Library) framework,”
PSI, Zurich, Switzerland,
Rep. PSI-PR-08-02, 2008-2017.

\bibitem{nlopt}
M. Powell, “The BOBYQA algorithm for bound constrained
optimization without derivatives,” University of Cambridge,
U.K., Rep. NA2009/06, Oct. 2009.

\bibitem{denmark}
N.Neveu, J.Larson, J. G. Power, L. Spentzouris, 
“Photoinjector optimization using a derivative-free, model-based trust-region algorithm for the Argonne Wakefield Accelerator”
in \textit{Proc. IPAC’17}, 
Copenhagen, Denmark, May 2017, 
paper WEPVA022.

\bibitem{lclsii}
C.~Mitchell \emph{et al.}, “RF Injector Beam Dynamics Optimization 
and Injected Beam Energy Constraints for LCLS-II”
in \textit{Proc. IPAC’16}, 
Busan, South Korea, May 2016, 
paper TUPOR019.

\bibitem{hofler}
A. S. Hofler \emph{et al.},
"Innovative applications of genetic algorithms to 
problems in accelerator physics",
\emph{Phys. Rev. ST Accel. Beams}, vol. 16,
p. 010101, 2013.


\bibitem{nsgaii}
K. Deb, A. Pratap, S. Agarwal and T. Meyarivan, 
"A Fast and Elitist Multiobjective Genetic Algorithm: NSGA-II," 
\emph{IEEE Trans. on Evolutionary Comp.} , 
vol. 6, no. 2, pp. 182-197, Apr 2002.

\bibitem{jeff}
J. Larson, and S.M. Wild, ""
\emph{Math. Prog. Comp.}, vol. 10, 
p. 1-30, Feb. 2018. 

\end{thebibliography}



\end{document}
	
