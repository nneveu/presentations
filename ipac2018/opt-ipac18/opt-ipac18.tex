\documentclass[letterpaper,  %a4paper
               %boxit,
               %titlepage,   % separate title page
               %refpage      % separate references
              ]{jacow-2_3}   %jacow}
%
% CHANGE SEQUENCE OF GRAPHICS EXTENSION TO BE EMBEDDED
% ----------------------------------------------------
% test for XeTeX where the sequence is by default eps-> pdf, jpg, png, pdf, ...
%    and the JACoW template provides JACpic2v3.eps and JACpic2v3.jpg which
%    might generates errors, therefore PNG and JPG first
%
\makeatletter%
	\ifboolexpr{bool{xetex}}
	 {\renewcommand{\Gin@extensions}{.pdf,%
	                    .png,.jpg,.bmp,.pict,.tif,.psd,.mac,.sga,.tga,.gif,%
	                    .eps,.ps,%
	                    }}{}
\makeatother

% CHECK FOR XeTeX/LuaTeX BEFORE DEFINING AN INPUT ENCODING
% --------------------------------------------------------
%   utf8  is default for XeTeX/LuaTeX 
%   utf8  in LaTeX only realizes a small portion of codes
%
\ifboolexpr{bool{xetex} or bool{luatex}} % test for XeTeX/LuaTeX
 {}                                      % input encoding is utf8 by default
 {\usepackage[utf8]{inputenc}}           % switch to utf8

\usepackage[USenglish]{babel}			 

\usepackage[final]{pdfpages}
\usepackage{multirow}
\usepackage{ragged2e}
\usepackage{tikz}
\usetikzlibrary{shapes,arrows,snakes,backgrounds}
\usetikzlibrary{mindmap,trees}
\usetikzlibrary{decorations.pathreplacing}
\usetikzlibrary{plotmarks}
%
% if BibLaTeX is used
%
\ifboolexpr{bool{jacowbiblatex}}%
 {%
  \addbibresource{jacow-test.bib}
  \addbibresource{biblatex-examples.bib}
 }{}
\listfiles

%
% command for typesetting a \section like word
%
\newcommand\SEC[1]{\textbf{\uppercase{#1}}}

%%
%%   Lengths for the spaces in the title
%%   \setlength\titleblockstartskip{..}  %before title, default 3pt
%%   \setlength\titleblockmiddleskip{..} %between title + author, default 1em
%%   \setlength\titleblockendskip{..}    %afterauthor, default 1em

%\copyrightspace %default 1cm. arbitrary size with e.g. \copyrightspace[2cm]

% testing to fill the copyright space
%\usepackage{eso-pic}
%\AddToShipoutPictureFG*{\AtTextLowerLeft{\textcolor{red}{COPYRIGHTSPACE}}}

\begin{document}

\title{Photoinjector Optimization Studies at the AWA}

\author{N. Neveu\thanks{nneveu@anl.gov}\textsuperscript{1}, 
	    L. Spentzouris, Illinois Institute of Technology, Chicago, USA \\
	    J. Larson, J. G. Power, \textsuperscript{1}Argonne National Laboratory}
\maketitle

%
\begin{abstract}
With a variable charge range of 0.1 nC - 100 nC, 
the Argonne Wakefield Accelerator facility (AWA) 
has a unique and dynamic set of operating parameters. 
Adjustment of the optics and occasionally the rf phases is 
required each time the bunch charge is changed. 
Presently, these adjustments are done by the operator during each experiment. 
This is time consuming and inefficient, more so at high charge and for complex experimental set ups.
In an attempt to reduce the amount of time spent adjusting parameters by hand, 
several optimization methods in simulation are being explored. 
This includes using the well-known Genetic Algorithm (NSGA-II),
incorporated into OPAL-T. 
We have also investigated a model-based method and novel
structure based algorithms developed at ANL. 
Ongoing efforts include using these optimization methods to improve operations at the AWA. 
Simulation results will be compared to measured beam parameters at the AWA, 
and one optimization method will be selected for use in guiding operations going forward.
\end{abstract}


\section{AWA Facility}
The AWA Facility houses two rf photoinjectors, both 
operating at \SI{1.3}{GHz}. The photoinjector used for 
these studies consists of a gun and solenoids followed
by six accelerating cavities, as shown in Fig.~\ref{beamline}. 
This beam line is capable of low (\SI{0.1}{nC}) and 
high charge (\SI{100}{nC}) operation. The bunch charge is 
routinely adjusted for depending on the requirements 
of the experiments downstream of the photoinjector.
Typical operating charges are 1, 4, 10, and \SI{40}{nC}. 
While these are the most
common operating modes, other charges have been requested 
and provided depending on the experiment.
Recent experiments include emittance exchange \cite{eex}, 
structure tests \cite{pets}, thermal emittance measurements \cite{therm}, 
and two beam acceleration \cite{tba}. 

\section{Simulations}
Simulations of the AWA beam line shown in Fig~\ref{beamline}
were performed in the PIC codes OPAL~\cite{opal}.
The gun, accelerating cavities, and solenoids were modeled with 2D
Poisson/Superfish~\cite{fish} files. All field maps were in the T7 format.
Input parameters for the simulations are shown in Table~\ref{simparam}.
Note that on crest refers to the phase of max energy gain.
In the case of the gun, a -~$20^{\circ}$ phase is measured 
w.r.t the phase of max energy gain. 
In other words, we ran $-20^{\circ}$ off crest.

\section{Model based method: BOBYQA}
Last year...

\section{Current Work: GA}
Now....

\section{Future Work: Novel Methods}
Next year....



\begin{figure*}[!tbh]
	\centering
	\begin{tikzpicture}[scale=0.7, text=black]
	\input{tikz-opt.tex}
	\end{tikzpicture}	
	\caption{Beam line layout at the AWA.}
	\label{beamline}
\end{figure*}


\begin{table}[hbt]
	%   \vspace*{-.5\baselineskip}
	\centering
	\caption{Simulation Parameters}
	\begin{tabular}{lcc}
		\toprule
		\textbf{Parameter} & \textbf{Low Charge}  & \textbf{High Charge} \\
		\midrule
		Charge       & \SI{1}{nC}        & \SI{40}{nC}    \\ %[3pt]
		Gun Gradient & \SI{65}{MV/m}     & \SI{65}{MV/m}  \\ %[3pt]
		Gun Phase    & \SI{-20}{}$^{\circ}$ & \SI{-20}{}$^{\circ}$ \\		 
		$S_1$        & \SI{500}{A}		 & \SI{500}{A}	  \\
		$S_2$		 & \SI{}{A}   	 & \SI{185}{A}		 \\
		Linac Phases & On crest          & On crest       \\
		Laser FWHM   & \SI{1.5}{ps}      & \SI{1.5}{ps}   \\ %[3pt]
		Laser Radius & \SI{2}{mm}        & \SI{9}{mm}     \\
		\bottomrule
	\end{tabular}
	\label{simparam}
	%   \vspace*{-\baselineskip}
\end{table}


\begin{figure}[!htb]
	%   \vspace*{-.5\baselineskip}
	\centering
	%\includegraphics*[width=174pt]{JACpic_mc}
	\caption{Layout of papers.}
	\label{l2ea4-f1}
	%   \vspace*{-\baselineskip}
\end{figure}

\section{Conclusion}
Any conclusions should be in a separate section directly preceding
the \SEC{Acknowledgment}, \SEC{Appendix}, or \SEC{References} sections, in that
order.

\section{acknowledgments}
We gratefully acknowledge the computing resources
provided on Bebop, a high-performance computing cluster
operated by the LCRC at Argonne National Laboratory.
This material is based upon work supported by the 
U.S. Department of Energy, Office of Science, under 
contract number DE-AC02-06CH11357 and grant number DE-SC0015479. 
Travel to IPAC'18 supported by the Division of Physics 
of the United States National Science Foundation 
Accelerator Science Program and the Division of 
Beam Physics of the American Physical Society.


\begin{thebibliography}{9}
\bibitem{eex}
G.~Ha \emph{et al.}, “Demonstration of Current Profile 
Shaping using Double Dog-Leg Emittance Exchange Beam 
Line at Argonne Wakefield Accelerator”
in \textit{Proc. IPAC’16}, 
Busan, South Korea, May 2016, 
paper TUOBB01.\\

\bibitem{pets}
J.~Shao \emph{et al.}, “PETS....”
in \textit{Proc. IPAC’18}, 
Vancouver, Canada, May 2018, 
paper xxx.\\

\bibitem{therm}
L.~Zheng \emph{et al.}, “Measurements of Thermal Emittance 
for Cesium Telluride Photocathodes in an L-Band RF Gun”
in \textit{Proc. IPAC’17}, 
Copenhagen, Denmark, May 2017, 
paper TUPAB074.\\

\bibitem{tba}
J.~Shao \emph{et al.}, “Recent Two-Beam 
Acceleration Activities at Argonne Wakefield Accelerator Facility”
in \textit{Proc. IPAC’17}, 
Copenhagen, Denmark, May 2017, 
paper WEPVA022.\\

\bibitem{ctr}
A.~Alpha \emph{et al.}, 
“A fascinating paper about CTR,”
in \emph{Proc. FEL’13}, 
New York, NY, USA, Aug. 2013, 
paper WEP033, pp. 27--29.\\

\bibitem{impact}
J.~Qiang \emph{et al.},
“...,”
, California, USA,
Rep. xxxx, 20xx-20xx.

\bibitem{opal}
A.~Adelmann \emph{et al.},
“The OPAL (Object Oriented Parallel Accelerator Library) framework,”
PSI, Zurich, Switzerland,
Rep. PSI-PR-08-02, 2008-2017.

\bibitem{fish}
\emph{Reference Manual for the POISSON/SUPERFISH Group of 
	Codes},  Los Alamos Accelerator Code Group,  
 Los Alamos, NM, USA, 
 Rep. LA-UR-87-126, Jan. 1987.
\end{thebibliography}



\end{document}
	
