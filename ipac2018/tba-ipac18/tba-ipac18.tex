\documentclass[letterpaper,  %a4paper
               %boxit,
               %titlepage,   % separate title page
               %refpage      % separate references
              ]{jacow-2_3}   %jacow}
%
% CHANGE SEQUENCE OF GRAPHICS EXTENSION TO BE EMBEDDED
% ----------------------------------------------------
% test for XeTeX where the sequence is by default eps-> pdf, jpg, png, pdf, ...
%    and the JACoW template provides JACpic2v3.eps and JACpic2v3.jpg which
%    might generates errors, therefore PNG and JPG first
%
\makeatletter%
	\ifboolexpr{bool{xetex}}
	 {\renewcommand{\Gin@extensions}{.pdf,%
	                    .png,.jpg,.bmp,.pict,.tif,.psd,.mac,.sga,.tga,.gif,%
	                    .eps,.ps,%
	                    }}{}
\makeatother

% CHECK FOR XeTeX/LuaTeX BEFORE DEFINING AN INPUT ENCODING
% --------------------------------------------------------
%   utf8  is default for XeTeX/LuaTeX 
%   utf8  in LaTeX only realizes a small portion of codes
%
\ifboolexpr{bool{xetex} or bool{luatex}} % test for XeTeX/LuaTeX
 {}                                      % input encoding is utf8 by default
 {\usepackage[utf8]{inputenc}}           % switch to utf8

\usepackage[USenglish]{babel}			 

\usepackage[final]{pdfpages}
\usepackage{multirow}
\usepackage{ragged2e}
\usepackage{tikz}
\usetikzlibrary{shapes,arrows,snakes,backgrounds}
\usetikzlibrary{mindmap,trees}
\usetikzlibrary{decorations.pathreplacing}
\usetikzlibrary{plotmarks}
%
% if BibLaTeX is used
%
\ifboolexpr{bool{jacowbiblatex}}%
 {%
  \addbibresource{jacow-test.bib}
  \addbibresource{biblatex-examples.bib}
 }{}
\listfiles

\newcommand\SEC[1]{\textbf{\uppercase{#1}}}


\begin{document}

\title{Staged Two Beam Acceleration Beam Line Design for the AWA Facility}

\author{N. Neveu\thanks{nneveu@anl.gov}\textsuperscript{1}, 
	    L. Spentzouris, Illinois Institute of Technology, Chicago, IL, USA \\
	    J. G. Power, W. Gai \textsuperscript{1}Argonne National Laboratory, Lemont, IL, USA \\
	    C. Jing, Euclid Techlabs LLC, Solon, OH, USA}
\maketitle

%
\begin{abstract}
Two beam acceleration is a candidate for future high energy physics machines and FEL user facilities. 
This scheme consists of two independent electron accelerators operating synchronously. 
High-charge, 70 MeV bunch trains in the drive accelerator are injected from the rf photoinjector into
decelerating structures to generate a few hundred MW of rf power. 
This rf power is transferred through an rf waveguide to accelerating structures that are used to
accelerate the witness beam in an otherwise independent second accelerator and beam line. 
Staging refers to the sequential acceleration (energy gain) 
in two or more structures on the witness beam line. A kicker was incorporated on the 
drive beam line to accomplish a modular design so that each accelerating structure 
can be independently powered by a separate drive beam. 
Simulations were performed in OPAL-T to model the two beam lines. 
Beam sizes at the center of the structures was minimized to ensure good charge transmission. 
The resulting design will be the basis for proof of principle experiments that will take place 
at the Argonne Wakefield Accelerator (AWA) facility.
\end{abstract}


\section{AWA Facility}
The AWA facility houses two rf photoinjectors
operating at \SI{1.3}{GHz}. 
The bunch charge is routinely adjusted depending on the requirements 
of the experiments downstream of the photoinjector.
Typical operating charges are 1, 4, 10, and \SI{40}{nC}. 
While these are the most
common operating modes, other charges have been requested 
and provided depending on the experiment.
Recent experiments include emittance exchange \cite{eex}, 
high gradient structure tests \cite{pets}, thermal emittance measurements \cite{therm}, 
and two beam acceleration \cite{tba} (TBA), which is the subject of this proceeding. 

\section{Two Beam Acceleration Layout}
The AWA facility lends itself well to TBA experiments due to the 
close proximity of both operational photoinjectors. 
They are located in the same bunker about two meters apart.
This distance slightly varies along the beam 
lines. For the remainder of this paper, we will
refer to the low charge beam line as the "witness" line, 
and the high charge beam line is called the "drive" line.
While the charge on each line can be varied, for TBA experiments, 
the witness line is operated at \SI{1}{nC} and the drive line
is operated at \SI{40}{nC}.

The planned TBA layout is shown in Fig.~\ref{beamline}.
The drive line has six accelerating 
cavities with a maximum beam energy of \SI{70}{MeV}. 
The witness line has one accelerating cavity with a maximum beam energy of \SI{15}{MeV}.
The guns are located at opposite ends of the bunker and 
the propagation direction of the beam lines are opposing.
The witness line is operated in single bunch mode, and 
the drive line supplies high charge bunch trains. 
The planned experiments will include trains of eight bunches
with about \SI{40}{nC} in each bunch.

\begin{figure*}[!tbh]
	\centering
	\begin{tikzpicture}[scale=\textwidth/33cm, text=black]
	%\begin{tikzpicture}[scale=0.5, text=black]
	\input{tikz-tba.tex}
	\end{tikzpicture}
	\caption{TBA beam line layout at the AWA. The arrow at the end of each line indicates what direction the beam is traveling.
		PETS stands for Power Extraction and Transfer Structure, and ACC
		stands for Accelerating structure. The subscript index on each structure refers to which stage the structures belong to, first or second stage. }
	\label{beamline}
\end{figure*}

The kicker will ensure that one bunch train is supplied to 
each stage. This allows for more energy transfer in stage 2.
If the same bunch train supplies both stages \cite{tba}, the amount
of available energy for stage 2 would be decreased by the 
amount of energy deposited in stage 1. This would cause
unequal energy gain in each stage.
After passing through the kicker, 
the high charge bunch trains are guided to Power Extraction
and Transfer Structures (PETS) downstream. These are decelerating 
structures that extract power from the bunches through wakefield generation.
The structures take advantage of superposition and allow the wake from each 
bunch to combine with the others. The high power pulse generated by 
the combination of wakes is transfered through a waveguide to the 
witness line. 
There are two decelerating stages on the drive line, and two 
corresponding accelerating sections on the witness line.
The accelerating structures, ACC$_1$ and ACC$_2$, are only 
powered by the PETS on the drive line. There is no  
external power source. 

\subsection{Stage 1}
The first accelerating stage refers to portions of the beam line that 
include ACC$_1$ and PETS$_1$. This is the straight 
through portion of the drive beam line. The first bunch train passes through
 the kicker when it is unactivated, i.e. no pulse and no 
field are present.
The bunch train will then arrive at the quadrupoles before PETS$_1$. 
Focusing will ensure maximum transmission of the \SI{40}{nC} 
bunch train through PETS$_1$. Integrated Current Transformers (ICT's), 
are located before and after all PETS and ACC structures to monitor the 
transmission.   
\begin{figure}
	\begin{tikzpicture}[every node/.style={anchor=south west,inner sep=0pt},x=1mm, y=1mm,]   
	\node (fig1) at (0,0)
	{\includegraphics[width=0.5\textwidth]{stage}};
	\node[fill=white, inner sep=2pt] (txt2) at (15,10) {ACC};
	\node[fill=white, inner sep=2pt] (txt2) at (66,35) {PETS};
	\node[fill=white, inner sep=2pt, rotate=32] (txt2) at (35,33) {Waveguide};
	\end{tikzpicture}
	\caption{Example of a TBA stage at the AWA. This picture was taken during 
	a TBA experiment \cite{tba}. }
	\label{stage}
\end{figure}

Although not shown in Fig~\ref{beamline}, 
a spectrometer will be located at the end of the line for energy determination. Energy loss
in the bunch train will serve as a way to infer how much power was
deposited in PETS$_1$. Energy measurements in combination
with rf probe measurements in the transfer waveguide and ACC$_1$,
will be used to calculate losses in the stage and gradient. 

\subsection{Stage 2}
Stage 2 includes ACC$_2$ and PETS$_2$. After the first 
bunch train passes the kicker and goes on to PETS$_1$, 
the kicker will be pulsed. The second
bunch train will be directed to the bent beam line on the drive side.
Meanwhile on the witness side, the same bunch that was accelerated in 
stage 1 will arrive at stage 2, receiving a second increase in energy.
Successful energy gain in both stages is key to demonstrating staged TBA.
A second spectrometer will be located at the end of the bent beam line.
There will also be another set of spectrometers on the witness line.


\section{Synchronization}
The rf power generated on the drive line
must reach the accelerating structures as the witness beam is arriving.
This requires detailed understanding of the laser trigger, 
bunch train spacing, beams travel time, 
rise time of the rf pulse generated in the PETS, 
travel time of the pulse in the waveguide, and 
the fill time needed in the accelerating structures. 
While implementing this level of synchronization is not trivial, 
it is possible, and has been demonstrated in previous experiments
at the AWA \cite{tba}. 


To start, a \SI{248}{nm} ultraviolet (UV) laser is pulsed at 
the same repetition rate as the beam line. This is either 1, 2, or \SI{5}{Hz}.
The full width half maximum (FWHM) of the laser can be varied from \SI{1.5}{ps}
to \SI{10}{ps}. Depending on the FWHM, the duration of the UV 
pulse ranges from 1 to 3 degrees of the \SI{1.3}{GHz} rf frequency. 
 
Electron bunches in both the drive and witness line 
originate from the same UV laser pulse. 
A  network of UV optics and splitters 
deliver one laser pulse to the witness gun, and a pulse train to the drive gun \cite{korea}.
Each individual bunch is generated within one period (T) of the rf frequency; \SI{769}{ps}.
This brings the length of an eight bunch train to 7~T or about \SI{5.4}{ns}.
Spacing between the bunches allows the wake from previous bunches to 
start filling the PETS before the wake from the next bunch is introduced.

The drive bunch train must travel a longer distance to reach PETS$_1$ 
than the distance the witness bunch must travel to reach ACC$_1$.
The difference in travel time is accounted for by adding an optical delay 
to the UV optics preceding the witness gun. 
Once the rough timing is adjusted with the optics, fine timing is adjusted
by calculating the rise and fill times of the PETS and accelerating structures. 
The overall timing depends on two key events. When drive bunch train 1 reaches PETS$_1$, 
the witness bunch must be near the entrance of ACC$_1$. Next, when 
drive bunch train 2 reaches PETS$_2$, the same witness bunch that was accelerated
in ACC$_1$, should be approaching ACC$_2$. If the timing is correct, 
there will be sequential energy gain in both stages.


\section{Kicker}
A kicker was specifically fabricated 
for this experiment. The initial design was adapted from 
work done at Indiana University \cite{kicker,korea}. The plates were lengthened
to increase the beam deflection angle and the gap adjusted based on 
beam size simulations and mechanical constraints at the AWA. 
The design specifications and final kicker parameters 
are shown in Table~\ref{tkick}. The plates will be 
operated in differential mode, to get the highest field
possible from the available pulsar.
\begin{table}[hbt]
	%   \vspace*{-.5\baselineskip}
	\centering
	\caption{Final Kicker Parameters}
	\begin{tabular}{lc}
		\toprule
		\textbf{Parameter} & \textbf{Value} \\
		\midrule
		Charge       		& \SI{40}{nC}   \\ %[3pt]
		Beam Energy  		& \SI{70}{MeV}  \\ %[3pt]
		Angle 	     		& $2^{\circ}$ 	\\
		Gap Between Plates  & \SI{40}{mm}	\\		 
		Length of Plates    & \SI{500}{A}	\\
		Pulsar Voltage      & $\pm$\SI{24}{kV} \\
		Field Rise Time  	& \SI{3}{ns}    \\
		Field Duration 		& \SI{12}{ns}  \\ %[3pt]
		\bottomrule
	\end{tabular}
	\label{tkick}
	%   \vspace*{-\baselineskip}
\end{table}

After fabrication, a high voltage test was performed to ensure 
the electrical feedthroughs were sound. Special thanks to 
the Power Systems group at the Advanced Photon Source (APS) for testing 
the kicker in one of their rf cages, see Fig.~\ref{cage}.
A high voltage \SI{60}{Hz} 
source was used to probe one blade of the kicker at a time. 
Both sides performed well, with one blade showing slight discharge
at \SI{8}{kV} RMS, and the other showing no discharge up to \SI{9}{kV} RMS
(the limit of the source). These results are in line with other kickers
designed and tested at the APS \cite{mbakicker}.  

\begin{figure}
	\centering
	\begin{tikzpicture}[every node/.style={anchor=south west,inner sep=0pt},x=1mm, y=1mm,]   
	\node (fig1) at (0,0)
	{\includegraphics[width=0.47\textwidth]{kicker-hv}};
	\node[fill=white, inner sep=2pt] (txt2) at (40,15) {Kicker};
	\node[fill=white, inner sep=2pt] (txt2) at (10,40) {HV Source};
	\end{tikzpicture}
	\caption{Kicker in high voltage (HV) testing cage at the Advanced Photon Source. }
	\label{cage}
\end{figure}

\section{Ongoing Optimization}
While the number of optics elements and their 
general locations is known, there are still many free parameters that 
are not determined by that information alone. 
The strength of each magnet, the phase in each cavity, 
and the laser profile can all be freely adjusted.
This leads to a high dimensional optimization problem. 
The parameters on the drive line are further complicated
by strong space charge forces and bending elements (kicker, septum, dipole).

A first round multi-objective optimization of the drive line has been performed 
using the built in genetic algorithm (GA) in OPAL-T \cite{opal}. 
These simulations included the gun and all elements leading up 
to the entrance of the septum, as shown in Fig.~\ref{beamline}.
The objective was to optimize the beam size, emittance, and energy spread before the kicker.
Simulation parameters are shown in Table~\ref{simparam}.
The resulting Pareto fronts before the kicker at $Z=16.5m$, and after the 
kicker at $Z=18.5m$. Results so far indicate that both drive trains can be transported successfully.
A detailed description of the optimization work will be published soon.
\begin{table}[hbt]
	%   \vspace*{-.5\baselineskip}
	\centering
	\caption{Drive Line Simulation Parameters}
	\begin{tabular}{lcc}
		\toprule
		\textbf{Parameter} & \textbf{Fixed Values}  & \textbf{Optimization Values} \\
		\midrule
		Charge       & \SI{40}{nC}       & --   \\ %[3pt]
		Laser Radius & \SI{9}{mm}        & --    \\
		Gun Gradient & \SI{65}{MV/m}     & --  \\ %[3pt]
		Laser FWHM   & --      			 & \SI{1.5}{ps} $\leq F \leq $ \SI{10}{ps}	  \\ %[3pt]
		Gun Phase    & -- 				 & \SI{-30}{}$^{\circ} \leq \phi_g \leq$ \SI{0}{}$^{\circ}$  \\	
		$S_1$        & --		 		 & \SI{350}{A} $\leq B \leq $ \SI{500}{A}	  \\
		$S_2$		 & --  	 			 & \SI{170}{A} $\leq M \leq $ \SI{260}{A}	  \\
		Quads 		 & --				 & \SI{-8}{Tm^{-1}} $\leq M \leq $ \SI{8}{Tm^{-1}}	  \\
		\bottomrule
	\end{tabular}
	\label{simparam}
	%   \vspace*{-\baselineskip}
\end{table}
\begin{figure}
	\includegraphics[width=0.5\textwidth]{paper-pareto-ex-vs-rmss}
	\caption{Comparison of Pareto fronts before ($Z=16.5m$) and after ($Z=18.5m$) the kicker. }
	\label{initial}
\end{figure}

\section{Conclusion}
The TBA layout for upcoming experiments at the 
AWA facility is approximately decided. A fast rise time kicker
was designed and fabricated for this experiment. 
A successful high voltage test was preformed at the 
Advanced Photon Source. Initial optimization 
work has been done to determine
optics parameters for the high charge drive line.
Additional design work and optimization studies will be done 
to finalize the optics further downstream. 

\section{acknowledgments}
We gratefully acknowledge the computing resources
provided on Bebop, a high-performance computing cluster
operated by the LCRC at Argonne National Laboratory.
This material is based upon work supported by the 
U.S. Department of Energy, Office of Science, under 
contract number DE-AC02-06CH11357 and grant number DE-SC0015479. 
Travel to IPAC'18 supported by the United States National Science Foundation, 
the Division of Physics of Beams of the American Physical Society, and TRIUMF.


\begin{thebibliography}{9}
\bibitem{eex}
G.~Ha \emph{et al.}, “Demonstration of Current Profile 
Shaping using Double Dog-Leg Emittance Exchange Beam 
Line at Argonne Wakefield Accelerator”
in \textit{Proc. IPAC’16}, 
Busan, South Korea, May 2016, 
paper TUOBB01.

\bibitem{pets}
J.~Shao \emph{et al.}, 
“Recent Progress towards Dielectric Short Pulse Two-Beam Acceleration”
in \textit{Proc. IPAC’18}, 
Vancouver, Canada, May 2018, 
paper TUYGBE3.

\bibitem{therm}
L.~Zheng \emph{et al.}, “Measurements of Thermal Emittance 
for Cesium Telluride Photocathodes in an L-Band RF Gun”
in \textit{Proc. IPAC’17}, 
Copenhagen, Denmark, May 2017, 
paper TUPAB074.

\bibitem{tba}
J.~Shao \emph{et al.}, “Recent Two-Beam 
Acceleration Activities at Argonne Wakefield Accelerator Facility”
in \textit{Proc. IPAC’17}, 
Copenhagen, Denmark, May 2017, 
paper WEPVA022.

\bibitem{korea}
N.~Neveu \emph{et al.}, 
“Drive Generation and Propagation Studies for the Two
Beam Acceleration Experiment at the Argonne Wakefield
Accelerator,”
in \emph{Proc. IPAC'16}, 
Busan, South Korea, May 2016, 
paper TUPMY036.

\bibitem{kicker}
T.~H.~Luo \emph{et al.}, “Design of a Fast
Extraction Kicker for the ALPHA Project”
in \textit{Proc. IPAC’10}, 
Kyoto, Japan, May 2010, 
paper THPEA052.

\bibitem{mbakicker}
C.~Yao \emph{et al.}, 
“Development of Fast Kickers for the APS
MBA Upgrade,”
in \emph{Proc. IPAC'15}, 
Richmond, VA, USA, May 2015, 
paper WEPTY014.

\bibitem{opal}
A.~Adelmann \emph{et al.},
“The OPAL (Object Oriented Parallel Accelerator Library) framework,”
PSI, Zurich, Switzerland,
Rep. PSI-PR-08-02, 2008-2017.

\end{thebibliography}

\end{document}
	
