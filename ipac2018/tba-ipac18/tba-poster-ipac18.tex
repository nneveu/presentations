\documentclass[portrait,final,paperwidth=92cm, paperheight=152cm,  fontscale=0.277]{baposter}
%a0paper,
\usepackage{calc}
\usepackage{graphicx}
\usepackage{amsmath}
\usepackage{amssymb}
\usepackage{relsize}
\usepackage{multirow}
\usepackage{rotating}
\usepackage{bm}
\usepackage{url}
\usepackage{setspace}
\usepackage{graphicx}
\usepackage{multicol}
\usepackage{siunitx}
\usepackage{vwcol} 
\usepackage{varwidth}
\usepackage{tikz}
\usepackage{graphicx}
\graphicspath{{/home/nicole/Documents/presentations/ipac2018/logos}{/home/nicole/Documents/presentations/ipac2018/tba-ipac18}}
\usetikzlibrary{calc}
%\usepackage{times}
%\usepackage{helvet}
%\usepackage{bookman}
\usepackage{palatino}
\newcommand\Tstrut{\rule{0pt}{2.6ex}}         % = `top' strut

\newcommand{\captionfont}{\footnotesize}

\graphicspath{{images/}{../images/}}
\usepackage{tikz}
\usetikzlibrary{shapes.misc}
\usetikzlibrary{shapes,arrows,decorations.markings,shadows,positioning}
\usetikzlibrary{snakes,backgrounds,spy}
\usepackage{tcolorbox}
\newcommand{\SET}[1]  {\ensuremath{\mathcal{#1}}}
\newcommand{\MAT}[1]  {\ensuremath{\boldsymbol{#1}}}
\newcommand{\VEC}[1]  {\ensuremath{\boldsymbol{#1}}}
\newcommand{\Video}{\SET{V}}
\newcommand{\video}{\VEC{f}}
\newcommand{\track}{x}
\newcommand{\Track}{\SET T}
\newcommand{\LMs}{\SET L}
\newcommand{\lm}{l}
\newcommand{\PosE}{\SET P}
\newcommand{\posE}{\VEC p}
\newcommand{\negE}{\VEC n}
\newcommand{\NegE}{\SET N}
\newcommand{\Occluded}{\SET O}
\newcommand{\occluded}{o}
\renewcommand{\baselinestretch}{-1.5} 
%%%%%%%%%%%%%%%%%%%%%%%%%%%%%%%%%%%%%%%%%%%%%%%%%%%%%%%%%%%%%%%%%%%%%%%%%%%%%%%%
%%%% Some math symbols used in the text
%%%%%%%%%%%%%%%%%%%%%%%%%%%%%%%%%%%%%%%%%%%%%%%%%%%%%%%%%%%%%%%%%%%%%%%%%%%%%%%%

%%%%%%%%%%%%%%%%%%%%%%%%%%%%%%%%%%%%%%%%%%%%%%%%%%%%%%%%%%%%%%%%%%%%%%%%%%%%%%%%
% Multicol Settings
%%%%%%%%%%%%%%%%%%%%%%%%%%%%%%%%%%%%%%%%%%%%%%%%%%%%%%%%%%%%%%%%%%%%%%%%%%%%%%%%
\setlength{\columnsep}{1.5em}
\setlength{\columnseprule}{0mm}

%%%%%%%%%%%%%%%%%%%%%%%%%%%%%%%%%%%%%%%%%%%%%%%%%%%%%%%%%%%%%%%%%%%%%%%%%%%%%%%%
% Save space in lists. Use this after the opening of the list
%%%%%%%%%%%%%%%%%%%%%%%%%%%%%%%%%%%%%%%%%%%%%%%%%%%%%%%%%%%%%%%%%%%%%%%%%%%%%%%%
\newcommand{\compresslist}{%
	\setlength{\itemsep}{1pt}%
	\setlength{\parskip}{0pt}%
	\setlength{\parsep}{0pt}%
}

%%%%%%%%%%%%%%%%%%%%%%%%%%%%%%%%%%%%%%%%%%%%%%%%%%%%%%%%%%%%%%%%%%%%%%%%%%%%%%
%%% Begin of Document
%%%%%%%%%%%%%%%%%%%%%%%%%%%%%%%%%%%%%%%%%%%%%%%%%%%%%%%%%%%%%%%%%%%%%%%%%%%%%%
\begin{document}

%%%%%%%%%%%%%%%%%%%%%%%%%%%%%%%%%%%%%%%%%%%%%%%%%%%%%%%%%%%%%%%%%%%%%%%%%%%%%%
%%% Here starts the poster
%%%---------------------------------------------------------------------------
%%% Format it to your taste with the options
%%%%%%%%%%%%%%%%%%%%%%%%%%%%%%%%%%%%%%%%%%%%%%%%%%%%%%%%%%%%%%%%%%%%%%%%%%%%%%
\definecolor{lightblue}{rgb}{0.145,0.6666,1}

\definecolor{silver}{cmyk}{0,0,0,0.3}
\definecolor{black}{cmyk}{0,0,0.0,1.0}
\definecolor{darkYellow}{cmyk}{0,0,1.0,0.5}
\definecolor{darkSilver}{cmyk}{0,0,0,0.1}

\definecolor{KTHBlue}{cmyk}{.71,.37,0.07,0}
\definecolor{KTHsilver}{cmyk}{0,0,0,0.35}
\definecolor{KTHbeige}{cmyk}{0,0.03,0.19,0.04}

% Argonne Logo Colors
\definecolor{ArgoneLogoBlue}{RGB}{4,146,210}
\definecolor{ArgoneLogoRed}{RGB}{228,32,41}
\definecolor{ArgoneLogoGreen}{RGB}{120,202,42}
\definecolor{PMSCoolGray}{RGB}{112,109,110}


%\definecolor{KTHBlue}{RGB}{25,84,166}
\begin{poster}{
  % Poster Options
	% Show grid to help with alignment
	grid=false,
	columns=6,
	% Column spacing
	colspacing=1em,
	% Color style
	bgColorOne=white,
	bgColorTwo=white,
	borderColor=KTHBlue,
	headerColorOne=black,
	headerColorTwo=KTHBlue,
	headerFontColor=white,
	boxColorOne=white,
	boxColorTwo=KTHBlue,
	% Format of textbox
	textborder=roundedleft,
	% Format of text header
	eyecatcher=true,
	headerborder=closed,
	headerheight=0.1\textheight,
	%  textfont=\sc, An example of changing the text font
	headershape=roundedright,
	headershade=shadelr,
	headerfont=\Large\bf\textsc, %Sans Serif
	textfont={\setlength{\parindent}{0em}},
	boxshade=plain,
	%  background=shade-tb,
	background=plain,
	linewidth=2pt
}
% Eye Catcher
{
	
	%\makebox[3em][l]{
		
} %}
% Title
{\bf\textsc{Staged Two Beam Acceleration Beam\\
		\vspace{0.3em} Line Design for the AWA Facility}}%\vspace{10em}}
% Authors
{\vspace{1em}
	N. Neveu\textsuperscript{1}, L. Spentzouris, Illinois Institute of Technology, Chicago, IL, USA \\
	J. G. Power, W. Gai, \textsuperscript{1}Argonne National  Laboratory, Lemont, IL, USA \\
	C. Jing, Euclid Techlabs LLC, Solon, OH, USA \\
    }
% University logo
{% The makebox allows the title to flow into the logo, this is a hack because of the L shaped logo.
	%\makebox[3em][r]{
		\includegraphics[height=9em]{logos/combo_logo}%}
}




%%%%%%%%%%%%%%%%%%%%%%%%%%%%%%%%%%%%%%%%%%%%%%%%%%%%%%%%%%%%%%%%%%%%%%%%%%%%%%
\headerbox{AWA Facility}{name=problem,column=0,row=0, span=2}{
%%%%%%%%%%%%%%%%%%%%%%%%%%%%%%%%%%%%%%%%%%%%%%%%%%%%%%%%%%%%%%%%%%%%%%%%%%%%%%
The AWA facility houses two rf photoinjectors
operating at \SI{1.3}{GHz}. 
The bunch charge is routinely adjusted for experiments downstream of the photoinjector.
Typical operating charges are 1, 4, 10, and \SI{40}{nC}. 
While these are the most
common operating modes, other charges have been requested 
and provided depending on the experiment.
Recent experiments include emittance exchange, 
high gradient structure tests, thermal emittance measurements, 
and two beam acceleration (TBA). 
}

%%%%%%%%%%%%%%%%%%%%%%%%%%%%%%%%%%%%%%%%%%%%%%%%%%%%%%%%%%%%%%%%%%%%%%%%%%%%%%
\headerbox{Introduction}{name=sims,column=2,row=0,span=2}{
%%%%%%%%%%%%%%%%%%%%%%%%%%%%%%%%%%%%%%%%%%%%%%%%%%%%%%%%%%%%%%%%%%%%%%%%%%%%%%
The AWA facility lends well to TBA experiments due to the 
close proximity of both operational photoinjectors. 
They are located in the same bunker about two meters apart.
The low charge beam line is refered to as the "witness" line, 
and the high charge beam line is called the "drive" line.
For TBA experiments, the witness line is operated at \SI{1}{nC} 
and the drive line is operated at \SI{40}{nC} per bunch.
The witness line is operated in single bunch mode, and 
the drive line supplies high charge bunch trains. 

 
}

%%%%%%%%%%%%%%%%%%%%%%%%%%%%%%%%%%%%%%%%%%%%%%%%%%%%%%%%%%%%%%%%%%%%%%%%%%%%%%
\headerbox{Wakefield Generation}{name=staging,column=4,row=0,span=2,}{
%%%%%%%%%%%%%%%%%%%%%%%%%%%%%%%%%%%%%%%%%%%%%%%%%%%%%%%%%%%%%%%%%%%%%%%%%%%%%%
High charge bunch trains are supplied to Power Extraction
and Transfer Structures (PETS) downstream. These are decelerating 
structures that extract power from the bunches through wakefield generation.
Supperposition combines the wake from each bunch into one rf pulse.
The high power pulse is generated by the combination of wakes
and transfered through a waveguide to the witness line. 
The accelerating structures, ACC$_{1,2}$ are only 
powered by the PETS on the drive line. There is no  
external power source.

}

%%%%%%%%%%%%%%%%%%%%%%%%%%%%%%%%%%%%%%%%%%%%%%%%%%%%%%%%%%%%%%%%%%%%%%%%%%%%%%
\headerbox{Two Beam Acceleration Layout}{name=beamline,column=0,below=staging, span=6}{%, bottomaligned=conclusion}{
%%%%%%%%%%%%%%%%%%%%%%%%%%%%%%%%%%%%%%%%%%%%%%%%%%%%%%%%%%%%%%%%%%%%%%%%%%%%%%
%\noindent
\begin{centering}
	%\begin{tikzpicture}[scale=\textwidth/33cm, text=black]
	\begin{tikzpicture}[scale=0.72, text=black]
	\input{tikz-tba.tex}
	\end{tikzpicture}
\end{centering}

\vspace{1em}
The arrows indicate what direction the beams travels.
The guns are located at opposite ends of the bunker and 
the propagation direction of the beam lines are opposing.
PETS stands for Power Extraction and Transfer Structure, and ACC stands for Accelerating structure. 
The subscript on each structure refers to which stage the structures belong to (first or second). 
The drive line has six accelerating cavities with a maximum beam energy of \SI{70}{MeV}. 
The witness line has one accelerating cavity with a max beam energy of \SI{15}{MeV}.
}
%%%%%%%%%%%%%%%%%%%%%%%%%%%%%%%%%%%%%%%%%%%%%%%%%%%%%%%%%%%%%%%%%%%%%%%%%%%%%%
\headerbox{Staging}{name=stage1,column=2,below=beamline, span=2}{
%%%%%%%%%%%%%%%%%%%%%%%%%%%%%%%%%%%%%%%%%%%%%%%%%%%%%%%%%%%%%%%%%%%%%%%%%%%%%%
\textbf{Stage 1:}
The first accelerating stage refers ACC$_1$, and PEST$_1$. This is the straight 
through portion of the drive beam line. The first bunch train will be allowed
to pass the kicker when it is unactivated, i.e. not pulse and no field present.
Focusing in the quads will ensure maximum transmission of the \SI{40}{nC} 
bunch train through PEST$_1$. Integrated Current Transformers (ICT's), 
are located before and after all PETS and ACC structures to monitor the 
transmission. 
\vspace{1em}

\textbf{Stage 2:}
Likewise, stage 2 supports ACC$_2$ and PETS$_2$. After the first 
bunch train passes the kicker and goes on to PETS$_1$, 
the kicker will be pulsed. The second bunch train will be directed to the bent beam line on drive side.
Meanwhile the same bunch on the witness side that was accelerated in 
stage 1 will arrive at stage 2, receiving a second increase in energy.
\textbf{Successful energy gain in both stages is key to demonstrating staged TBA.}
	
}
%%%%%%%%%%%%%%%%%%%%%%%%%%%%%%%%%%%%%%%%%%%%%%%%%%%%%%%%%%%%%%%%%%%%%%%%%%%%%%
\headerbox{Experimental Stage}{name=stage3,column=4,below=beamline, span=2}{
%%%%%%%%%%%%%%%%%%%%%%%%%%%%%%%%%%%%%%%%%%%%%%%%%%%%%%%%%%%%%%%%%%%%%%%%%%%%%
\centering
\begin{tikzpicture}[every node/.style={anchor=south west,inner sep=0pt},x=1mm, y=1mm,]   
\node (fig1) at (0,0)
{\includegraphics[width=1\textwidth]{stage}};
\node[fill=white, inner sep=2pt] (txt2) at (10,10) {ACC};
\node[fill=white, inner sep=2pt] (txt2) at (57,30) {PETS};
\node[fill=white, inner sep=2pt, rotate=32] (txt2) at (27,29) {Waveguide};
\end{tikzpicture}

\vspace{1em}
Example of a TBA stage at the AWA. This picture was taken during 
metallic TBA experiment \cite{tba}.  
	
}

%%%%%%%%%%%%%%%%%%%%%%%%%%%%%%%%%%%%%%%%%%%%%%%%%%%%%%%%%%%%%%%%%%%%%%%%%%%%%%
\headerbox{kicker}{name=kicker,column=0,row=0,span=2, below=beamline}{
%%%%%%%%%%%%%%%%%%%%%%%%%%%%%%%%%%%%%%%%%%%%%%%%%%%%%%%%%%%%%%%%%%%%%%%%%%%%%%

		A kicker was specifically fabricated 
		for this experiment. The initial design was adapted from 
		work done at Indiana University. The plates were lengthened
		to increase the angle and the gap adjusted based on 
		beam size simulations and mechanical constraints at the AWA.
		The plates will be 
		operated in differential mode, to get the highest field
		possible from the available pulsar.
		\vspace{-1.2em}
\begin{center}
	\singlespacing
		\begin{tabular}{lc}
		\hline  %\toprule 
		\textbf{Parameter} & \textbf{Value} \\ \hline
		%\midrule
		Charge       		& \SI{40}{nC}    \\ %[3pt]
		Beam Energy  		& \SI{70}{MeV}  \\ %[3pt]
		Angle 	     		& $2^{\circ}$ 	\\ 
		Gap Between Plates  & \SI{40}{mm}	\\		 
		Length of Plates    & \SI{1}{m}	\\
		Pulsar Voltage      & $\pm$\SI{24}{kV} \\
		Field Rise Time  	& \SI{3}{ns}    \\
		Min Field Duration 	& \SI{12}{ns}  \\ \hline%[3pt]
		%\bottomrule
	\end{tabular}
\end{center}

}


%%%%%%%%%%%%%%%%%%%%%%%%%%%%%%%%%%%%%%%%%%%%%%%%%%%%%%%%%%%%%%%%%%%%%%%%%%%%%%
\headerbox{High Voltage Test}{name=hv,column=0,row=0,span=2, below=kicker}{
%%%%%%%%%%%%%%%%%%%%%%%%%%%%%%%%%%%%%%%%%%%%%%%%%%%%%%%%%%%%%%%%%%%%%%%%%%%%%%
\begin{center}
\begin{tikzpicture}[every node/.style={anchor=south west,inner sep=0pt},x=1mm, y=1mm,]   
\node (fig1) at (0,0)
{\includegraphics[width=1\textwidth]{kicker-hv}};
\node[fill=white, inner sep=2pt] (txt2) at (40,15) {Kicker};
\node[fill=white, inner sep=2pt] (txt2) at (12,40) {HV Source};
\end{tikzpicture}

\vspace{1em}
Kicker in high voltage (HV) testing cage at the Advanced Photon Source (APS). 
%\vspace{1em}
\end{center}

After fabrication,
a high voltage test was performed to ensure 
the electrical feedthroughs were sound. Special thanks to 
the Power Systems group at the APS for testing 
the kicker in one of their rf cages.
A high voltage \SI{60}{Hz} 
source was used to probe one blade of the kicker at a time. 
Both sides performed well, one blade showed a slight discharge
at \SI{8}{kV} RMS, and the other showing no discharge up to \SI{9}{kV} RMS
(the limit of the source). These results are in line with other kickers
designed and tested at the APS.  	
}
%%%%%%%%%%%%%%%%%%%%%%%%%%%%%%%%%%%%%%%%%%%%%%%%%%%%%%%%%%%%%%%%%%%%%%%%%%%%%%
\headerbox{Optimization}{name=kicker,column=2,row=0,span=4, below=stage3}{
%%%%%%%%%%%%%%%%%%%%%%%%%%%%%%%%%%%%%%%%%%%%%%%%%%%%%%%%%%%%%%%%%%%%%%%%%%%%%%
While the number of optics elements and their 
general locations is known, there are still many free parameters that 
are not determined by that information alone. 
The strength of each magnet, the phase in each cavity, 
and the laser profile can all be freely adjusted.
This leads to a high dimensional optimization problem. 
The parameters on the drive line are further complicated
by strong space charge forces and bending elements (kicker, septum, dipole).

A first round multi-objective optimization of the drive line has been performed 
using the built in genetic algorithm (GA) in OPAL-T \cite{opal}. 
These simulations included the gun and all elements leading up 
to the entrance of the septum, as shown in Fig.~\ref{beamline}.
The objectives were beam size, emittance, and energy spread before the kicker.
Simulation parameters are shown in Table~\ref{simparam}.
Results of this optimization work will be published soon.

\centering
\vspace{-0.5em}
\begin{minipage}{0.99\textwidth}
	\centering
	\begin{tabular}{ lcc}
		Solenoid Strength & $ 200 \le \text{I}_{\text{bs}}\text{(A)} \le 500$  \Tstrut \\
		Solenoid Strength & $ 170 \le \text{I}_{\text{s}}\text{(A)} \le 260$  \Tstrut \\
		Phase of Gun & $-30^\circ \le \phi_{\text{gun}} \le 0^\circ$  \\
		Quad 1  & $-8.8 \le \text{kQ}_{1}\text{(m}^{-1}) \le 8.8$  \\
		Quad 2 & $-8.8 \le \text{kQ}_{2}\text{(m}^{-1}) \le 8.8$ 
		%\bottomrule    
	\end{tabular}
\end{minipage}	
}


%%%%%%%%%%%%%%%%%%%%%%%%%%%%%%%%%%%%%%%%%%%%%%%%%%%%%%%%%%%%%%%%%%%%%%%%%%%%%%
\headerbox{Conclusion}{name=ref,column=2,above=bottom, span=2}{%
	%%%%%%%%%%%%%%%%%%%%%%%%%%%%%%%%%%%%%%%%%%%%%%%%%%%%%%%%%%%%%%%%%%%%%%%%%%%%%%
	\noindent 
The TBA layout for upcoming experiments at the 
AWA facility is approximately decided. A fast rise time kicker
was designed and fabricated for this experiment. 
A high voltage test was preformed at the 
Advanced Photon Source. Initial optimization 
work has been done to decide  
optics parameters for the high charge drive line.
Additional design work and optimization studies will be done 
to finalize the optics further downstream. 
}

%%%%%%%%%%%%%%%%%%%%%%%%%%%%%%%%%%%%%%%%%%%%%%%%%%%%%%%%%%%%%%%%%%%%%%%%%%%%%%
\headerbox{Acknowlegements}{name=ref,column=4,above=bottom, span=2}{%
%%%%%%%%%%%%%%%%%%%%%%%%%%%%%%%%%%%%%%%%%%%%%%%%%%%%%%%%%%%%%%%%%%%%%%%%%%%%%%
\noindent 
We gratefully acknowledge the computing resources
provided on Bebop, a HPC cluster operated by the LCRC at ANL.
This material is based upon work supported by the 
U.S. DOE, Office of Science, under 
contract number DE-AC02-06CH11357 and grant number DE-SC0015479. 
Travel to IPAC'18 supported by the Division of Physics 
of the United States NSF 
Accelerator Science Program and the Division of 
Beam Physics of the APS.

\vspace{1em}
\begin{centering}
	\includegraphics[width=0.1\textwidth]{logos/aps-logo}\hspace{5mm}
	\includegraphics[width=0.3\textwidth]{logos/DOE_logo_color_cmyk}
	
\end{centering}

}


\end{poster}%
%
\end{document}
