\documentclass[portrait,final,paperwidth=120cm, paperheight=240cm,  fontscale=0.277]{baposter}
%a0paper,
\usepackage{calc}
\usepackage{graphicx}
\usepackage{amsmath}
\usepackage{amssymb}
\usepackage{relsize}
\usepackage{multirow}
\usepackage{rotating}
\usepackage{bm}
\usepackage{url}

\usepackage{graphicx}
\usepackage{multicol}
\usepackage{siunitx}
\usepackage{vwcol} 
\usepackage{varwidth}
\usepackage{tikz}
\usepackage{graphicx}
\graphicspath{{/home/nicole/Documents/presentations/ipac2018/logos}{/home/nicole/Documents/presentations/ipac2018/tba-ipac18}}
\usetikzlibrary{calc}
%\usepackage{times}
%\usepackage{helvet}
%\usepackage{bookman}
\usepackage{palatino}
\newcommand\Tstrut{\rule{0pt}{2.6ex}}         % = `top' strut

\newcommand{\captionfont}{\footnotesize}

\graphicspath{{images/}{../images/}}
\usepackage{tikz}
\usetikzlibrary{shapes.misc}
\usetikzlibrary{shapes,arrows,decorations.markings,shadows,positioning}
\usetikzlibrary{snakes,backgrounds,spy}
\usepackage{tcolorbox}
\newcommand{\SET}[1]  {\ensuremath{\mathcal{#1}}}
\newcommand{\MAT}[1]  {\ensuremath{\boldsymbol{#1}}}
\newcommand{\VEC}[1]  {\ensuremath{\boldsymbol{#1}}}
\newcommand{\Video}{\SET{V}}
\newcommand{\video}{\VEC{f}}
\newcommand{\track}{x}
\newcommand{\Track}{\SET T}
\newcommand{\LMs}{\SET L}
\newcommand{\lm}{l}
\newcommand{\PosE}{\SET P}
\newcommand{\posE}{\VEC p}
\newcommand{\negE}{\VEC n}
\newcommand{\NegE}{\SET N}
\newcommand{\Occluded}{\SET O}
\newcommand{\occluded}{o}
\renewcommand{\baselinestretch}{-1.5} 
%%%%%%%%%%%%%%%%%%%%%%%%%%%%%%%%%%%%%%%%%%%%%%%%%%%%%%%%%%%%%%%%%%%%%%%%%%%%%%%%
%%%% Some math symbols used in the text
%%%%%%%%%%%%%%%%%%%%%%%%%%%%%%%%%%%%%%%%%%%%%%%%%%%%%%%%%%%%%%%%%%%%%%%%%%%%%%%%

%%%%%%%%%%%%%%%%%%%%%%%%%%%%%%%%%%%%%%%%%%%%%%%%%%%%%%%%%%%%%%%%%%%%%%%%%%%%%%%%
% Multicol Settings
%%%%%%%%%%%%%%%%%%%%%%%%%%%%%%%%%%%%%%%%%%%%%%%%%%%%%%%%%%%%%%%%%%%%%%%%%%%%%%%%
\setlength{\columnsep}{1.5em}
\setlength{\columnseprule}{0mm}

%%%%%%%%%%%%%%%%%%%%%%%%%%%%%%%%%%%%%%%%%%%%%%%%%%%%%%%%%%%%%%%%%%%%%%%%%%%%%%%%
% Save space in lists. Use this after the opening of the list
%%%%%%%%%%%%%%%%%%%%%%%%%%%%%%%%%%%%%%%%%%%%%%%%%%%%%%%%%%%%%%%%%%%%%%%%%%%%%%%%
\newcommand{\compresslist}{%
	\setlength{\itemsep}{1pt}%
	\setlength{\parskip}{0pt}%
	\setlength{\parsep}{0pt}%
}

%%%%%%%%%%%%%%%%%%%%%%%%%%%%%%%%%%%%%%%%%%%%%%%%%%%%%%%%%%%%%%%%%%%%%%%%%%%%%%
%%% Begin of Document
%%%%%%%%%%%%%%%%%%%%%%%%%%%%%%%%%%%%%%%%%%%%%%%%%%%%%%%%%%%%%%%%%%%%%%%%%%%%%%
\begin{document}

%%%%%%%%%%%%%%%%%%%%%%%%%%%%%%%%%%%%%%%%%%%%%%%%%%%%%%%%%%%%%%%%%%%%%%%%%%%%%%
%%% Here starts the poster
%%%---------------------------------------------------------------------------
%%% Format it to your taste with the options
%%%%%%%%%%%%%%%%%%%%%%%%%%%%%%%%%%%%%%%%%%%%%%%%%%%%%%%%%%%%%%%%%%%%%%%%%%%%%%
\definecolor{lightblue}{rgb}{0.145,0.6666,1}

\definecolor{silver}{cmyk}{0,0,0,0.3}
\definecolor{black}{cmyk}{0,0,0.0,1.0}
\definecolor{darkYellow}{cmyk}{0,0,1.0,0.5}
\definecolor{darkSilver}{cmyk}{0,0,0,0.1}

\definecolor{KTHBlue}{cmyk}{.71,.37,0.07,0}
\definecolor{KTHsilver}{cmyk}{0,0,0,0.35}
\definecolor{KTHbeige}{cmyk}{0,0.03,0.19,0.04}

% Argonne Logo Colors
\definecolor{ArgoneLogoBlue}{RGB}{4,146,210}
\definecolor{ArgoneLogoRed}{RGB}{228,32,41}
\definecolor{ArgoneLogoGreen}{RGB}{120,202,42}
\definecolor{PMSCoolGray}{RGB}{112,109,110}


%\definecolor{KTHBlue}{RGB}{25,84,166}
\begin{poster}{
  % Poster Options
	% Show grid to help with alignment
	grid=false,
	columns=6,
	% Column spacing
	colspacing=1em,
	% Color style
	bgColorOne=white,
	bgColorTwo=white,
	borderColor=KTHBlue,
	headerColorOne=black,
	headerColorTwo=KTHBlue,
	headerFontColor=white,
	boxColorOne=white,
	boxColorTwo=KTHBlue,
	% Format of textbox
	textborder=roundedleft,
	% Format of text header
	eyecatcher=true,
	headerborder=closed,
	headerheight=0.14\textheight,
	%  textfont=\sc, An example of changing the text font
	headershape=roundedright,
	headershade=shadelr,
	headerfont=\Large\bf\textsc, %Sans Serif
	textfont={\setlength{\parindent}{0em}},
	boxshade=plain,
	%  background=shade-tb,
	background=plain,
	linewidth=2pt
}
% Eye Catcher
{
	
	\makebox[10em][l]{
		\includegraphics[height=7em]{logos/argonne_logo}
} }
% Title
{\bf{Bunch Length Measurements Using CTR \\
		at the AWA with Comparison to Simulation}}%\vspace{0.1em}}
% Authors
{\vspace{1em}
	N. Neveu\textsuperscript{1}, L. Spentzouris, Illinois Institute of Technology, Chicago, IL, USA \\
	A. Halavanau, P. Piot\textsuperscript{2}, Northern Illinois University, DeKalb, IL, USA \\
	J. G. Power, E. Wisniewski, C. Whiteford, \textsuperscript{1}Argonne National  Laboratory, Lemont, IL, USA \\
	S. Antipov, Euclid Techlabs LLC, Solon, OH, USA \\
	\textsuperscript{2} also at Fermilab, Batavia, IL, USA \\
    }
% University logo
{% The makebox allows the title to flow into the logo, this is a hack because of the L shaped logo.
		\makebox[10em][r]{%
			\hfill
				\includegraphics[height=5em]{logos/IIT_logo}
}

}


%%%%%%%%%%%%%%%%%%%%%%%%%%%%%%%%%%%%%%%%%%%%%%%%%%%%%%%%%%%%%%%%%%%%%%%%%%%%%%
\headerbox{AWA Facility}{name=problem,column=0,row=0, span=2}{
%%%%%%%%%%%%%%%%%%%%%%%%%%%%%%%%%%%%%%%%%%%%%%%%%%%%%%%%%%%%%%%%%%%%%%%%%%%%%%
The AWA facility houses two rf photoinjectors
operating at \SI{1.3}{GHz}. 
The bunch charge is routinely adjusted for depending on the requirements 
of the experiments downstream of the photoinjector.
Typical operating charges are 1, 4, 10, and \SI{40}{nC}. 
While these are the most
common operating modes, other charges have been requested 
and provided depending on the experiment.
Recent experiments include emittance exchange \cite{eex}, 
high gradient structure tests \cite{pets}, thermal emittance measurements \cite{therm}, 
and two beam acceleration \cite{tba} (TBA), which is the subject of this proceeding. 
}

%%%%%%%%%%%%%%%%%%%%%%%%%%%%%%%%%%%%%%%%%%%%%%%%%%%%%%%%%%%%%%%%%%%%%%%%%%%%%%
\headerbox{hold}{name=sims,column=2,row=0,span=2}{
%%%%%%%%%%%%%%%%%%%%%%%%%%%%%%%%%%%%%%%%%%%%%%%%%%%%%%%%%%%%%%%%%%%%%%%%%%%%%%
The AWA facility lends well to TBA experiments due to the 
close proximity of both operational photoinjectors. 
They are located in the same bunker and located about two meters apart.
This number slightly varies depending on which part of the beam 
lines you are comparing. For the remainder of this paper, we will
refer to the low charge beam line as the "witness" line, 
and the high charge, beam line as the "drive" line.
While the charge on each line can be varied, for TBA experiments, 
the witness line is operated at \SI{1}{nC} and the drive line
is operated at \SI{40}{nC}.

The planned TBA layout is shown in Fig.~\ref{beamline}.
The drive line has six accelerating 
cavities with a maximum beam energy of \SI{70}{MeV}. 
The witness line has accelerating cavity with a max beam energy of \SI{15}{MeV}.
The guns are located at opposite ends of the bunker and 
the propagation direction of the beam lines are opposing.
The witness line is operated in single bunch mode, and 
the drive line supplies high charge bunch trains. 
The planned experiments will include trains of eight bunches
with about \SI{40}{nC} in each bunch.
}

%%%%%%%%%%%%%%%%%%%%%%%%%%%%%%%%%%%%%%%%%%%%%%%%%%%%%%%%%%%%%%%%%%%%%%%%%%%%%%
\headerbox{AWA Model Parameters $\vec{\lambda}$}{name=multiple,column=4,row=0,span=2}{
%%%%%%%%%%%%%%%%%%%%%%%%%%%%%%%%%%%%%%%%%%%%%%%%%%%%%%%%%%%%%%%%%%%%%%%%%%%%%%
\centering
\vspace{-0.5em}
\begin{minipage}{0.99\textwidth}
	\centering
	\begin{tabular}{ lcc}
		Solenoid Strength & $ 200 \le \text{I}_{\text{bs}}\text{(A)} \le 500$  \Tstrut \\
		Solenoid Strength & $ 170 \le \text{I}_{\text{s}}\text{(A)} \le 260$  \Tstrut \\
		Phase of Gun & $-30^\circ \le \phi_{\text{gun}} \le 0^\circ$  \\
		Quad 1  & $-8.8 \le \text{kQ}_{1}\text{(m}^{-1}) \le 8.8$  \\
		Quad 2 & $-8.8 \le \text{kQ}_{2}\text{(m}^{-1}) \le 8.8$ 
		%\bottomrule    
	\end{tabular}
\end{minipage}
}

%%%%%%%%%%%%%%%%%%%%%%%%%%%%%%%%%%%%%%%%%%%%%%%%%%%%%%%%%%%%%%%%%%%%%%%%%%%%%%
\headerbox{Two Beam Acceleration Layout}{name=beamline,column=0,below=problem, span=6}{%, bottomaligned=conclusion}{
%%%%%%%%%%%%%%%%%%%%%%%%%%%%%%%%%%%%%%%%%%%%%%%%%%%%%%%%%%%%%%%%%%%%%%%%%%%%%%
\noindent
\begin{centering}
	%\begin{tikzpicture}[scale=\textwidth/33cm, text=black]
	\begin{tikzpicture}[scale=0.9, text=black]
	\def \gunleft {-1.0}
\def \gunright {0.3}
\def \loneright {1.0}
\def \ltworight {2.0}
\def \lthreeright {3.0}
\def \lfourright {4.0}
\def \lfiveright {5.0}
\def \lsixright {6.0}
\def \quadone {7.5}
\def \quadfour{16}

\draw[fill=orange, thick, rounded corners =0.1cm] (\gunleft-0.1,0.5)rectangle (\gunright,1.5) node[pos=.5, white] {\textbf{Gun}} ;

%S1
\node[] at (-1,2.9) {$S_1$};
\draw[thick, fill=black!60!green] (-1.4,-0.5)rectangle  (-1.0,0.5) node[pos=.5, white] {} ;
\draw[black,  thick] (-1.4,-0.5) -- (-1.0,0.5);
\draw[black,  thick] (-1.4,0.5) -- (-1.0,-0.5);
\draw[ thick, fill=black!60!green] (-1.4,1.5)rectangle  (-1.0,2.5) node[pos=.5, white] {} ;
\draw[black,  thick] (-1.4,1.5) -- (-1.0,2.5);
\draw[black,  thick] (-1.4,2.5) -- (-1.0,1.5);

\draw[ thick, fill=black!60!green] (-1.0,-0.5)rectangle  (-0.6,0.5) node[pos=.5, white] {} ;
\draw[black,  thick] (-1.0,-0.5) -- (-0.6,0.5);
\draw[black,  thick] (-1.0,0.5) -- (-0.6,-0.5);
\draw[ thick, fill=black!60!green] (-1.0,1.5)rectangle  (-0.6,2.5) node[pos=.5, white] {} ;
\draw[black,  thick] (-1.0,1.5) -- (-0.6,2.5);
\draw[black,  thick] (-1.0,2.5) -- (-0.6,1.5);

%S2
\node[] at (0.2,2.9) {$S_2$};
\draw[ thick, fill=black!60!green] (-0.1,-0.5) rectangle  (0.3,0.5) node[pos=.5, white] {};
\draw[black,  thick] (-0.1,-0.5) -- (0.3,0.5);
\draw[black,  thick] (-0.1,0.5) -- (0.3,-0.5);
\draw[ thick, fill=black!60!green] (-0.1,1.5) rectangle  (0.3,2.5) node[pos=.5, white] {};
\draw[black,  thick] (-0.1,1.5) -- (0.3,2.5);
\draw[black,  thick] (-0.1,2.5) -- (0.3,1.5);
%Linac drawings 
\node[] at (4,2.5) {Accelerating Cavities};
\draw[fill=blue,  thick, rounded corners =0.1cm] (\loneright,0)rectangle  ({\loneright+0.84},2) node[pos=.5, white] {$L_1$} ;
\draw[fill=blue,  thick, rounded corners =0.1cm] (\ltworight,0)rectangle  ({\ltworight+0.84},2) node[pos=.5, white] {$L_2$};
\draw[fill=blue,  thick, rounded corners =0.1cm] (\lthreeright,0)rectangle ({\lthreeright+0.84},2) node[pos=.5, white] {$L_3$};
\draw[fill=blue,  thick, rounded corners =0.1cm] (\lfourright,0)rectangle ({\lfourright+0.84},2) node[pos=.5, white] {$L_4$};
\draw[fill=blue,  thick, rounded corners =0.1cm] (\lfiveright,0)rectangle ({\lfiveright+0.84},2) node[pos=.5, white] {$L_5$};
\draw[fill=blue,  thick, rounded corners =0.1cm] (\lsixright,0)rectangle ({\lsixright+0.84},2) node[pos=.5, white] {$L_6$};

%current optimization point
%\node[draw, fill=yellow, star, star points=5, star point ratio=0.6, minimum size=0.1cm]
%at (12.5,1.0) {$z_1$};


%Quad drawings
\node[] at (8.2,2.4) {Quads};
\draw[fill=black!60!green,  thick] (\quadone, 1.0) ellipse (0.2cm and 0.9cm);
\draw[fill=black!60!green,  thick] (\quadone+0.5, 1.0) ellipse (0.2cm and 0.9cm);
\draw[fill=black!60!green,  thick] (\quadone+1.0, 1.0) ellipse (0.2cm and 0.9cm);
\draw[fill=black!60!green,  thick] (\quadone+1.5, 1.0) ellipse (0.2cm and 0.9cm);

%Line between kicker and septum
\draw[very thick] (\lsixright+5.2,1.0) -- (12.5,0.7);

%Kicker 
\draw[fill=orange,  thick, rounded corners =0.1cm] (\lsixright+3.3,0.5)rectangle ({\lsixright+0.84+4.6},1.5) node[pos=.5, white] {$Kicker$};

%Septum
\node[] at (12.7,-0.8) {Septum};
\draw[fill=black!60!green,  thick, rounded corners =0.1cm] (12.2,0.9)rectangle ({13.2},-0.1) node[pos=.5, white] {};
%\draw[latex-latex] (\gunleft,-5.0) -- (14,-5.0) ;
%\foreach \x in  {0.3, 1.0, 3.5, 5.0, 7.0, 8.5, 10, 12.5} %tick marks
%\draw[shift={(\x,-5.0)},color=black] (0pt,3pt) -- (0pt,-3pt);
%\foreach \x in {0.3, 1.0, 3.5, 5.0, 7.0, 8.5, 10, 12.5}
%\draw[shift={(\x,-5.2)},color=black] (0pt,0pt) node[below] {$\x$};

%Line between kicker and septum
\draw[very thick] (13.25,0.2) -- (14.5,-0.5);

%Dipole
\node[] at (15,-1.7) {Dipole};
\draw[fill=black!60!green, thick, rounded corners =0.1cm] (14.5,0.0)rectangle ({15.6},-1.0) node[pos=.5, white] {};

%Line between dipole and quads
\draw[very thick, ->] (15.6,-0.5) -- (22,-0.5);
%Second set of quads
\draw[fill=black!60!green,  thick] (\quadfour+0.5, -0.50) ellipse (0.2cm and 0.9cm);
\draw[fill=black!60!green,  thick] (\quadfour+1.0, -0.50) ellipse (0.2cm and 0.9cm);
\draw[fill=black!60!green,  thick] (\quadfour+1.5, -0.50) ellipse (0.2cm and 0.9cm);

%Straight through
\draw[very thick, ->] (11.4,1) -- (30,1);

\def \quadfive{22}
%Third set of quads
\draw[fill=black!60!green,  thick] (\quadfive, 1.0) ellipse (0.2cm and 0.9cm);
\draw[fill=black!60!green,  thick] (\quadfive+0.5, 1.0) ellipse (0.2cm and 0.9cm);
\draw[fill=black!60!green,  thick] (\quadfive+1.0, 1.0) ellipse (0.2cm and 0.9cm);


%Witness
\draw[very thick, <-] (16,-2.5) -- (30,-2.5);

%Waveguide
\draw[very thick] (20,-0.5) -- (20,-3);
%Waveguide
\draw[very thick] (25.5,1.5) -- (25.5,-3);

%PETS2
\draw[fill=black!60!yellow,  thick, rounded corners =0.1cm] (18.5,0.0)rectangle (20.5,-1) node[pos=.5, white] {$\text{PETS}_2$};
%PETS1
\draw[fill=black!60!yellow,  thick, rounded corners =0.1cm] (24,1.5)rectangle (26,0.5) node[pos=.5, white] {$\text{PETS}_1$};

%ACC2
\draw[fill=black!60!yellow,  thick, rounded corners =0.1cm] (18.5,-2)rectangle (20.5,-3) node[pos=.5, white] {$\text{ACC}_2$};
%ACC1
\draw[fill=black!60!yellow,  thick, rounded corners =0.1cm] (24,-2)rectangle (26,-3) node[pos=.5, white] {$\text{ACC}_1$};


\begin{scope}[yscale=1,xscale=-1, yshift=-3.5cm, xshift=-29.5cm]
	\draw[fill=orange, thick, rounded corners =0.1cm] (\gunleft-0.1,0.5)rectangle (\gunright,1.5) node[pos=.5, white] {\textbf{Gun}} ;
	
	%S1
	\node[] at (-1,2.9) {$S_1$};
	\draw[thick, fill=black!60!green] (-1.4,-0.5)rectangle  (-1.0,0.5) node[pos=.5, white] {} ;
	\draw[black,  thick] (-1.4,-0.5) -- (-1.0,0.5);
	\draw[black,  thick] (-1.4,0.5) -- (-1.0,-0.5);
	\draw[ thick, fill=black!60!green] (-1.4,1.5)rectangle  (-1.0,2.5) node[pos=.5, white] {} ;
	\draw[black,  thick] (-1.4,1.5) -- (-1.0,2.5);
	\draw[black,  thick] (-1.4,2.5) -- (-1.0,1.5);
	
	\draw[ thick, fill=black!60!green] (-1.0,-0.5)rectangle  (-0.6,0.5) node[pos=.5, white] {} ;
	\draw[black,  thick] (-1.0,-0.5) -- (-0.6,0.5);
	\draw[black,  thick] (-1.0,0.5) -- (-0.6,-0.5);
	\draw[ thick, fill=black!60!green] (-1.0,1.5)rectangle  (-0.6,2.5) node[pos=.5, white] {} ;
	\draw[black,  thick] (-1.0,1.5) -- (-0.6,2.5);
	\draw[black,  thick] (-1.0,2.5) -- (-0.6,1.5);
	
	%S2
	\node[] at (0.2,2.9) {$S_2$};
	\draw[ thick, fill=black!60!green] (-0.1,-0.5) rectangle  (0.3,0.5) node[pos=.5, white] {};
	\draw[black,  thick] (-0.1,-0.5) -- (0.3,0.5);
	\draw[black,  thick] (-0.1,0.5) -- (0.3,-0.5);
	\draw[ thick, fill=black!60!green] (-0.1,1.5) rectangle  (0.3,2.5) node[pos=.5, white] {};
	\draw[black,  thick] (-0.1,1.5) -- (0.3,2.5);
	\draw[black,  thick] (-0.1,2.5) -- (0.3,1.5);
	%Linac drawings 
	%\node[] at (4,2.5) {Accelerating Cavities};
	\draw[fill=blue,  thick, rounded corners =0.1cm] (\loneright,0)rectangle  ({\loneright+0.84},2) node[pos=.5, white] {$L_1$} ;
	
\end{scope}



	\end{tikzpicture}
	\label{beamline}
\end{centering}

\vspace{2em}
The arrow at the end of each line indicates what direction the beam is traveling.
PETS stands for Power Extraction and Transfer Structure, and ACC stands for Accelerating structure. 
The subscript index on each structure refers to which stage the structures belong to, first or second stage. 
There are six accelerating cavities on the drive line, and only one accelerating cavity on the witness line.



}
%%%%%%%%%%%%%%%%%%%%%%%%%%%%%%%%%%%%%%%%%%%%%%%%%%%%%%%%%%%%%%%%%%%%%%%%%%%%%%
\headerbox{Acknowledgements}{name=doe,column=3,above=bottom, span=3}{
%%%%%%%%%%%%%%%%%%%%%%%%%%%%%%%%%%%%%%%%%%%%%%%%%%%%%%%%%%%%%%%%%%%%%%%%%%%%%%
%\begin{minipage}{0.5\textwidth}
	We gratefully acknowledge the computing resources
	provided on Bebop, an HPC cluster operated by the LCRC at ANL.
	This work is partially supported by the U.S. DOE, Office of Science, under contract number DE-AC02-06CH11357 and grant number DE-SC0015479.\\
}

%%%%%%%%%%%%%%%%%%%%%%%%%%%%%%%%%%%%%%%%%%%%%%%%%%%%%%%%%%%%%%%%%%%%%%%%%%%%%%
\headerbox{Non-Intrusive Polynomial Chaos (PC)}{name=conditions,below=beamline, above=bottom, column=3,row=0,span=3}{
%%%%%%%%%%%%%%%%%%%%%%%%%%%%%%%%%%%%%%%%%%%%%%%%%%%%%%%%%%%%%%%%%%%%%%%%%%%%%%
\noindent
\begin{minipage}{0.45\textwidth}
{\bf hold:} hold

\end{minipage}
\begin{minipage}{0.55\textwidth}


\centering
\vspace{1em}
\textbf{hold}
%\includegraphics[width=0.7\textwidth]{awa-medium-o4/sensresults}

\end{minipage}
\begin{minipage}{1.0\textwidth}
	\centering
	\vspace{-1.5em}
	\textbf{hold}\\
	%\includegraphics[width=.25\textwidth]{awa-medium-o4/suroresults-i-emit_x}%
	%\includegraphics[width=.25\textwidth]{awa-medium-o4/suroresults-i-emit_z}% 	
	%\includegraphics[width=.25\textwidth]{awa-medium-o4/suroresults-i-energy.pdf}%
	%\includegraphics[width=.25\textwidth]{awa-medium-o4/suroresults-i-dE.pdf}%
\end{minipage}	 
}
%%%%%%%%%%%%%%%%%%%%%%%%%%%%%%%%%%%%%%%%%%%%%%%%%%%%%%%%%%%%%%%%%%%%%%%%%%%%%%
\headerbox{Acknowlegements}{name=ref,column=0,above=bottom, span=3}{%
%%%%%%%%%%%%%%%%%%%%%%%%%%%%%%%%%%%%%%%%%%%%%%%%%%%%%%%%%%%%%%%%%%%%%%%%%%%%%%
\noindent 
We gratefully acknowledge the computing resources
provided on Bebop, a high-performance computing cluster
operated by the LCRC at ANL.
This material is based upon work supported by the 
U.S. Department of Energy, Office of Science, under 
contract number DE-AC02-06CH11357 and grant number DE-SC0015479. 
Travel to IPAC'18 supported by the Division of Physics 
of the United States National Science Foundation 
Accelerator Science Program and the Division of 
Beam Physics of the American Physical Society.

\vspace{1em}
\begin{centering}
	\includegraphics[width=0.3\textwidth]{logos/aps-logo}\hspace{5mm}
	\includegraphics[width=0.6\textwidth]{logos/DOE_logo_color_cmyk}
	
\end{centering}




%\end{multicols}
}


\end{poster}%
%
\end{document}
